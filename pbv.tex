% This file was converted to LaTeX by Writer2LaTeX ver. 1.4
% see http://writer2latex.sourceforge.net for more info
%
% And latex manual modificatin by Medini  yotam.medini@gmail.com
\documentclass[twoside,11pt]{book}
% \usepackage[ascii]{inputenc}
% \usepackage[T1]{fontenc}
\usepackage{geometry}
\geometry{
  paperwidth=6in,
  paperheight=9in,
  left=0.6in, right=0.3in,
  top=0.6in, bottom=0.6in}

\renewcommand{\chaptername}{} %% remove the word \chapter

\title{Pale Blue Valley}
\author{Yael Medini}
\date{2020-07-20}
\begin{document}


\thispagestyle{empty}
% \maketitle
{\nullfont dum}
\vspace{.3\textheight}
\begin{center}
{\Huge \textbf{Pale Blue Valley}}

\bigskip

\textbf{\Large Yael Medini}
\end{center}
\vspace{.2\textheight}

Sincere thanks to Mary Vaadia who was the first to go over my draft
English translation of the Hebrew book ``\emph{Balada La'teror}''
published in Israel in 2000 by G'vanim Publishing House.

\bigskip 
I am no less indebted to Valerie Arnon for her insightful redaction of the
draft English version. Her sensitivity with words and uncluttered style
contribute significantly to the novel's directness and flow.

\bigskip
{
\parindent=0pt
Yael Medini\\
Ramat Gan, Israel. \\
October 2019 \\
\texttt{yael.medini@gmail.com}
}


\chapter{}

'Here I go again, my love, saying that you haven't changed. Not one
little bit. Time after time I've said it. I'll say it till the end of
days. Yes, Saffia Saffia, my~darling. First and last name the
same. They used to call you Double Saffia - make a joke of it. No, you
haven't changed at all, my dearest, which is something you can't say
about me your eternal widow. I keep growing older. Older and older.
Today you wouldn't recognize me. This is what happened to our
severed~life.'

Rimat speaks these words voicelessly, gazing at the faded photo, then kissing it.
It doesn't do Saffia
justice because Boni took it just as he was getting up - face
puffy, hair disheveled,
eyes unfocused. But this is the only photo that survived the
disaster. How close were those two brothers -- ``Big-Saff'' and
``Li'l-Saff``.

She hangs onto the photo till she hears Glin's footsteps in the
evening quiet. When they get louder it's time to return the photo to
its place under the old cradle's small mattress. She tugs at a
corner of the cloth lying on top of the
mattress looking for where she left off. The yellow flower is almost
complete. How many more flowers will she need to complete the
garland? Five, she thinks, maybe six. Afterwards she'll embroider
another ring, perhaps even a third.~
How will this piece of embroidery end
up? 
A table-cloth, a bedspread - who knows? She can't remember when she
began working on it, nor does she know if she'll ever finish it. Who
cares? The days when she measured the passing of time by days, weeks, months, seasons, years - are long gone.

Outside Glin mounts the stairs
two at a time and bursts into the room, Walkman dangling on his
chest, earphones~slung around his neck. He lays a few discs on the
table, swiftly crosses the room and opens the window.
A women's lament drifts in from the rosy twilight.

``Hi, Glin,''
says Rimat,  ``it's cold outside -''

``And if I close the window \dots?''

``No, it won't be warm outside -''

``Ha-ha-ha!'' Glin forces a
laugh. ``Hi, mom -'' 

``And while you're at it, would you be good enough to switch on the
light - it's getting dark.''

``One~at a time, if~you don't
mind,'' says Glin. He closes the window and presses his face against
the glass. ``Who are those
two women singing out
there?''

``The two Balad women, I
expect,'' Rimat answers.

``What Balad women?''

``The ones who started working in the
village.``

``Come to think of it, I've never met a Balad -- man,
woman or child.''

``They came looking for work. Jobs were arranged for them in the
packing shed.''

``Gidal had a hand in it, no doubt -''

``I expect
so,''~Rimat says. 

``You
don't~expect, mom. You know.
D'you know them?''

``Only from afar.''

``So where are they heading~now?''

``They live in that shack up the hill,
past the bend.''

``In that hovel?'' Glin turns to his
mother.

Only now does Rimat notice a~new plaster
on his right temple. She doesn't ask about
it - that would only rub him the wrong
way. Instead she explains, ``It's been fixed up.''

That too is Gidal's doing, Glin is certain. Not wishing to annoy his
mother again he keeps quiet. She has a soft spot for Golden Hands -
 let the old boy enjoy it.

Rimat bends over the old cradle, her fingers searching for the
skein of embroidery thread under the soft mound of cloth. She pulls
out some white thread and examines it against the white cloth.
``Should I use white in the empty spaces? White on white?''

``Void on void.'' Glin turns back to the window and opens it
again. Meanwhile the twilight has deepened and the gloomy melody
filters in from~a distance. ``Mother, come see how beautiful they look
against the sunset -''

Rimat leaves off her embroidery, gets up with a sigh and goes over to
the window. 

``A picture postcard,'' she says. The singing momentarily gives way
to a tinkle of silvery laughter. ``I think they're~mother and~daughter. Would you mind shutting this window for the
night?'' Returning
to her armchair she turns on the light, then trails her fingers
against the violin case on top
of the bookcase. She
collects the fresh crop of flakes that the cover has recently shed and
puts them in her pocket. Later on,~in the darkness,
she'll let them fall from her hands into the trash bin
outside. She'll do it slowly, deliberately.~She eases herself back
into her chair and threads her needle with red thread. Up and down, up and down through the white calico she
plies her needle.

``If it isn't
too much to ask,'' she turns to her son,
``could you draw the blinds? The sun's going
down. You know I don't like to see the dark
outside.''

Not only does Glin
ignore her bidding, but he again turns off the light. 

``The light inside obscures
the dark outside.'' He returns to the window. ``Did you hear what I
said? I repeat: The light
inside obscures the dark outside. You're not impressed by the
symbolism resonating in these words. With all due respect, dear mom,
you do lack poetic sensibility. My `void on void' before also left
you unimpressed. Where are their men?
Making those bombs?''

Rimat's scissors slip from her knees to the floor with a loud clang. How can her son mention those horrific bombs so
lightly? It's her own fault - she didn't do enough to make him hate the idea of bloodshed.

``The guard's closing the gate,'' says Glin, drawing the curtains. ``He's limping, that means it's Moshko. The clang
of your scissors was the overture to the clang of the gate, a prelude. So those Balad women work in the packing shed,''
Glin commits this piece of information to memory. ``You're a mine of knowledge.''

''What will happen to you when I'm gone?''

``That won't necessarily happen. Death doesn't always go by seniority -- father, mother, child.'' Glin regrets his
flippancy. He's gone too far. Although he cannot truly grieve for a father he doesn't remember, he's deeply conscious
of his mother's everlasting mourning for her love - \ victim of a heart attack - snatched from life at its most
dynamic. He steals a glance at her. He's already been forgiven.

Rimat doesn't always understand her son's humor. Saffia, too, used to
say that a keen sense of humor wasn't exactly her forte. He forgave
her this deficiency just as he forgave her for others far worse. But
she will never forgive his friends nor his brother Boni for having
left him to his tragic fate.  Nor will she ever forgive herself.  She
should have trusted her gut and begged him not to leave the house
carrying those damned bombs. She should've shrieked to high heaven,
laid herself on the doorstep to stop him. Instead, she'd yielded to
fatigue. Nursing Glin, pacing the floor with him those sleepless
nights - she was exhausted. When the final hour comes - is this the
way she'll close her eyes, with anger still smoldering inside? A
thought crosses her mind - maybe now is the time to learn from Saffia
the art of forgiveness. If she can ever forgive his brother and his
friends she might be able to forgive herself too.

Glin asks, ``How long have they been working in the village?''

``Almost a week.''

``It's the first time I've seen them here -''

``At first they were told to use the new gate, but this morning they
began opening the old gate for them.''

``Why
the change?''

``You're a genius if you can't figure that out yourself. The new gate made them go a long way round
through the village and that took time. \ Using \ this old gate makes it a short walk. And you have to remember that
they are after all Balads. Not everybody's happy about that.''

''Why do people care? Anyhow, is this also
thanks to Golden
Hands?''

``Probably.''

Suddenly something from outside shatters the window and hits Glin on
the cheek.

``Glin!'' Rimat rushes to her son and tries to stem the bleeding
with a handkerchief.

``The daily gift,'' says Glin. ``If you hadn't insisted that
I close the window, I would have gotten only a little
tap.''

``Press hard with that
handkerchief,'' Rimat advises as she hurries to get
a plaster.

Glin bends down to pick up the muddy
package from the floor. Opening it he reads the attached note aloud:
``'Every bullet has its address.' How very
original.''

Rimat returns, dabs iodine on the cut and applies the
dressing.

Glin -- his face now adorned with two strips
- puts a closed fist
to his mouth. ``Tarrum-tarrum-tarrum!'' he \ trumpets. ``On the
one hand, mom, this is not one of your annual springtime letters from
your old friend that history professor. But on the
other'' - he pulls a folded scrap of paper from his pocket and hands it to her
- ``it sure is interesting -'' 

Rimat flattens the scrap of paper. ``Alir!'' she cries.

``Your old poet-friend, indeed -''

``It's his picture from the old days,'' Rimat says. Scanning the
text underneath she purses her lips. ``Where did you find this?'' 

``On the path coming home.'' 

Those Balad women may have dropped it accidentally or maybe on purpose, Rimat thinks to
herself. 

``Your stories about Alir made me imagine him as a
saintly soul hovering up above,'' Glin continues, ``And here he's so down to earth. But the
main thing is his poetic device.'' Glin is about to quote from memory beginning with the poem's title, ``Next -'' 

``Stop!'' Rimat shuts him
up. She can't bear to hear the terrible words. What happened to Alir?
- she asks herself - Alir, who wouldn't  hurt a fly?

Glin resents her intervention. All he wanted to do
was explain how overwhelmed he was by the third and last line -- how
its deviation from the poem's initial structure sharpened its meaning. Instead he says: ``The way he used to say that
in a moment he'd commit to memory the poem he'd conceived and then
jot it down in his notebook -'' 

``You remember my stories.'' 

``That's such an original way of putting it.''

Rimat doesn't tell him that when a moment of inspiration came over
Alir, Saffia would neither sing nor play his violin -- the moment was too precious. Those unforgettable times, living
from hand to mouth in that hole-in-the-wall. But Saffia insisted that you don't abandon a friend to
homelessness, so he'd invited Alir to move in with his brother Boni in the
glassed-in veranda, beds foot to foot. \ Glin doesn't even know of his uncle Boni's existence - formally known as
Bonimi -- whose life has not ground to a halt. Golden-Hands Gidal whose veins flow with the milk of human kindness, who
reads the papers, listens to the radio and watches television, shares with her bits and pieces he gleans about their
old friends. It's thanks to him she knows that Boni has become Chief Executive Officer of the Public Television
Network. Whatever that means. Not to mention the fact that she has no TV. But thanks to Gidal too she knows about the
calamity that befell Zakod and about Michlor who went abroad to pursue his medical career and became a world-famous
pediatrician. Gidal attributes the absence of any mention of Shouba on the media to his being an applied chemist. And
Alir's absence from the media? Well, how many poets enjoy the limelight? She returns the disturbing
scrap of paper to Glin and takes up her embroidery.

After folding the note
and putting it back in his pocket Glin dons his earphones and turns
on the Walkman. 

``No!!!'' Rimat stops her ears. 

Glin quickly switches it off. ``Sorry, mom.'' He comes over to pat her
greying hair.  When he reaches her age his hair too will be grey.  His
father died before a single hair of his

turned white - that's what his mom told him. She also told him that
his father was dashingly handsome and had an aura of grace about him,
that his speaking voice was mellifluous, that he had a lovely tenor
singing voice, that from the violin he had saved up every penny to buy
he produced divine music. Glin knows that he didn't inherit any of his
father's traits except perhaps the love of music. No, not
love. Passion. He presses a button on the Walkman and the music soon
has him under its spell.


\chapter{}

{Aera sits on the wooden bench
out}side{. }It's getting darker.
Soon{ Yamik will completely disappear into~the trunk~of the tree he's
leaning against. Maybe he's fallen asleep standing. When he }arrived he'd hugged her and Mother, kissed them both and
then fell silent. He looks tired out. Who knows what he went through till he got here? Who knows what lies ahead of him
when he leaves?

Mother emerges from the shack with a loaded tray. {{}``Eat, my hungry
son, drink, my thirsty son,'' she urges her
}{nephew}{, ``you must've been on the road for
hours.''}

{{}``Many thanks, auntie,'' Yamik says, ``but
you`re overdoing it. I'm not hungry }and not {thirsty.''}

\ {}``Because of what's on your mind,'' Mother sympathizes, seating herself on the doorstep. ``Even as a child you
didn't feel hunger or thirst when you had something on your mind.''~ She's afraid it's not only~Balad matters that
Yamik has on his mind. She fears \ something has come between Aera and Yamik. If her intuition is right - how dreadful!
She can't stop thinking about what she'd witnessed: the way Aera looked when she chatted with that
skinny{ }young fellow from the village as the two of them stood outside the packing shed. Aera's
hands~hung loosely at her sides, but her twitching fingers gave her away.~ She knows~her daughter like the palm of her
hand.~ Once again she turns to Yamik, ``Tell us something. Only what's fit for our ears of course,'' adding in a
whisper as if alluding to some mysterious vision,''did{ }you get to the Valley?''

{}``Not yet, auntie,'' says Yamik. ``It's not easy for us Balads to get there. But I will. As shall we all.''

{}``These words heal my wounded soul -'' she sighs.

Only in her imagination, Aera knows, can she remove Yamik from her life. But for the moment she must put that aside.
Pertinent issues have to be settled now between the two of them. ``Mother dear,'' she says, ``bed time.''~

{}``Since when does a daughter tell her mother it's bed time?'' \ Mother remonstrates.

Aera has her riposte ready: ``Ever since~wisdom passed from one generation to the next. It's late, Mother.''

Ignoring her, Mother turns to Yamik. ``Wouldn't you like some coffee, my son?''

{}``Yamik will tell me when he wants coffee. I too am capable of making coffee,'' her daughter insists.

{}``I blend special spices in the coffee,'' argues Mother.

Aera gets the point. Her wise Mother has sensed that something has come between her and Yamik. That's why she wants to
keep them from being alone together.

{}``Maybe \textit{you} should be the one to make yourself scarce,'' Mother adds stubbornly. ``I have things to say to my
nephew, to my daughter's husband who is like a son to me, and they aren't meant for your ears.''

{}``Like what, Mother dear?''

{}``About that skinny{ }lad from their village, for example -''

Aera releases a tinkle of laughter. ~``About that skinny lad from their village, for example, I shall~tell Yamik myself.
That story is mine.'' How lucky the allusion to Glin came from Mother. She's relieved. She feels Glin all around her.

Mother says, ``Time was when my daughter was mine.''

\ {}``Time was when my Mother was mine,'' Aera responds.

{}''God bless you, my son,'' Mother kisses Yamik on the forehead and disappears indoors.

{}''I didn't say, `Time was when my wife was mine','' Yamik remarks.

Aera doesn't quite know what to make of Yamik's words. ``Well, now you did{
-}{}'' she retorts, expecting that the business at hand is about to~be broached.

{}``Aera, you haven't done it yet.''~

Aera stares through Yamik into the dark hill behind.

Yamik \ adds, ``Don't put it off. I told you what'll happen to me if you don't do it. And soon.''

Mother's voice is heard from the shack, ``Daughter, listen to your~man.''

~``This is between the two of us!'' Aera calls back.

{}``Where's your father?'' ~Mother doesn't let up. ``Where{ }are your brothers
and sisters?''

Neither does Aera let up, ``Mother, \textit{go to sleep.}{}''

{}``Too much thinking is no good,'' Yamik tells his wife softly. ``It makes the actual doing
even harder. It also increases fear. Are you afraid?''

{}``No,'' says Aera -{ }but she hasn't the courage to tell him that fear has
become secondary. The big thing now is that she has ceased to feel at one with her assignment. She makes a suggestion:
``How about you separating from the men? \ Then you'd be free to do things on your own -''

{}``This is silly girl's talk,'' once again Mother's voice comes over. ``How much strength
does one person have on his own?''

{}``Mother,'' Yamik pleads, directing his voice to the shack, ``good night.''

{}``There are people who can do things by themselves, but I'm not one of them,''
Yamik{ }confesses to Aera who has to admire his candor and modesty. ``The guard
won't see that you're coming back'' he goes on, ``because it'll be pitch dark. To-night, when I leave, I'll dig a bit
deeper there. You'll have plenty of room to slide through there and back.''

Aera's heart skips a beat. Yamik will not stay the night!

Yamik feels that Aera needs some more encouragement, ``Let's go over the instructions again.''

{}``I know them by heart,'' she says.

{}``There~can never~be enough of~going over instructions,'' he insists. ``Mother and you go
out through the gate while the guard is busy with the lock. Mother keeps singing. You pick up those dry twigs that
you've~already tied together and slide back under the fence. The guard starts walking away slowly
{{}- }you said he limps. You wait patiently until he's very close to that~iron
peg protruding from the ground. That's \ critical. To wait patiently and not to act too soon. And then - \ only then -
one strong push! Then you walk backwards to the fence, continuing to sweep over your tracks using those dry
twigs{ }like a broom. You then slide outside. That's it. Done. They'll think
that he tripped{ }because of his limp.''

{}``And if he yells?''

{}``At most he'll cry out only once. Do it well and there won't be any problems.''

{}``And if there are?''

``There won't be.'' Yamik remembers the instructions he was given for \textit{his} first assignment. How
encouraged he was by the rugged determination of the experienced older man who gave them. ``And now, what's that about
that skinny lad from their village?''

{}``I'll make you some coffee and tell you,'' she says, moving towards the doorway. Yamik's question has~wiped away
Glin's {imminence}. Passing by him she carefully dodges his hand, but his voice pursues her.

{}``Do you know how to blend in Mother's secret spices?''

She transfers this question to Glin's lips and can't contain her silvery laughter. She knows Yamik loves that. He's
always telling her that it's how his aunt used to laugh - \ for she'd adopted him when he was a small baby before Aera
was born. But when Mother herself was struck by disaster she stopped laughing. Since then she has worn only black. That
too. He moves away from the tree-trunk and sits down next to the tray. Now the food begins to whet his appetite.

Aera reappears carrying a coffee pot in one~hand and a~cup in the other. She fills the cup and hands it to Yamik. ``I
blended in mother's~secret spices,'' she smiles.~

Yamik eats and drinks with relish. ``Why don't you eat too?'' he asks.

{}``I ate before you came.''

{}``And why don't you drink?''

{}``Coffee keeps me awake - and Mother and I have to be up early in the morning for work.''

{}``I have a long walk tonight. This coffee will keep me going.'' Yamik empties the small cup in one gulp, Aera refills
it.

{}``Now I'll tell you about that lad from their village,'' she says. ``He's different from the rest of them there. He's
taking his high school final exams right now. He already got the forms for their army service. He'll apply for
deferment on the grounds that his mother is a widow and he's an only son - but that's just a pretext. The truth is that
he objects to anything military, anything to do with armies, fighting,~wars. In his spare time

he works in a shop in the village that sells radios and all that. One day he came to the packing shed office to talk to
the boss about something and that's~how we met. Only one thing interests him -- music.'' Aera realizes there's no point
telling Yamik about Glin's {\ }original{ }sense of humor - he wouldn't get it.

{}``Music?'' Yamik repeats, a bit surprised.

~``He doesn't go anywhere without his Walkman and earphones.''

{}``What's his name?''

{}``Glin.'' Aera touches the spot on her hair~where Glin had kissed her when they happened to bump into each other on
one of the village paths. She was without Mother then on an errand for the boss and there was no one around. ``We talk
about nothing much. I thought it might be useful if he believed that I've got some{\dots} \ that I'm kind of {\dots}
attracted to him{\dots} That it might be helpful for us.''

{}``My Aera is the wisest of women,'' Yamik claps his hands softly. ``And why is Mother so bothered?''

{}``If I explained this to her, in no time it'd be all over among our people. And that'd be the end of this{\dots} plan.
``

Yamik is~impressed, ``Go ahead and keep on talking with him about anything under the sun!'' He empties his second~cup of
coffee. ``Delicious.'' He catches Aera's hand as she refills his cup third time round, kisses it, then gently caresses
her hair, ``In the whole wide world no hair shines like yours,'' he whispers.

Aera clenches her teeth - her hair is not for him to touch. Now it belongs to Glin who uses the very same words.~

Yamik, sated by now with food and drink, stretches his legs leisurely. ``Shouba -- the man I told you about -- set up
some kind of a center for us, to collect second-hand stuff, clothes, bedding, appliances etcetera.''

{}``People in this village also help us,'' says Aera, ``The work here is fine. And though this shack has obviously seen
better days, the roof doesn't leak and the wind doesn't blow in.''

{}``Is it so good~here that you forget about our Pale Blue Valley?''

{}``No. But it does make me remember that there are good people among them too.''

{}``Good and bad people are everywhere. That's not the point. And this business with the guard has to be done. My wife
has to prove that she's in it with us all the way.''

Yamik's life is fraught with hardship and danger but Aera's secret love bores deep into her heart. How she wishes she
were someone else. She wants to extricate herself from her predestined share in the Balads' fate, distance herself from
their cause. ``When will you come for a longer stay?'' Her conscience lashes her for a question that \ springs from an
ulterior motive.

{}``Depends on how things go. I might get a pass to stay overnight closer to work.''

{}``That'd be great.'' Aera's heart again misses a beat. Yamik's visits will then be even rarer.

{}``Absolutely. Life will be easier for me. The camp is a long way from town.''

{}``Is it your contractor who'll arrange for your permit?'' Aera manages to exhibit serious interest.

{}``No. He did try once but wasn't successful. Now I intend to ask this well-meaning Shouba of the Center to try for me.
What did you say his name was?''

Aera feigns innocence, ``Who?''~

{}``That skinny lad from their village?''

She has no choice, ``Ah{\dots} Glin -''

{}``Glin,'' Yamik commits the name to memory. ``By the way, do they know anything about me in the
village?''

``They don't``, Aera answers, ``they don't ask any personal questions and Mother
and I, of course, don't offer anything.''

``Let's keep it this way -'' Yamik says.

Aera sees that he's about to spread his coat on the ground. ``Don't you have to get going? Aren't they waiting for
you?''

{}``There's still some time.'' Yamik draws~her close and she cannot but give in. ``I have only two things always on my
mind, bombs and love. No, love and bombs. No, three things \ \ \ {}- love and bombs and a baby.''

Aera's shell-of-a-being is in Yamik's arms.{ }Her other self dissolves into Glin's.

~

\chapter{}

Zakod enters the storeroom and closes the door behind him. He lowers the shutters, fumbles his way to his desk, sits
down, turns on the reading lamp, puts a sheet of paper in front of him and starts to write:

{}``Rimat my true love,

Again it is the end of winter. Spring is in the air. In other words a year has passed since I last wrote. So once more
it's time to tell you, A: that you don't owe me any explanation for your silence during all these many years, and B:
that my arms are always open for you.

I hope you and Glin are~well.

By my reckoning Glin is going to be called up for the army any day now. I can imagine how you must feel about it. I pray
- metaphorically speaking of course~because the void around me is free of any entity of judgment and justice -- for his
safety.

Do you remember that it was I who suggested his name? How Saffia clapped his hands and said, ``Glin! It
sounds so clear and pure! Clear and pure like one unique {note of music}!''

Rimat, forever my beloved, there's been a change in my life. Two months ago I fixed up a storeroom~about thirty paces
from the house. I put in a chair, a desk and a reading lamp. This is where I sit now as I write to you. It is here that
shut myself in once a day. I lower the blinds and conjure up in the dark what would have happened
``if``.~ These thoughts~instead of dissolving into nothing get more and more fleshed out with
time. Rimat, you will not believe it. I -- the personification of reason and rational deduction --am carried away by
fantasy.~ Our Moiku and Blaya and Tiqvi are alive and well. They fill the house with their vitality. And not only is
Dela superbly healthy but as mother of our three daughters her position at home has become equal to mine. What a cruel
hand life has dealt her! Her total dependence on me makes the complexities of our relationship from the beginning all
the more difficult. \ If only it was me who was the half-paralyzed one sentenced to a wheel chair! That would be
justice. With her natural gift for devotion and solicitude she would b \ taking care of me.~

I find work on the farm salubrious and enjoy the physical effort outdoors that it entails.~ And that is~in addition to
the common idea that flexing one's muscles is the secret elixir for good health. These muscles of mine were almost
totally degenerate. Can you imagine that I do not have the slightest desire to open a book? I read only newspapers now.
And that too- only superficially. Here and there I listen to the radio and watch stupid programs~on TV. Ever since
HISTORY in capital letters happened to me, I say, LET IT BE DAMNED! This is what happened to the history professor who
lost his mind when the world around him was blown apart.

Rimat, my beloved for ever and ever, I am signing off.''

Zakod goes over the letter, here deleting a letter that somehow got doubled, there adding a comma. He attributes these
mistakes to the natural, age-related, decline~of cognitive faculties. \ He signs off, folds the letter, puts it in an
envelope which he addresses, and affixes a stamp in the top right corner. He will not hold it till he can post it in
town when

he goes there in three days' time to do the~weekly shopping. Instead, he will get into his jeep right away, drive to the
far end of the farm and wait there for the mail- collection van.

~

\chapter{}

A car~draws up outside.{ }

Shouba hurries from the kitchen into the living room, not removing his stained apron. He rushes round the roomful of
odds and ends, junk strewn about. He'd intended to make the

place look a bit more civilized in preparation for Boni's visit but~had naturally forgotten all about it. Now, of
course, it's too late. He rolls down his sleeves only to discover that the buttons have fallen off the cuffs. Before
rolling them up again he hears the doorbell. He runs to the door, grabs the handle and lets it go - twice. The third
time he steels himself to open the door. Though he was expecting Boni he freezes at the actual sight of him. Earlier
he'd sworn to himself that in their conversation he would find the right moment to mention Saffia's bomb-timers. Now
he's sure he won't be up to it.

``Hello Shouba.'' Bonimi~smiles warmly offering his hand, which Shouba takes~hesitantly.
Bonimi isn't too surprised: even in the old days Shouba was considered a bit dithery. \ Also, when at the end of their
phone conversation yesterday he had instructed~his secretary to fix this appointment, there might have been some
confusion about the date and time. ``Is this the right day and time?''

``Yes{\dots}'' Shouba stammers not knowing why.

``May I come in then?'' Bonimi asks good-humoredly.

``Oh, sorry, Boni,'' Shouba steps back into the room, clearing a path through the~clutter for
his guest.

~Being addressed by his old nickname affects Bonimi. It's been so long since he~heard it, so long that he's been known
by{ }one and all as Bonimi Saffia. He follows Shouba, apologising, ``Sorry I
couldn't get here sooner.''

~``You look so much like Saffia!'' Shouba puts a hand over his mouth, uncomfortable about the
way that slipped out.

~``A veritable spitting image,'' Bonimi quips. ``My brother was handsome and I'm
~far from it. He was tall and I'm short, he was well-built and I'm{\dots}well... plain fat. ''

``I'm sorry,'' Shouba apologizes, thinking: `Saffia's face was unblemished,
and{ }you have a nasty scar on your cheek '. \ How did that happen, he wonders?
Saffia's young brother was always so quiet, so withdrawn, never got into fights.

Bonimi stretches his stubby fingers, ``He had a fine hand, such delicate fingers - '' What's
all this gushing? -- he asks himself, but carries on, ``and when he played his
fiddle{\dots}''

The room descends into silence.

Shouba breaks it. ``I really appreciate your taking time to come here. Being CEO of the Public Television
Network is no small matter.''

Bonimi looks around and concludes with a genial wink, ``You evidently have no time for the
gasbag.'' He removes~his glasses and polishes them with a snow-white handkerchief.

``It's true I don't have much spare time,'' Shouba says shoving aside a heap of old clothes to
reveal an ancient TV set, ``but we do have to know what they want us to know.''

Bonimi accepts the barb with equanimity. ``It's eighteen years now since we saw each other.''

``Eighteen years?''

``Since Saffia -''

``Oh my God!'' Shouba exclaims.

~Bonimi attributes this outburst to the painful memory, but realizes he's wrong~as Shouba cries,
``Heavens!{ }I forgot!''
{\ }{}- and darts out of the room. He returns a moment later and explains,
``I forgot I left something on the burner in the kitchen.``~ He glances at Boni's face: as
placid as ever. \textit{He} hasn't become forgetful. Because he's younger? Not necessarily. It's because he's solid.
That was their family. Saffia too was solid, so very solid. Saffia Saffia. Given name and surname the same - they
called him ``Double Saffia'' for a laugh. He pulls a threadbare \ blanket off the one
armchair in~ the room and leans both hands on its armrests testing its stability ``Please -''
he offers Boni the chair then perches himself on a drawer-less chest of drawers.

Bonimi cautiously eases his thick form into the dubious armchair. This chaotic room in no way surprises him. Having
never come across Shouba's name anywhere all these years he has concluded that his older brother's friend \ {}``didn't
make it``. He would very much like to know what sparked this reunion. ``So how are things,
Shouba?'' he asks.

``Do you remember our old buddies?'' Shouba answers with a question.

``\textit{Your}~buddies,'' Bonimi corrects him, ``I was ~much younger than all of you. I looked up to
'Saffia's buddies'.''

Shouba's heart skips a beat at the memory. It's so long since he last heard it. He asks, ``And do you
remember Alir?''

``Of course,'' Bonimi answers. ``Didn't we live together at Saffia and Rimat's?
Slept feet-to-feet in{~} that glassed-in-veranda?''

``I forgot -''

``How's he doing? Still writing~poetry?''

~What luck that it was Boni who mentioned Alir's poetry! Shouba is encouraged. Now it's easier for him to go on:
``He published two collections... small{ }editions.''

``He was the first poet I ever knew,'' Bonimi recollects fondly. ``I shall never
forget the way he would say that in a single moment he would commit to memory the poem he'd thought up and then jot it
down in his notebook. What's he doing these days?''

``He's teaching.''

``At what university?''

``Not even {at} a high-school.{ \ }{In a} primary
school{\dots} and last month he got the sack -''

``Why?'' Bonimi asks. This, he surmises, must be the reason for Shouba's call.

``They suddenly discovered that he didn't have a teaching diploma. A pretext, of course -''

``Ah,'' Bonimi makes a shrewd guess. ``So the real reason was that poem he read
out~at the protest rally for the Balads?''

For Shouba the fact that his guest distinguishes between~a pretext and a genuine motive bodes well. It's going the right
way.

``Yes and no,'' he says. ``It was his poem, alright, but \textit{he} didn't
recite it. Nor was he at the protest rally - he's not the type for demonstrations. And he wasn't commissioned to write
the poem. It was inspired. It got published in one of the literary supplements, with his picture. And sometime later,
without asking him, the Balads made hundreds of~copies of it and distributed them as political leaflets - with his name
and photo.''

Bonimi nods understandingly,~``I come across \textit{your }name in every flyer promoting their cause, every
leaflet. I so admire people who struggle for pure justice.''

``Justice is justice, no adjectives needed.''

``Right. But as for myself - I don't preach,'' Bonimi smiles and continues in his mind,
\ {}'Nor do I like to be preached at.'

Shouba is angry with himself for having allowed their friendly chat to take this unwelcome turn. He shouldn't have been
so damned impulsive, so stupid. What should he say now for the conversation to revert to its amicable track?

Bonimi is engulfed with past memories. This kind of argument reminds him of those shouting matches between his brother
and buddies in the old days. It's terrible that this happened to Alir because of one short poem,`` he says. ``Just
because of{ }{a }mere three lines.''

Shouba is relieved - there's still hope. ``He's completely gone to pieces,'' he says and
decides not to waste another moment. ``I was wondering if you could perhaps be of assistance to him, if
maybe there's something for him in one of your broadcasts, say a series about poetry{\dots}?''

``I would be more than happy to help.'' \ The upcoming inauguration of the refurbished
auditorium springs to Bonimi's mind. Alir might fill a slot in the proceedings. That would be original.
``It may be an idea for the two of us to meet, talk it over.''

``Provided you don't give him false hopes -''

~``Naturally.''

~``And that he doesn't suspect that I have anything to do with it -''

~``I won't implicate you with the slightest hint,'' Bonimi winks again. ``He once
immortalized me in a poem. He described the way I wake up early in the morning when everything is milky white. Do you
remember why?'' Shouba shakes his head. ``Don't you remember that once upon a time I was a
milkman?''

``It begins to ring a bell. You used Saffia's bike.''

``The bike belonged to both of us,'' \ Bonimi wishes to be precise. ``Could you
give me Alir's phone number?''

``Here -'' Shouba tears off {a bit of an \ }old newspaper and jots it down.
\ ``And{\dots} that poem{\dots}?'' he asks concerned.

``Forget it.'' From the inside pocket of his tailored jacket Bonimi produces an elegant
leather wallet. He inserts the scrap of paper into one of its folds, takes out a visiting card from another and hands
it to Shouba. ``Here's my direct number.''

This card must not get lost, Shouba warns himself, but where will he find a safe place for it in this mess? He folds his
fingers round it. \ He also knows that the visit should end now. His esteemed~guest's time is precious.
He's{ }angry with himself for being too

cowardly to even hint at the other \ subject he wanted to raise. He stands up quickly, inadvertently hitting a cooking
pot placed on a bar stool, knocking it on the floor. ``Sorry.'' It isn't clear to whom he's
apologizing when he lifts up the pot.

Bonimi guesses that Shouba wants to wind up this visit, but unlike him he'd prefer to prolong it. ``So this
is where you live?'' \ he asks conversationally.

``And work.''

``I was quite excited when my secretary told me you were on the line,'' Bonimi warmly
confesses. ``Does Alir have a family?''

``No.''

``Me neither,'' says Bonimi. ``And you?''

``I'm not eligible. The stuff I handle, the experiments.''

``Now it all falls into place.'' Bonimi's dark eyes open wide with enlightenment.
``It was your photo in the papers reciting that poem at the rally. It said you're an applied chemist. So
you turned your hobby into a profession. Saffia said you mix all the components like one of those ancient
alchemists.'' Shouba lowers his eyes. Maybe now is the time? Yes, a kind of a-propos, but he's too late.
Boni continues, ``The sparks that \ flew sometimes when the lot of you \ got together.''
Bonimi feels the nostalgia welling up -- not like him. Meeting Shouba has brought it on. ``They continued
to fly even after everyone left{\dots} \ and Saffia and Rimat didn't stop arguing -''

~Shouba doesn't believe this. ``What about?''

``She wanted {him} to give it all up,'' Bonimi says. ``She said
'A married man who's going to be a father should devote his life to his family and that alone. Nothing else. And Saffia
insisted one could fill several functions and meet many obligations at one and the same time.''

``Saffia was that kind of person,'' Shouba nods. \ ``He adored Rimat, loved her
passionately{\dots} and he was also a very good painter and decorator, and had a beautiful singing voice, and played
the fiddle like a master. And{\dots} for the shortest time he was a father. And he was \textit{also} a freedom
fighter.'' After a pause he continues, ``They'd sometimes ask my opinion, and I'd tell them
that I had none. And Saffia'd say, 'If everybody had no opinion, like you, what would become of the world?' And I'd say
that if everyone was like me, nobody would go without milk.'''

Shouba is tickled.

``All three of us would also laugh then,'' Bonimi says in a quiet voice. He clears his throat
a few times, and finally pops the question that has been there from the moment his secretary told him somebody by the
name of Shouba was on the line, ``Do you know anything{\dots} about {\dots} Rimat?''

``No,'' Shouba shakes his head. ``Do you?''

``No,'' Bonimi says quietly. ``I heard that after the tragedy she went with the
baby - my nephew - to a village up north. Cut off all ties with me.'' Painful as it is, he must go on: ``The baby was
tiny then, maybe a month old. He had a lovely cradle, bought by Saffia at the flea market. And one of your crowd, who
was very handy{\dots}''

``Gidal!'' Shouba interrupts excitedly.

``Right! Golden-Hands Gidal. Now I remember. He replaced the old worn out handles with new ones. Oh, the
memories that surface!''

``I forgot the baby's name,'' says Shouba.

``Glin. That was his name. Glin,'' Bonimi says. ``It was your friend the history
professor who suggested it{\dots} I forget his name -''

``Zakod,'' Shouba supplies the name. ``What a powerful mind that man
had!''

``How true,'' Bonimi agrees. ``Being so young of course I didn't take part in
your conversations and arguments. But I listened to them. And Zakod made such an impression on me. He spoke slowly. And
his command of words! As if he was always lecturing to students. He also hated contractions. He wouldn't say 'I'm' but
'I am', not 'you're' but 'you are'.'' Bonimi regrets having dwelt on Zakod - he would prefer not to go
into the tragedy that befell Zakod and his family.

``Exactly,'' Shouba agrees, but the direction this conversation is taking is not at all to his
liking. He needs to clarify with his beloved friend's young brother the thing that weighs so heavily on him.
``I was close to Saffia because of our shared ideals,'' he says, believing he has thus found
the way to broach the subject.

``Yes, yes,'' Bonimi responds distractedly. ``If you should happen to hear
of{\dots} Rimat {\dots} anything -''

``Of course{{\dots}}'' Shouba inadequately leaves it
at that. ``Would you like to give your chauffeur a ring?''

``No need. I'll walk back to the office. \ A chance to use the old muscles, you know.'' This
time, when he extends his hand, Shouba grasps it warmly and prolongs the handshake. All inhibitions now forgotten he
finally comes out with it: ``You know the bomb-timers were absolutely ok. I couldn't be more sure of it.
For the life of me I don't know what happened. They weren't~faulty. I was not at fault.''

Bonimi~ feels the scar on his cheek starting to throb. He cherishes this~one and only souvenir his brother left him -
not to be shared with anyone. He sometimes feels he can never attach himself to a woman because then he'd have to
explain it -- and that's something he can't do. It's too intimate, too precious. Pushing away such thoughts he just
says, ``I'm sure you weren't.''

Boni sounds genuine - \ his words encourage Shouba to continue, ``I checked the mechanisms several times
myself and also together with Saffia. I'll never understand till my dying day{
}why the damned{ }bombs exploded before they were meant to -''

``You have nothing to blame yourself for,'' Bonimi says~emphatically. He waves his arms around
the room, ``May I ask what all this is about?''

``It's the least I can do for the Balads -''

``'The Center for Aid','' says Bonimi. ``The sign outside makes sense for me
now.''

``I didn't want to write 'The Center for Aid for Balads' - didn't want to spell it out. You can understand
why. What should be done for them? What do you think?''

Bonimi shrugs. ``When I was a milkman my only worries were that~the milk I deliver should be fresh and
clean and that~the bike's tires be pumped up. To-day I'm CEO of the~Public Television Network and~my only worry is
running it.~ Return to Pale Blue Valley? That's an idea I can't even begin to think about. Which marketing chain are
you connected with?''

``None. In the little lab in my kitchen all I produce is small quantities of household cleaning fluids and
detergents. I have an arrangement with some local shops. I'm not licensed to deal with hazardous
components.''

``Why?''

``The title~of `wild chemist' still clings to me.''

``The past does not let go, does it?'' Bonimi~chuckles then, after a moment's contemplation,
conjures up Alir's poem and recites it beginning with its title: ``Next.

Next~to a human being, I'm a human being / Next to a Balad, I'm a Balad / Next to a throwing hand / I'm the
grenade'.'' Shouba's moving lips having accompanying him he adds, ``I'm sure your rendition
at the rally was much better.'' He raises a hand in a parting gesture, opens the door and steps out.
Shouba closes the door behind him.

A resounding explosion rocks the house. Window panes rattle. Shouba rushes to open the door. Boni re-enters, brushing
dust off his elegant jacket. Shouba is horrorstruck. ``Are you alright!?''

``I am,'' Boni hastens to assure him. ``The bomb went off quite a long way away. The blast, though, covered
me with dust.''

``I had nothing to do with it,'' Shouba cries knowing full well his words are irrelevant.

``Come on, Shouba, I'm sure you don't meddle with those playthings anymore,'' Bonimi responds
calmly. \ ``But now, to be practical, I \textit{must }make this phone call.''

\ Before he locates it Shouba's phone is heard ringing somewhere.

``It might be for me,'' Bonimi says \ having no choice but to wait for Shouba to unearth the
phone~from under some tattered{ }blankets and hand him the receiver.

``Saffia speaking,'' he responds. ``Hello, Ezlip{\dots} I'm fine... alive and
kicking... Of course{\dots} Please let Janha know -'' By the time he puts the receiver down sirens of
police cars and ambulances shriek outside.

``The car will be here any minute,``~he tells Shouba. ``This is the world we live
in. And die in. For better or worse.''


\bigskip

\chapter{}

Oranges bounce along on the moving belt. Standing either side of it Aera and Mother pick out any imperfect fruit and
drop it in the box underneath. They work in sync, their hands briefly colliding now and again, at which Mother tut-tuts
and Aera \ releases \ a tinkle of laughter.

The crunch of bicycle tires on gravel is heard from outside. Aera's eyes meet her mother's for a fleeting second.

Glin~wheels his bike into the packing shed. He removes his earphones and smiles at Aera. She responds with a questioning
nod towards the new dressing he has - this time on his right cheek. His arm too is bandaged.

Mother purses her lips at this overt sign of their acquaintanceship.~~~~~~~

Goshu, the boss, opens the office door, ``Hi, Glin Saffia, is it all fixed up? Come in.''

Glin crosses the shed, leans his bike against the office partition and takes a small radio out of the saddle bag. Before
sitting down to explain to the boss what he has been able or unable to do he looks back to feast his eyes once more on
Aera. Instead, he's pierced by Black Mother's sharp glare. Why is she - forever clad in black - so angry with him?
Because Aera is a Balad and he isn't? An altogether meaningless difference.

``What's that for?'' Mother questions her daughter.

``Guess -``~ Aera teases, as she \ drops a few more perfect oranges into the box

``What will the boss say?'' Mother growls. ``Don't we have enough worries as it
is?''

``He's not as fussy as you, Mother dearest. He let me understand that we too may enjoy some perfect
oranges.''

Aera takes the full box, stands it by the office door and waits behind it.

Mother hisses through her teeth, ``And what are you up to now?''

Glin emerges from the office, to be impaled once again by Black Mother's glare.

``Boo-oo!''

~He looks round, startled. Aera's laughing eyes peer at him from behind a box. As always the happiness in her eyes is
marred. As ever, there is melancholy there.

``Some healthy vitamins for your cuts and bruises,'' says Aera lifting the box for him to
take.

``A cornucopia!'' As Glin relieves her of the box~their fingers touch.

``Aera!'' Mother prompts Aera from her spot at the conveyer belt

``Coming, Mother dear,'' Aera promises and whispers to Glin, ``She's afraid
-''

~``Of what?''

``Of everything.''

~``Same as my mom - \ something else we have in common.''

``Glin!'' Goshu calls from the office doorway, ``A minute?''

``Please, Goshu,'' says Glin, ``don't trouble yourself any more on my account --
it's no use.''

Ignoring Aera's presence the boss comes up to Glin and pleads with him, ``Haven't you had enough of these
cuts and bruises?''

``Not at all,'' Glin humors him, ``I enjoy them all to the full.''

``But seriously, my dear young man, with your skills the army would grab you with both hands. They're
looking for experts in sound-engineering. And that'll be a real stepping stone for you after the army!''

``And you'll have endless opportunities later, in civilian life!'' Glin follows
ironically{ }in the same vein, ``What with all the useful contacts you'll have made, your
career will be assured! ~Dear Goshu, I promise to think about it.''

``If only I could believe you -'' the boss shakes his head despairingly and
then adds, ``Congratulations, I heard that you passed the driving test - ''

``You see -'' Glin chuckles, ``I'm not a lost case altogether
-''

The boss disappears in his office without anothr word.

{\ }``What were you two talking about?'' Aera wants to know.

~``About nonsense,'' Glin answers her. ``But there \textit{is} something really
important to talk about. That song that you sing -- what is \textit{that} about?''

``It's about a place called 'Pale Blue Valley'.''

~``Is it real?''

``What do you mean real?''

``Is the valley real? Is the pale blue real? Is the song real?''

~The sadness only hinted at before now fills the \ young woman's face, ``Are \textit{you}
real?''

``Shall I come back when you're through?''

``We aren't allowed to remain here after work.''

``Aera!'' Once again Mother's fierce command is heard from her place at the conveyor belt.

``Just one minute, Mother dear!'' For a brief moment\MakeUppercase{ a}era turns to Mother to
reassure her and then immediately turns back to Glin. ``How d'you get all these cuts and
bruises?''

``Sheer carelessness -''

Aera points to the nape of Glin's neck, ``And this?''

``I lay on a sharp nail to deaden the pain of my longing for you.''

~``And the truth?''

``Some truths are~meaningless,'' Glin puts on a look of disgust, ``limited and
simplistic.''

``Not for~me.``~

``Where, Aera? \textit{Where}?''

Aera, lowering her voice, speaks quickly, ``Once you're out of the gate keep going along the fence as far
as the stream-bed. It's dry now. Cross it and climb up the opposite bank. There, facing you, is a .''

``When?''

``Two nights from tonight. But keep coming here. So my mother will still think we only see each other
here.''

Glin wheels his bike back through the packing shed taking care not to turn his head and risk Black Mother's piercing
eyes again.

Aera resumes her post opposite Mother at the belt.

``You learned to chatter in their language,'' Mother admonishes her.

``So did Yamik.''

``Not for chatter. For things that are important.''

``You'll yet be proud of your Aera, mother dear.~ Just wait and see.''

~

\chapter{}

``So what's new?'' asks Yamik, as the contractor climbs back in the cabin and sits in the
passenger seat.

``First let's get going,'' the contractor suggests, happy that his young foreman saves him
from having to navigate the city traffic. Once Yamik skillfully merges into the moving line of vehicles he asks him,
``What d'you want first -- the bad or the good?''

``First the good,'' Yamik responds.

``You always like the good news first, don't you?'' \ the contractor humors him.

``Who knows if I'll survive until the bad,'' Yamik replies.{ }

``I guess you're right, pal,'' says the contractor, ``but why so
gloomy?''

``Just a passing thought,'' Yamik says, continuing to himself - 'that doesn't leave me'. Three
of the men had been killed in the failed attack last month, and in the last roundup he wasn't arrested thanks only to a
negligent soldier who'd been fooled by his blind old Balad get-up.

``Well, the CEO seems to be all for giving us the project,'' the contractor is pleased to
share the good news with his \ trusty foreman. ``He's quite unique around these parts. He only thinks of
the main points. No ulterior motives, no ego games. I'd never imagine that our Public Television Network could have
such a CEO.''

``And the bad news?''

``The bad news has to do with his number one aide, a bigshot with an inflated ego who says there are a few
more things they still need to talk about.''

``Did you tell him about the renovations we've done at the museum?''

``Of course. And he said he'd go over there and take a look. Let's hope he's impressed.''

Yamik joins the contractor in mute prayer.

``They promised to have an answer within a week,'' says the contractor{,}
adding, ``but truth to tell, Yamik, I myself started having doubts whether we can handle it
-''

``Why not?'' Yamik asks, hoping not to give away his serious{ }concern about
such possible doubts. Renovating the auditorium in the television building means a close knowledge not only of that
specific structure but of the whole surrounding area, in fact the whole town. Most precious information. ~

``Aren't we going to over-extend ourselves?'' the contractor knows he can share his misgivings
with his foreman. ``Between the Eastern Hills Project and the town center?''

``The main consideration -- to my mind -- is to get a foot in the doorway and not let a good opportunity
slip by.''

``Were it not for you, young man, I wouldn't even consider it. Because if we do land this job we'll have to
split up for a while. I'll be here in town at the TV building site, and you'll be up there on the Eastern Hills
Project.''

``It'll be OK,'' Yamik assures the contractor hoping that in time he'll be able to convince
him to swap places. ~

``Oh, Yamik,'' the contractor pats his foreman's sturdy shoulder - the time and energy he's
invested in training him has borne fruit. ``You won't fall for any of those silly ideas like some of
your{\dots} people -'' he says awkwardly. ``You'll get ahead in life.''

``Thanks for the compliments and the prediction.''

``As for the renovation of the auditorium in the TV building -- there'll be~a tight schedule. They're
planning a big inauguration ceremony, the whole works including at least one VIP -''

``That's certainly one more reason for you to be involved,'' Yamik interjects.
``It'll give you a lot of free publicity. How much time do we have for the renovation?''
Yamik wonders if his ``we'' was wise.

The contractor, however, is pleased to hear that ``we'' - it means real collaboration. ``Four
months.'' he says.

The truck goes three quarters of the way around the remodeled roundabout and begins to ascend the newly-paved road~to
the site of the Eastern Hills Project.

``Whenever my son, who manages{ }to take a university course while he's doing his army
service, has to submit a paper by a certain date, he calls that date a 'deadline' because of the killing
stress.'' The contractor regrets his words. His Ouli and his foreman Yamik are about the same age -- but
how different their lives! Babies are born and their lot~is cast. How fortunate. How unfortunate.

This expression - ``a deadline'' - is new to Yamik. He makes a mental note. Aloud he ponders,
``Four months -''

``Is that a lot or a little?'' asks the contractor.

``\textit{It's four months}.''

The contractor points to the top of the hill their truck is presently climbing, ``Coming along nicely,'' he
observes. ``By the way, I'm so sorry I didn't \ succeed to get you a pass to stay there overnight.''

``There's somebody else looking into it,'' says Yamik. ``Let's hope
-''

``Let's hope, indeed -'' the contractor agrees


\bigskip

\chapter{}

Shouba is looking for Boni's business card. Finally he gets down on all four and finds it in a crate-full~of cast-off
toys. Happily it's perfectly intact, most probably thanks to the high-quality paper. He goes over to the phone and
dials the number. After one and a half rings~the receiver is picked up.

``Saffia speaking,'' Bonimi identifies himself in his formal yet pleasant voice - mentally
scanning the list of people entrusted with his personal phone number.

``Boni{\dots}?'' ventures Shouba.

Who could call him by his old nickname and with such familiarity, Bonimi asks himself , checking that the receiver is
plugged into the black phone between the red and the green ones. Then it clicks: ``Shouba?''

``Sorry to bother you -''

``Not at all,'' Bonimi responds warmly. ``Did you speak with Alir?''

``I did and he's back to his old self. But this call isn't to do with him. Do you have a
moment?''

``But of course -'' Bonimi hears the tap at the door and watches Janha enter and place a stack
of papers on his desk. She motions that she has something to say to him. Covering the receiver he tells her he'll speak
with her after he's done with the present call. She leaves the room and he uncovers the mouthpiece,
``Sorry, Shouba, you were saying{\dots}?''

``I haven't said anything yet. First of all I wanted to know whether you have a moment --''

``But of course --''

``There are two things -''

``One {\dots}?'' Bonimi though~pressed for time is encouraging.

``Yes. One . After you left, I couldn't remember if I told you that the bomb-timers I \ made for Saffia
were perfect~-''

``You certainly did, Shouba,'' says Bonimi sensing the birth of an itch on his scar,
``and I said that from the very beginning I didn't for a second doubt it.''

``I can't imagine what happened -''

``Shouba, it wasn't your fault.''

``What could've gone wrong there at the last minute?''

Bonimi asks himself why his late brother's friend keeps harping on the tragedy. Shouba cannot know that \textit{he} -
Bonimi - was the last person to be with his brother before the explosions. Nobody knows about that and so it must
remain. Shouba~ goes on and on about those bomb-timers he devised for Saffia because he cannot do otherwise. Bonimi can
sympathize with that. Too bad his itch will soon wear off - he cherishes its every moment. ~``And
two?'' he prompts Shouba, sorry he can't get the edge off \ his voice because he is indeed pressed for
time.

Shouba, sensing that vocal nuance blames himself again for his inability to keep himself in check. If he doesn't learn
to be careful he could block this channel of communication Boni has opened for him so generously. He says,
``Number two has to do with a young man who is a Balad -''

``What about him?''
~~~~~~~~~~~~~~~~~~~~~~~~~~~~~~~~~~~~~~~~~~~~~~~~~~~~~~~~~~~~~~~~~~~~~~~~~~~~~~

``He's the foreman of a big contractor in town who's presently involved in renovation of the auditorium of
your TV building and in the famous Eastern Hills Project. Anyhow, this young Balad would like to be able to stay
overnight in town. For obvious reasons he~needs a pass -''

~``Sorry, but I don't know anybody~in that area.'' Bonimi debates with himself whether or not
to stay with the subject, and decides in the affirmative. ``But, really, Shouba, why should you take
responsibility for such a thing? It's so risky.''

``This Balad is a quiet, decent, solid young man,'' Shouba continues ignoring Boni's
reservations. ``He sometimes comes to the Center, so I know him. He's married and is soon to be a father.
Right now he lives in a camp on the other side of town that was set up for Balads working in town. But it's quite a
long walk twice a day back and forth. I heard of two quite similar cases where they did get those
permits.''

``It's playing with fire.''

``But in any case they go through security and all that when they enter town -''

Bonimi restrains himself as best he can, ``And if he \textit{does} get involved with something? Even though
you may be only a small link in the chain it'll reach you. They won't let your Center for Aid remain open. And that
Center of yours is doing such an important job. Is it worth jeopardizing it? In any case, like I said, I don't have any
contacts in that area. But as far as Alir is concerned, I'm pretty sure it'll work out.''

``Well, thanks so much for what you did for him. And sorry for bothering you.''

``But you haven't, Shouba. \ Really. You haven't at all.'' Bonimi realizes it's his turn to
apologize. ``What I said about that Balad - that was just my personal opinion. For your own good. It's
been really good{ }talking to you.''

``Same here, Boni,'' Shouba says, happy that the communication channel is still open.
``Goodbye then -''

``Goodbye,'' Bonimi replaces the receiver. The connection~with Shouba remains open. So one of
these days Rimat might surface.

Replacing the receiver, Shouba tells Yamik who's awaiting the outcome of the phone conversation, ``He
can't.''

Yamik feels free to add, ``Nor does he want to.''

``And I don't know anybody else.''

``Thanks all the same for your time and willingness.''

Once again Shouba is impressed by Yamik's command of the language. Only his accent gives his Balad origins away.

~

\chapter{}

Dela's wheelchair is pushed back to the kitchen table. She's singing ``Happy birthday to
Moiku``. ~In her hands she holds a picture of a girl. How lucky~she'll never be sure whose face it is. All
three girls' lost their front left tooth at the same age - six and a month. All three of them had this round face with
plump cheeks framed by golden curls. And that gingham dress with the white baby collar was handed down from Moiku to
Blaya to Tiqvi. On each of their birthdays she holds in her hands the only picture that survived the fire.

\ ``At the time we didn't feel we should write a name and a date on the back,'' she remarks to
Zakod seated opposite

``Sing some more,'' he says, his eyes directed below the table. The last repair the carpenter
did under his instructions proved a success. Dela's lifeless feet stopped bumping into the table legs. He raises his
head again. A sumptuous birthday cake decorated with colored frosting and a myriad of brightly burning ornamental
candles~stands on the table between him and Dela. He'd set out the three alternatives on the first birthday they
celebrated: no candle at all, a single candle, many many candles.~ Dela opted for the third. Why should he be the one
to lay down the rules? - he later chided himself. Did~they not share equally in the conception of each of the three
girls? Did she not carry each of them full term? Did she not suffer childbirth three times? Did she not suckle each
baby day and night? And then later, all along, had she not done far more than he did? But from the very beginning,
because of their different natures and their incompatibility, it was always \textit{his} say that counted.

~Dela dares to shake her head. ``Please, once is enough,'' she pleads. She doesn't say that
even already at age five Moiku said exactly that. To-day, at fifteen, she might not have agreed to have the birthday
song sung for her even once. But, perhaps on the contrary, she would have started to mature and mellow? Dela gazes
tenderly at the small flames, happy to have gone for the third option - the mass of candles is gaudy, but they do
remove the gloom.

Zakod begs, ``Please, yes.''

``Let's sing together,'' Dela gently suggests.

``I'll hum along,'' Zakod concedes, ``you know I cannot carry a
tune.'' He could never abide those inane words of the birthday song. But he knows all too well what made
him so amenable today - it's Rimat's letter. Rimat was always teasing him about his tone-deaf singing - which would get
Saffia really mad at her. Till she stopped. The song concluded he slides the cake closer to Dela for her to blow out
the candles. At night, after bedtime, he'll return the cake to its box. It's a sweet cake made with fatty ingredients
and too much sugar - not good for either of them.

Dela asks, ``Who will you give it to this time?''

``To the party for orphans of fallen soldiers,'' says Zakod~as he wheels Dela opposite the TV
set and turns it on. He then prepares their supper, setting the table for two, every once in a while glancing at the
screen. When the hero and heroine finally walk hand in hand into the sunset he turns it off, then wheels Dela to the
sink.

``Already?'' Dela dares to express some annoyance. She wanted to hear the music till the end
and see the list of the cast \ and the credits.

``For all practical purposes the movie's over, and soon the news will be on. Remember what the doctor said
- that it is not good for us to watch the news.'' Using the first person plural has become second nature
with him in conversations with his wife. He helps her~wash and dry her hands and wheels her back to her place at the
table. ``Bon app\'etit.'' He slips into her hands the special~knife and fork he found in the
rehabilitation section of the army surplus store.

``Thank you,``~ Dela says and her heart goes out to him. Not enough that he puts so much
effort into preparing the food -- he also has to eat the tasteless fruits of his labor. ``Mm, this is
good!'' she compliments him.

``Is it hot enough?''

``It's just right.''

``I'm going into town to-morrow``.

Dela is quite conscious of her husband's stress the day before his weekly trip into town. On those mornings he prepares
a sandwich and a drink for her lunch. As he leaves he never fails to ask her without turning to face her, 'Anything
else you'd like from town? Apart from what's on the list?' \ {}'Nothing, thanks,' she always replies. Then he opens the
door and with a perfunctory word of farewell leaves the house.

Zakod begins to clear up, weaving back and forth between the table, the sink and the fridge. With his back to her as he
starts on the dishes he addresses his wife. ``A while ago we spoke about having some help at home, that it
would be a good idea. But I could not find anyone suitable. Now, at long last, I have. It is a woman about our age, a
widow.~ I don't need her to send me recommendations. I can tell by the way she presents herself in
writing.'' \ He assumes, correctly, that Dela will not ask to see that letter. \ ``She made
two conditions,'' he goes on, ``one,~that the place of work should also provide accommodation - which is
not only possible but would suit us very well too.~ She could move in downstairs.''

``That's very good,'' Dela responds, thinking: very good for which of us?

``She has a son aged about eighteen,'' Zakod continues, now turning to face Dela,
``so her second condition is that her son stays with her. He could help me on the farm.''

``Excellent!'' Dela responds sincerely.

{}``The son is now in the middle of his final matriculation exams. They will come when he has finished.''


\bigskip

\chapter{}

``Glin!'' Rimat is shocked at the sight of her son, his face again bruised and bleeding.

``It's nothing, mom,'' \ he'll not have her worrying about him. ``I just tripped over, stupid
me.''

``And what kind of a fool do you take \textit{me} for?'' His arm too is bleeding and his
sleeve's torn. ``Go wash your face.'' She hurries after him to the sink to clean and dress
his wounds. ``The moment you're through with those exams we say goodbye to this place, get away from these
hooligans {\dots}What's this now?'' she runs to the window. Sirens! An ambulance~is climbing the dirt road
from~the old gate. People milling around it add to the commotion. ``What happened?''

``They found Moshko lying on the ground near the gate. Dead.''

``What?!``Rimat is aghast{. }``Yesterday evening, when you looked
out of the window at those two singing Balad women, you said afterwards that you'd seen Moshko fall down and then get
up -''

``My mistake because it was dark outside -'' says Glin.

``What do you mean?'' Rimat \ asks.

``Seems it was someone else who got up,'' he says.

Rimat's first reaction is to feel sorry for Moshko. Then she remembers the grievance she's held against him these
past{ }eighteen years. He was the one who stopped her at the village gate questioning her so nastily
about who she was coming to see. When she told him it was Gidal, he said he'd go and check. He'd slammed and bolted the
gate in her face, one of her hands gripping the handles of the baby's cradle, the other holding the violin case, a
rucksack full of clothes on her back. Next moment she was sorry for him again, knowing his wife had died not long ago
and since then there was no one waiting for him at home, no one who'd notice his absence all night. By the fourth
moment she's getting worried. ``Did you tell anyone what you saw - or what you think you
saw?'' \ she asks.

``No.''

``Good. Why get involved? Aren't we out of here -- practically speaking?''

Meanwhile the ambulance has driven past the window on its way to the village center. The sirens have gone quiet. Rimat
turns again to Glin and adjusts the bandage on his arm.

``Mom, did you ever hear of a place called 'Pale Blue Valley'?''

``I think the Balads say they came from there -''

``'Pale Blue Valley','' Glin repeats the name. ``Sounds like a legend, a
myth.''

``Doesn't take much to make up names for places or ideas,'' Rimat says. ``And
didn't \ I already tell you a thousand times that it's dangerous to be tempted by legends and myths? I don't mind
saying it again for the thousand and one time.''

They both know who's knocking at the door. Still, Rimat calls out, ``Who's there?''

Gidal opens the door just a crack. ``May I come in?''

``Can't you come in like a normal human being?'' Rimat responds with slight
impatience{. \ }

Gidal takes two steps inside and then stops, aware of Glin's presence. He asks, ``Did you hear about what
happened to Moshko?''

``Yes,'' says Rimat. ``Terrible!''

``They think the Balads had a hand in it,'' he says. ``They say that the police
are going to interrogate the two Balad women who work in the packing shed. Tell me, could a woman do such a thing?
Blaming people for nothing. I myself often saw Moshko stumble and fall over. And this time - very very unfortunately! -
he fell on that iron peg sticking out of the ground. I once tried to pull it out but I couldn't.''

Rimat decides that for her own and Glin's sake she should add, ``Yes, I too often saw him
fall.''

``And you,'' Gidal looks at both of them, ``did you see anything? Hear anything?
You're the closest to this old gate -''

``We didn't,'' Rimat shakes her head decidedly.

``I expect they'll want to put a few questions to you too,'' Gidal feels he should prepare
Rimat for any unpleasantness he won't be able to prevent. To be honest, he should also tell \ that to Glin - but he
knows that as far as this youngster~is concerned he doesn't amount to much.

``Let them,'' Rimat says. ``Let them ask questions to their heart's
content.'' Gidal's incessant solicitousness~has become a damned nuisance. But she's so indebted to him for
all he's done for her. Yes. On the one hand she can understand why Glin can't take him. On the other, she'd rather have
him be nice to this old friend of hers, this kind man who is blessed \ with golden-hands.

``The final \ exam in history is in half an hour,'' announces Glin. ``Mom, won't
you wish me something?''

``Good luck -''

``What an original creative mind!'' Glin clicks his tongue and leaves.

Now, alone with~Rimat, Gidal feels at ease. He makes himself comfortable in the armchair without being asked.
``I always had a soft spot for Moshko,'' he says. ``He was on duty, remember,
the night you arrived. He came to my place and told me there was a woman at the gate. She's coming to see you, he
said.''

``Right at the beginning I should have made it clear to you that~my life with Saffia was not over, that you
should have no illusions,'' Rimat speaks softly. ``I shouldn't have put it off. But I was afraid that otherwise you
might not help me settle here with the baby.''

``Rimat,'' Gidal caresses her with his bovine eyes. Now is the time to tell her what has been
on the tip of his tongue for quite{ }a while. ``Listen, Rimat, everybody knows that Glin's
deferment is only a way of buying time. Until he can get out of the army altogether -''

``To you I can say that's true.''

``Don't you two know that the army offers boys like Glin wonderful opportunities?''

``Wonderful opportunities -'' Rimat sneers. ``All kinds of opportunities whose
final aim is to kill. I brought Glin up to abhor all manner of killing. And I succeeded.''

Unable to argue with this Gidal looks for something to do with his hands. He picks up the cradle and checks the handles.
Still ok. He's modestly proud of his handwork.

Smiling warmly Rimat finds a way to \ compensate him, ``I'll never forget how you replaced the old ones
with these. So clever.''

``When are you leaving?''

``The day after the last exam.''

``I{\dots}I've renovated my place{\dots}'' Gidal stammers,~'' it's really
comfortable{\dots} now - ''

``What are you talking about? \ That Glin would leave and I'd stay here?''

``He isn't a child anymore.''

``You know that he's like a bond{ }with me.''

``And when {\ }\textit{he'll} want to leave \textit{you}?''

``When that happens, it'll be \textit{his} decision.''

She has an answer for everything, Gidal realizes, but still whispers, ``May I ask{\dots}?''

``Where we're going to pitch camp?''

Gidal nods.

``Remember Zakod?''

``That history professor?''

``Throughout all the years he's written to me every once in a while,'' Rimat owns up but
unwilling to hurt Gidal \ she doesn't specify that it has been{
}{punctually}{ }every spring. ``He's always offered his help. I never
responded. Some weeks ago I wrote him and explained our situation. Well{\dots} once upon a time you made it possible
for us stay here thanks to{ }the goodness of your heart. Now it's the same thing at his
place.''

``That shocking disaster {\dots}'' Gidal{ \ \ }shudders at the
{\ }memory. ``It was so like him to get the best~defense lawyer for those two Balads -
described as penniless. He always had to \ argue objectively no matter what. Thanks to him the two of them got off
because of \ {}'a shadow of doubt'. What did Saffia see in him?''

``What did Saffia see in \textit{you}?'' Rimat responds and hastens to add,
``What did Saffia see in \textit{me}? Saffia's eyes were big and his heart warm. In each one of us he saw
part of his own{ big heart}.

``If I'd been able to overcome my panick - hearing about his plan about the bombing, I'd have gone with him
to the citadel and been killed there with him.''

``But you might have saved him!''

``You'd never stop blaming me.''

``Never. And neither shall I ever forget that had you not panicked in time, {\ }that had you
stuck with Saffia, I wouldn't have had whom to turn to with my baby.''

Gidal grasps Rimat's hands, almost kisses them. ~


\bigskip

\chapter{}

Aera clearly doesn't want to talk about it. But Yamik, sitting with her on the bench outside, can't help himself.
\ ``The men know you did it,'' he says quietly.

A shudder runs through her ``Sh{\dots}'' \ she hushes him, grateful to Mother for keeping
silent as she prepares supper indoors.~ Ever since the \textit{act}, Mother~has honored her request to keep quiet about
it.

Yamik moves closer, takes Aera's hand, covers it with kisses.

``The main thing is that you did it and you did it well,'' he says.

Aera refuses to listen to him, again stops him. Pre-empting{ }\ his imminent embrace she pulls her hand
away finding an excuse to stand up - she must stop the shutter's rattling.

With Yamik present Mother allows herself to break her daughter's request. She appears at the doorway waving a wooden
spoon, ``You must always remember why we three are the only survivors!''

``Let us be, Mother dear,'' Aera entreats, attributing Mother's loose tongue to Yamik's
presence.

``I'm old,'' Mother continues, ``I'm weak. You two are young and strong. You two
know what you should do.''

``And we'll do it,'' Yamik promises, relieved when Mother disappears inside the shack leaving
the two of them to themselves.

``Don't you ever~ask me to do such a thing again,'' Aera hisses.

``There won't be any~need.'' Yamik sincerely wants make her feel easy. ``You
should've heard what the men said about my wife --''

``I don't want to hear!'' \ Aera shuts him up.

But Yamik can't stop himself - he so loves quoting the words of praise heaped on his wife when her \textit{act} became
known. ``That she's not only brave but is also capable and clever. I wanted to say, 'and also beautiful'
but I didn't. Your beauty is purely ours and ours alone.'' ~

Yamik's loving words are an anathema to Aera. She's on edge. ``What did \textit{you} do to be
accepted?'' she asks sarcastically. ``No-no, don't tell me. I don't want to
hear.'' She's silent for a moment and then threatens, ``If you ever do such a thing again
-''

``I won't,'' Yamik stretches an arm to put round her. Aera diverts it with a handshake,
``Promise. Swear.''

``I promise. I swear. \textit{Where it's up to me}. Don't you understand that if we want~to achieve our
goal, it is sometimes \textit{necessary}? Sometimes we have no choice?''

``If that's the case maybe it means we should give~up.'' The words just popped out -- she
doesn't know how.

{}``What do I hear?'' Mother shouts from indoors. ``Give up?''

{}``Auntie, please,'' Yamik begs.

{}``Why not stay away from the men~and be independent?'' Aera hurtles on not knowing where she's heading,
``One has to think for oneself, do things alone.''

``Yamik, my son, something's happened to our Aera,'' Mother reappears at the window.
``My daughter \ {}- your wife - \ doesn't understand how these things are done.''

Yamik turns to Mother again, affectionately, ``Auntie, please -''

Aera \ calms down, ``I think to myself, why don't they accept our demand? Perhaps because they don't
understand it. If they did, if \textit{many} of them would, then perhaps they'd come round and change their
minds.''

``Right,'' Yamik agrees -- for the sake of peace.

``We must explain what happened,'' Aera goes on, ``what we demand. Our legitimate rights.
Justice.''

Yamik would so like to go along with Aera. ``I'll put it to the men,'' he says.

``Don't!'' Aera jumps at him. She can't understand what induced her to speak this way to
Yamik. Is she trying to convince him to stay away from the men because she -- for her own reasons --wants to part from
him? Doesn't she know that these are two wholly separate issues? How painfully aware of it she is! But now it's too
late. She rushes ahead, ``The men won't understand it, won't accept it.~ They won't agree.''

~``You're right,'' Yamik nods, not knowing what else to say. ``That nice \ man of
theirs,'' he changes the subject, ``the one that set up that Center for Aid for us, couldn't get me a permit to stay
overnight on the site of the new building project I told you about. But I stay there~all the same and nobody knows. And
it suits my contractor very well because we have to split up for some time. He's supervising~the renovation of the
auditorium in the TV~building in the middle of town and~I'm in charge of that~new project.''

``So at long last your contractor~did get that job with~the television authorities?''

``He did. I forgot to tell you.''

Aera has a brainwave, ``Perhaps you can switch places?''

``What for?'' Yamik is intrigued.

``There's a TV set in our boss's office,'' she says, feeling Yamik's eyes looking deeply into
her own. ``It's always switched on. Sometimes, when I'm waiting until he's free to speak with me about
whatever he asked me to come in for, I watch~it. Movies, news, all kinds of programs. Just imagine that hundreds of
people, thousands, tens of thousands of people watch and hear what's broadcast. Food for thought, isn't
it?``~ What she had blurted out earlier to displace the oppressive feeling dogging her about her and Yamik
has now emerged as a good idea

``My Aera -'' Yamik is excited. He knows about different programs broadcast on television.
He's watched them in places where he worked. Aera's idea is a challenge -\newline
 though heaven knows how he could put it into practice. But when he does, he'll prove his independence of thought and
action and win her admiration. He's aware of the way she proved her own creative thinking{. }through
cultivating the friendship between between herself and that lad from the village. He asks jokingly. ``And
what's going on with that skinny one? Who's only crazy about music?''

``He's got one foot outside the village.''

``Why?''

``Doesn't want to sign up for their army. And they're giving him a hard time. Calling him names, pushing
him into thorny bushes, throwing things at him - even stones. ~He told me that he and his mother will soon be leaving
the village and move to somewhere near town.''

``Does he{\dots} still have a soft spot for you{\dots}?''

``Seems like it -''

``Keep it up after he leaves the village.''

``How?''

``By mail.''

``Mother will hit the roof if she sees that we're \ writing to each other -''

``Tell him to bring his letters to that Center for Aid in town. You know where it is? In the new shopping
mall. \ Opposite the supermarket there's an apartment building on the left with three entrances. Next to the middle
entrance there's a sign: 'Center \ For Aid'. And that's also where I'll leave your letters for him. I'll be your
go-between mailman.''

~

\chapter{}

Bonimi's ~desk is covered{ }with sketches detailing the renovation. He and his aide Ezlip pore \ over
them, helping each other ~decipher them by locating the~auditorium's demolished walls, the newly built walls, the angle
of the stage, the expanded seating area that would double the number of seats at the front from fifty to a
full{ }hundred.~

{}``It's thanks to you that this renovation is going so well,'' Bonimi{ }praises his aide.

``Thanks for the compleiment, sir,'' Ezlip responds , ``but the contractor
deserves a lot of credit too - he's really topps. And this morning he told me about a slight change. Taking into
account that he's going to be busy with the Eastern Hills Project for some time, his foreman will be in charge here.
~He's a Balad, but the contractor swears{ }by him.'' Observing that the CEO is studying the
sketch of the lobby, he adds, ``he thinks your suggestion to{ }have the information desk
against the wall was nothing short of genius.''

``No need to go overboard,'' Bonimi smiles.

But Ezlip continues in this vein, ``this change, first of all, enlarges the lobby physically, and secondly,
it gives a sense of space.''

``I confess that I didn't think about the 'secondly','' Bonimi says. ``And that's
not false modesty.''

``I have to tell you, sir, you have a terrific grasp of sketches and drawings. Very few - if any -
non-professionals have that.''

``Oh, please!'' Bonimi is wondering what all this flattery's about -- what's his aid \ after?

Ezlip, while sensing the CEO's reservations, remarks casually, ``in 'Who's Who' I discovered that you used
to be a photographer.''

``One of my sins,'' Bonimi smiles, dismissively waving his hand. He knows Ezlip can't help
looking at his scar now and again. On more than one occasion he's mentioned names of reputable{ }plastic
surgeons he could consult. Bonimi has always thanked him, said he'll look into it, but of course didn't.
``That was how I got my foot in the TV world - as a photographer,'' he adds.

``You probably have a treasure-trove of pictures somewhere in an old shoe box -''

``I do admit to an old shoe box. As for a treasure-trove - that's doubtful.``~

``It's just an idea I have - that along with the inauguration of the new auditorium we could have a
retrospective `Past and Present' exhibit - photographs of people who are part of this place taken at different stages
of their lives.''

``Excellent idea, indeed. But very sadly I~have nothing to contribute from the past.''

``Nothing?''

``At the time I gave all the photographs of any historic value to the Institute of Historic
Documentation.'' \ Bonimi is sure the wave of sadness that{ }almost paralyzed him has escaped Ezlip's
notice. \ He knows exactly which photos did remain in the old shoe box. Every once in while he looks at them. Saffia,
eyes closed,, playing the violin. Rimat, washing the dishes, making a face at him. Baby Glin half smiling half yawning.
Alir - holding an{ }\ eraser between his teeth - immersed in his notebook. Two pairs of shoes -- one his
and one Alir's -- on the floor of the galssed-in veranda. A textbook of Zakod's entitled ``Old Versus
Obsolete'' in the middle of the table. Saffia's overalls hanging out to dry, his paint-spattered ladder
leaning against the wall, some cans and brushes next to it. ~Saffia propping up baby Glin on the bicycle seat
and~Rimat's fingers keeping his small body from slipping off. ~Her scream, 'Careful, you loony!' echoes in his ears.
What's missing in this old collection is the photo he once took of Saffia waking up from a nap - face puffy, hair
disheveled,{ }{eyes
unfocused.} ~Rimat loved that photo and appropriated it. He'd planned to print another picture from~the negative, but
the tragedy devastated everything before he got around to it, and the negatives were lost forever. ``What
about the ceremony?'' Ezlip asks. ``Is it finalized?''

``Alir -- the poet - is still working on it,'' Bonimi answers. ``It'll take him a
few days.''

From the very start Ezlip had been against the choice of Alir to emcee the ceremony. He gave the impression of being a
scatterbrain - in the middle of the proceedings he could simply lose it. But the CEO had thrown his full weight behind
his choice -maintaining~that appointing an MC who is a poet and not one of those~slick professionals would help rid TV
of its stuffy image.{ }~Ezlip now finds a way to avoid denigrating Alir as a person and simply remarks,
\ ``His notorious poem did not increase his popularity.''

``There's no need to blow that poem out of all proportion,'' Bonimi responds.
``What really~happened when all's said and done? The Balads and their supporters let off a bit of
steam.''

Ezlip doesn't~deny this but the expression on his~face clearly conveys lack of cnsent. He collects the renovation
sketches from the CEO's desk and puts them in his briefcase. ``And why did I think of Who's
Who?'' he asks and answers, ``I heard that in one chapter of our past history your brother
was a freedom fighter. I wanted to read about him. So I looked up all the Saffia's in the index. But I didn't find his
story. What was his first name?``~

``Saffia. It was both his first and last name. His friends would sometimes pull his leg,

\ call him 'Double Saffia' -'' Bonimi feels his scar~begin to itch. He's dying to touch it but \ doesn't of
course. How it bled then together with Saffia's blood, the earth soaking up the blood of both of them.
``It was a short, complicated, terrible episode in what was called the Phony Occupation.''

``They give it about two lines in \ history books,'' says Ezlip. ``How do you see
it now?''

``A wrinkle in history. My brother wanted to straighten it out.''

``How?''

``Do you know the Citadel?'' What's he after? - Bonimi asks himself - this highly intelligent,
calculating, ambitious young man?

``Of course I do.''

``The occupation force had its headquarters there. My brother blew up one wing of their building. And was
one of its casualties. It was a tragedy for the family, for his friends. But mainly \textit{for \ him}.''
\ While speaking Bonimi feels the itch of the scar is turning into a burn. How he loves that heat.
``Excuse me,'' he says to Ezlip and clicks \ on the internal line, ``Janha --
''

``Yes, Mr. Saffia?'' the secretary's voice comes over the speaker.

``Janha, could you remind me what time the group of 'Selected Cadets'{ }is
expected?'' ``In a quarter of an hour, sir, when they finish their tour.''

``I'd rather not receive any calls until then. I'm pressed for time. I have to prepare for the meeting with
the PM later today.''

``Of course, sir.''

``Thank you, Janha,'' Bonimi hangs up.

Ezlip gets up to leave. ``There's a year and a half to go till the elections,'' he says as if
making an announcement. ``It occurred to me that you might consider standing for -''

So that's what was behind his aide's sycophantic display{ }and his~``Who's
Who'' research! ``For what party?'' Bonimi asks

``Any party would grab you with both hands - ''

``I see. Well, there's one angle{ }I wouldn't have to worry about: who'd run my campaign.
Thanks anyhow for your encouragement, but it's out of the question.''

Ezlip waves a hand in the air and leaves the room. He has no doubt that the idea he raised with the CEO~will sooner or
later take root{. }

Bonimi quietly locks the door behind his energetic assistant, returns to his chair, takes a slip of paper from his
wallet, glances at it and dials.~ After several rings the receiver is picked up at the other end with an absent-minded
``Hello -''

``Hello Shouba,'' he says, ``Bonimi here -''

``Oh...I'm sorry{\dots} I mean, hello Boni --'' Shouba collects himself.

``Am I interrupting?''

``No, of course not --''

``Maybe you have something on the burner -''

``The burner?'' Shouba is slow on the uptake. ~``Oh, I see -'' he
chuckles. ``No, I've nothing on it at the moment -''

``I wanted to tell you \ that they've given you a license to work with materials of B and C
categories.''

``But I didn't ask{\dots}'' Shouba is certain about this. He wouldn't ask Boni for anything
for himself.

``Yours truly did it behind your back.''

``Really{\dots}'' Shouba is deeply moved, ``Oh, ~Boni -''

``They'll mail it to you. There'll be some forms to fill of course. Anyhow, I wanted to be the bearer of
good tidings.''

``Thank you,'' Shouba, finally focused, feels he has to be more than grateful.
``I so appreciate the trouble you've taken on my behalf.''

``No trouble at all. I was happy to do it. Two phone calls all in all. Apparently ten years ago they
cancelled the requirements that had become obsolete.''

``I should have inquired myself.''

``Now all this is behind us.''

{}'Behind us', thinks Shouba. Boni does resemble Saffia in a way - he too radiates human warmth. Not gushing like
Saffia. Still, it's there. ``It'll change my life.''

``What's the use of wasted talents, right?''

``In other words, I have a license!'' Receiver in hand Shouba cavorts like a drunken bear.

``I believe that is in fact what I'm trying to tell you.''

``I'm a slow learner. And while we're at it{\dots} Alir popped in last week. Exhilarated isn't the word!
That too is thanks to you. Boni, you're brilliant!''

``And you're not annoyed with me for what I said~about the overnight permit for that Balad
--''

``Of course not.'' Truth{ }was, he'd actually been more than annoyed with Boni
telling him he was playing with fire when he asked for his help to get that overnight pass for that{
}Balad. But he must now put that aside.

Bonimi clears his throat before asking, with some hesitation, '' Is there any{\dots} news?''

``What news?''

``When I speak to you I can't help thinking about~old times{\dots} is there{\dots} any word from{\dots}
Rimat -?''

``I told you that after the tragedy she severed all ties.''

``If you hear anything --''

``Not a chance. She won't make contact. Never.''

``Why?''

``She blamed me for what happened to Saffia. Insisted that the bomb-timers I madefor him were
defective.''

``She cut off all ties with me too. She also blamed me.''

``For what?``~

``For the fact that the pump wasn't fixed on the bike. That I kept it under my bed.''

``Ah, yes. Now I remember your pump,'' says Shouba - something rings a bell. ``It was the best that money
could buy.''

{}``I saved up for it. It was so precious to me. But keeping it under the bed was childish. Self-centered. \ Saffia of
course was free to use it any time. But,~being Saffia, he let me keep it under my bed. Didn't make a fuss. Rimat was
different.''


\bigskip

\chapter{}

This is the fourth night. Won't she come tonight either? Glin is frantic. Perhaps~Black Mother objects to her going out
alone at night? And she had no way of letting him know. He turns on the Walkman hoping the music will help, but it
helps for only one short minute. He's too strung up. He peers outside. It's already dark. Then suddenly there's a
rustle in the bushes. Aera! He runs up to her, wraps his arms around her. She wriggles out of his embrace and dives
into the \ mouth. He chases her, catches up with her. ``What happened?''

``Something terrible -''

``Nothing terrible can have happened if you're here -''

``What happened is terrible-terrible-terrible -''

Glin's heart freezes. He puts his hands over his ears.

Aera pulls them away. ``I must tell and you must listen -''

``I don't want to -''

Aera locks his hands in hers, keeping them away from his ears. Only one thought has kept her going since the
\textit{act}: \ to share it with Glin. \textit{He must listen! }

``\textit{You} can know about terrible things. \textit{I cannot}.''

``That man, from the village{\dots} He didn't die naturally. He was made to die.''

Glin tries to free himself from Aera's grip, but her hands are stronger.

``\textit{I} did it. I pushed him onto the iron peg. On purpose. Shouldn't I be made to die?''


``Aera -'' Glin is engulfed in a wave of nausea.

``I should pay the penalty!''

Glin feels suffocated. ``Aera -'' he pleads, not knowing what to say.

``I should be punished!'' Aera remembers hearing of a woman who went mad because of one
thought she could not shake off. Is this what's going to happen to her?

Glin steadies himself, ``It wouldn't bring Moshko back to life,'' he says.

``But I must pay the price.''

``Enough,'' begs Glin, ``no more.''

``It haunts me{\dots} It'll haunt me to my dying day.''

``You will not die. Never.''

``I want to die now - or I'll go crazy!''

``No-no-no!``~

``I should go to the people in your village and confess, 'It was because of me that the guard
died'.''

``No way! Don't you dare!'' ~~~~~~~~~~~~~~~~~~~~~~~~~~~~~~~~~~~~~~~~~~~~

Aera lets go of Glin's hands. She feels herself sinking. ``I won't go down there even if you tell me to. I
don't want to \textit{be}. But I do want to \textit{live}. Save me from myself!''

``I saw it happen.'' Glin wants to embrace her but she doesn't let him. ``That
evening I was standing near the window like I always do. Watching you and your mother going out through the gate,
walking and singing into the sunset. And I thought I saw Moshko fall down. And then he seemed to get up. Now I
understand. It was \textit{you} who got up. How could you go in and out with the gate already locked?''

``I slid under the fence. You don't ask me why I did it?''

``I don't want to.''

``But you must want to know!''

``But I don't{\dots}'' Glin stammers, ``please let me be -''

``There was a man who was alive, and suddenly he isn't anymore. Because of me. You must know
why!''

``Moshko was going to die anyhow. He was already weak, already old.''

``You're saying a terrible thing. It's forbidden to think this way. Even if that man was one minute away
from dying. I did a terrible thing.''

``Forget it.''

``How can I forget? It's forbidden to forget such a thing. I want to tell you why I did it.''

``I don't want to hear,'' Glin says, ``I'm afraid.''

Aera takes a few steps towards the \ mouth{ }intending to leave. Glin blocks her exit, ``Why
are you leaving me?''

``If you don't care about what I went through, what I'm going through, what I shall always go through, then
{\dots}''

``Why delve into what's depressing?''

``I cannot do otherwise. I am what I am and you're what you are. We think differently. We feel differently.
We are not destined to be together.'' Again she begins to walk out of the .

Glin forces her back. ``Say what you want to say -''

``That song Mother and I sing, about the Pale Blue Valley. Like I told you. My people -- the Balads -- come
from there. Your people made us leave. Many of us want to go back there but people on your side say that what happened
in the past is over and done with. ~They say \textit{they }live there now. That there's no place there for us~anymore.
And your side is stronger than ours. So your side has the upper hand. And there're people among us who say that we
mustn't give in. That it's possible to make your side let us return. How? By frightening your side. By threats. Bombs.
Killings. More threats. More bombs. More killings. Until in the end your side will give in. And whoever wants to belong
to those who think this way must prove that he himself is capable of~ doing such things --''

``You too?''

``I belong with these people. It's my family. It's my relatives. My friends. When I was small, people of
your side killed my father and my brothers and my sisters. My mother hid with me. She stopped my mouth with her hand so
my crying wouldn't be heard. This is how we survived. She and I and a cousin of mine. And ever since then we've been
wandering from place to place.''

``Did this happen in Pale Blue Valley?''

``Yes. But I was too small to remember. I don't remember Pale Blue Valley. ~But because of mother's
stories, it's as if I remember it{\dots} \ If only I had not done this terrible thing! It's a curse.''

``What you did, you did for a great~ideal.'' Glin, aroused by{ }by the ideal, proclaims,
``That's how you should see it.''

``It's a terrible ideal.''

``No. It's a perfect ideal. Like a perfect piece of~music. What you told me about Pale Blue Valley's name.
That it got its name from those pale blue flowers that bloom in that valley only one hour, early in the morning and
only for one week in the year{\dots} It's unique like a unique piece of music... And this desire to return there{\dots}
It's like a musical theme {\dots} So magical that you want to hear it over and over again, you want it to echo in your
ears forever -''

``You shouldn't make somebody die for a piece of music -''

``You have to forget it,'' Glin is suddenly filled with r resolve. ``That's
all.''

``But I can't.''

``If you carry on like this, you'll make me die too. Make{ }yourself{
\ }imagine that it wasn't you. That it was somebody else, somebody else who is no more.'' Glin takes her
in his arms and now feels her warm response.

``So tell me who I am -- '' Aera feels her \ body relax in his embrace.

``You are who's here with me now, in my arms''

She repeats his words as though quoting a coded message, ``I am who's here with you. In your
arms.''

It can't be put off any longer, Glin knows. It's time to tell her. ``Listen --''

Aera senses the slight change in his voice. ``When?'' Until now she didn't want to know \ the
exact day of his departure.

``To-morrow morning.''

She was not prepared for the tears that are filling her eyes, the pain crushing~her heart. She asks. ``Were
you ever in love?''

``So many times,'' Glin answers. ``But I'm not in love now. I'm in something else
now. Now I'm in something that has no words. Only music.'' ~

``Aren't you going to ask~me this same question?''

Glin kisses her fingers one by one, ``How do you divide one soul into two halves?''

Aera says, ``Letters~could keep us together.''

``Where should I address them to? This ?''

``There's a new shopping mall in town,'' Aera repeats Yamik's words as if they're her own.
``Just opposite the supermarket there's an apartment building on the left, with three entrances. Next to
the middle entrance there's a sign - {}'Center for Aid'. It's for us Balads. Leave the letters there. Write 'Care of
Yamik for Aera' on the envelope.''

``Who's Yamik?''

``The cousin I mentioned before. He works for a contractor in town. He sometimes goes to that Center to
pick up things or drop things off.~ He'll bring me your letters. And we'll only write simple messages. As if we just
happen to know each other. Because this cousin is nosy, he'll open them and read them. I'll understand between the
lines.''

``Let's just draw straight lines with a ruler and leave the spaces in-between empty,'' Glin
suggests.

Aera laughs. ~Glin's humor revives her. His wit was what first drew him to her - it was something new for her, something
different, exciting. ``Your laughter sounds like a tinkling bell,'' Glin says.

``Mother says I got it from her -''

``When the two of you laugh together it must sound like the peal of a pair of bells -''

``I don't remember Mother laughing. I was too small then. She stopped laughing after our
tragedy.''

After a moment's silence Glin says, ``I have a souvenir for you -''

``Don't give me any souvenir .''

``Why?''

``Nobody should see anything of yours around me.''

Glin wheels over the old cradle from deeper in the . ``I was a baby in this,'' he says,
``and when I outgrew it my mom made it her sewing basket. She decided not to take it with
us.''

``A souvenir of a souvenir, I'll tell Mother that I picked it up at the flea market. And what shall I give
you as a souvenir in return?''

``The very fact that you're taking from me something that I saw every day of my life. In both my conscious
and unconscious life.''

``Oh -- '' a desperate moan bursts from Aera's heart. ``When will we meet
again?''

``Soon.'' Glin holds her close, kisses her. ``Soon. And then we'll be together
never to part. We'll think about it with all our might and it'll happen.'' He strokes her hair, caresses
it with his lips. ``No hair in the whole wide world~shines like yours.''

``I'll never do anything like that again. Never.''

Glin soothes her, ``Sh{\dots}``~

``I'll do only good things.''

``Sh{\dots}''

``You said before, 'Moshko was going to die anyhow. He was already weak, already old.' Say you're sorry
about those terrible words, that you don't think like that anymore -''

Glin does as she asks, ``I'm sorry. I don't think like that anymore.'' He spreads a blanket on
the \ floor. ``Listen -'' he turns on the Walkman.

An antique melody transforms the \ into a haven.


\bigskip

\chapter{}

``But I've no experience whatsoever,'' Alir says dispiritedly.

Shouba encourages him, ``Trust Boni. He knows what he's doing. He wouldn't have offered you this without
good reason.''

~Shouba doesn't make light of Alir's misgivings. He too doubts whether Alir is competent to emcee a public ceremony
broadcast on radio and TV. For the life of him he would never have dreamt~that Boni would come up with this when he'd
asked him if he could find a slot for their poet friend. But once it became clear that this was Boni's decision he most
certainly didn't question it. First, because the very thought of doubting a decision made by an expert was not like
him; second, because he trusted Boni's judgment. And third, because if the ceremony went well, part of that success
would be attributed to Alir as its MC, and this could open up prospects for the poet - something Alir sorely needed.

``Boni is too good to me,'' Alir says thoughtfully.

``Why wouldn't he be? Didn't the two of you share that glassed-in veranda in the old days?
Feet-to-feet?''

``What does that have to do with me standing before a~hall full of people?'' Alir shudders.

``We'll practice,'' Shouba reassures him, ``try it out.''

``Your scientific approach,'' Alir grins desperately.

``Don't underrate it,'' Shouba responds. ``I'll be the audience. We'll do it a
few times till you feel confident and spontaneous. Until it becomes second nature.''

Alir shrugs. ``Do you remember~Boni's days as a milkman?'' \ He asks.

``Oh, that's right!'' Shouba says. ``I completely forgot that --''

``I wrote a poem about him then. I wrote that when he delivers milk, the sky is a milky hue and heavenly
white flakes drop gently into the milk. 'Heavenly Flakes'. This was the title of the poem. Who knows whether Boni
remembers it. I'm too embarrassed to ask him. But in my mind he'll forever be associated with that poem.''
Alir remembers he had written it in his first notebook of poems which is stashed somewhere among his other notebooks
nobody's is interested in ``Boni has a scar on his left cheek,'' he says, ``I
don't recall that he had it then. Do you?'' \

``Neither did I,'' Shouba agrees. \ He had in fact noticed that \ scar the moment Boni
appeared at his front door when he had invited him for the purpose of asking \ him whether he could do something for
Alir. But that rendezvous was of course a secret from the poet. ``On the contrary,'' he says.
``How well I remember~his smooth chubby cheeks. Who knows how he came to get a scar like
that?'' After a minute he adds tenderly, ``I loved the way he and~ Saffia called each other
Big-Saff and Li'l-Saff -''

Alir's ears still reverberate with Rimat's shrieks as she burst into the glassed-in veranda. \ {}'Boni! Saffia's gone
out! Did you give him the pump? \textit{Did you}?' It was still pitch dark. He heard Boni grab the pump from under his
bed and bolt outside with it. And he, scared to death, turned~to the wall pretending to be asleep.

~

\chapter{}

Was the cacophony in the cafeteria always that unbearable? Or maybe he'd forgotten? ~Listening to his~old colleagues
expressing their remarks, observations, associations, memories, Zakod~feels once again that this place is no longer
what it used to be -~ his real home. ~Since his resignation eight years ago after the tragedy, this is the feeling that
engulfs him whenever he is tempted to accept those \ nvitations sent automatically by the secretariat of the university
to lectures, symposiums, colloquiums, receptions, ceremonies and whatever else. Especially so if the day happens to
coincide with his weekly shopping trip into town.

``Who's that?'' he tilts his head in the direction of a man who has just picked up a tray at
the head of the line at the buffet and is now collecting his plates and cutlery. ``His face is familiar
-''

``That's the ex-ex Chief-of-Staff,'' answers colleague number one.

``Oh, I didn't recognize him,'' Zakod says, ``probably because he's~not in
uniform.'' ~

``He appears on TV as a political commentator,'' says colleague number two, immediately adding
- not to embarrass their erstwhile colleague - ``quite rarely, actually. He'll speak after the
break.''

``What about?'' Zakod is interested. On receiving the invitation to this particular symposium
he'd thrown it into the waste paper basket after a quick glance and only committed the date to memory.

Colleague number two responds, ``'The Balad Problem -- Background, Reasons, Resolutions,
Implications.''

A handful, Zakod says to himself, recalling that it was this very topic that had intrigued him and aroused his curiosity
when he'd glanced at the invitation. When they{ }came{ }to the
Questions-and-Answers{, }he saw himself standing up and asking, ``Why call it 'The Balad
Problem'? Why not 'We and the Balads -- Our Mutual Problem'?'' He had toyed with the hope that by the
time~this symposium{ }took place he would~formulate a cohesive body of ideas about this topic and be
ready to express them. But then he had forgotten all about it and only remembered that he would not \ mind attending.
``Who is chairing that session?'' he asks the colleague next to him.

``Yours truly,'' colleague number one raises a hand, and adds, ``there're~rumors
about more plans in the drawer to keep them out.''

Colleague number one raises an eyebrow, ``Come to think of it, it's been quiet lately on that front
-''

Colleague number two raises both eyebrows, ``Undoubtedly they're waiting for the
right~moment.''

``'They' of which side?'' Zakod asks nonchalantly.

The two colleagues grimace jointly.

Colleague number one is the first to return to the topic at hand, ``Does anyone have any idea about the
line of policy in that drawer?''

Colleague number two responds, ``To do it quickly, decisively and elegantly. In areas where there are large
concentrations of them, first try to persuade them by talks, leaflets, and such. Buses will be put at their disposal to
cross the border.''

Colleague number one asks, ``And if they refuse to cooperate?''

The expression on colleague number two's face reflects his doubts at the prospects. Colleague number one comments with a
worried look, ``There's talk about live ammunition being stockpiled -``~

``They don't have a chance in a frontal confrontation,'' colleague number two maintains.

A waitress appears, ``Gentlemen, anything else you'd like?''

``A glass of water, please,'' responds colleague number one. Interested in the same, colleague
number two and Zakod raise their hands.

Colleague number two carries on, ``Food, water, blankets, medicine, toys for the children will be made
available for them. ~And only specially trained soldiers -- of both sexes -- will take part in this
operation.''

Colleague number one is interested, ``What happens to them across the border?''

``They'll be the problem across the border -'' colleague number two smiles.

The~waitress reappears and puts three glasses of water on the table. They all thank her. She asks hesitantly,
``And {\dots} the bill?''

Colleague number two says, ``It'll be for all of us together, please.''

Colleague number one gathers that she's wondering about Zakod. Not being a \ habitue \ in the cafeteria she doesn't
recognize him. He explains: ``Our guest is an emeritus professor of this university.'' Once
the waitress disappears he wders aloud, ``Why isn't it possible to come to terms with them in that area
called Pale Blue Valley? To some extent, at least?''

``First of all, what do you mean by 'to some extent?'' colleague number two flares up.
``And secondly, how would you extract \ those of us who have already been living there for years and
years? Just to think about the compensation they'd demand. And rightly so. And third, the precedent it would set.
There'd be no end to the claims of all kinds of people who'll come along to prove that a grandfather of \ theirs lived
there and a great-great-grandmother had a stake there too. All the way back to Adam and Eve.''

``Not necessarily,'' colleague number one responds coolly. To relieve~the~tension that has
begun to threaten their serene academic threesome, he decides to steer the conversation to another topic and turns to
Zakod, ``It occurs to me that you might know Professor Michlor Nissen, a member of the International
Medical Society Acting Committee, who is going to participate in the discussion.''

``Yes,'' responds Zakod, ``we once knew each other.'' He's shocked.
~How could he have failed to notice old Michlor's name on the invitation? ~He feels uneasy about meeting the man face
to face. He never responded to his letter of condolence after the tragedy.

``I heard that he asked to have his name removed from the list of participants,'' says
colleague number two.

``Oh,'' Zakod is not displeased to learn. It doesn't only have to do with the fact that he
failed to respond to Michlor's letter of condolence - it also relates to that painful chapter in both their lives that
centers around Saffia.

~``The Dean's secretary told me they'd received a cable from him saying he's flying with disaster-relief
forces to {\ }that godforsaken area hit by that terrible hurricane. He'll be organizing there an
emergency children's hospital.''

``The man's a saint,'' says Zakod lowering his head. He'd rather not confront the pitying eyes
of the two colleagues.

Colleague number two asks him, ``Weren't the two of you connected to a man who was killed in a still
mysterious accident during the so-called `Phony Occupation'?''

``We were,'' Zakod answers laconically.

``What was that man's name?'' asks colleague number one.

Zakod is at a loss - if he says he forgot, what will they think - that he's losing
{\ }it?{ }That he is suffering from a cognitive decline? Or that for some reason he would
rather not spell out Saffia's name? That the question~embarrasses him? He says, ``Saffia. Saffia
Saffia.''

``The idea was doomed to fail from the start,'' says colleague number one, ``but it's still
interesting. It could be the basis for several doctoral theses. ``Do you have any material on the
subject?''

``Nothing in writing. Everything was oral then.'' The words lie heavily on Zakod's lips. He
knows how unconvincing he sounds, which is why he adds, ``Also, I have not had much contact with Saffia
Saffia before his plans crystallized.'' However feeble this may sound, it is the honest truth. But the
only person who could ever believe it is he himself.

Colleague number two tries to clarify the point, ``Was it to do with differences of opinion about the
purpose? The considerations? The means?''

``Hard to differentiate.'' Zakod gets up with a forced ironic smile. ``Pardon me,
gentlemen, I have to verify something on the bulletin board.''

Colleague number one waits until Zakod is out of earshot. ``That tragedy devastated him,'' he
murmurs. ``He threw his life's career to the winds. \ He's not his old self.''

Colleague number two also speaks from his heart, ``To go into farming at that age. Sad. If not downright
preposterous.''

Colleague number one wonders whether they were doing right or wrong by inviting him. ``I'm always uncomfortable
about~meeting him~face to face,'' he says, ``What can we talk about? More crucial, what shouldn't we talk
about?''

``What you said to him about~the `Phony Occupation'{}'' suggests colleague number two, ``that it could be a
subject for numerous doctoral theses -''

``I thought it might draw him back to academic life,'' colleague number one interrupts him, ``I feel so bad
about his horrible predicament.''

``I have a student who might fancy it for his doctorate,'' says colleague number two, ``I could ask him
whether he'd be ready to meet with him.''

``Excellent idea,'' colleague number one~nods in agreement.

``There's the bell!'' says his colleague. ``Where is he? I don't see him?''

{}``He's disappeared,'' Colleague number one says. ``And perhaps he decided to make himself
scarce.''


\bigskip

\chapter{}

Seeing Yamik appear up the hill Mother runs out to greet him: ``Congratulations! Bless you! Showers of
blessings!''

They embrace. He leans \ his head on her bosom. Tears long-forgotten sting her eyes as she strokes his head. \ {}``God
bless you and keep you, my son!'

``Why the tears, auntie?'' Yamik, all innocence, wants the reason spelled out.

Mother cries out to the star-studded heavens. ``Because of our boundless joy!``~ After a
moment she calls out to her daughter in the shack, ``Aera! Our man is here! The
father-to-be!''

Aera pulls her headscarf tighter and gets herself to the doorway.

With few quick strides Yamik covers the distance between them. He puts one arm round her shoulders, his other hand on
her belly. ``My Aera -'' he whispers holding her close, covering her with kisses. Then he
takes her to the~bench outside and sits her down next to him. After a moment he reaches for the knot of her kerchief
and unties it. His eyes meet the shorn crown of her head. \ He remains speechless as the kerchief slides down to reveal
his belovedm's entire shorn head. ``Why?'' he murmurs painfully.

``It's so unbearably hot,'' Aera says in colorless voice - that hurdle thankfully behind her.

``I guess you know what's best for you -'' Yamik sympathizes.

Aera retrieves her kerchief from Yamik's hands to cover her head again.

Mother~feels better. The moment she so dreaded is over. The first time she saw that terrible thing her daughter had done
a week ago she was aghast. She too got that flat response, 'It's so unbearably hot.' She'd immediately thought about
Yamik~who could never wait to caress Aera's abundance of shining hair. And that apathetic tone somehow matched Aera's
lusterless eyes when she'd told her about her pregnancy. She wasn't like a woman thrilled to share good tidings. Why
isn't she overjoyed the way she ought to be? -- Mother wondered. Because of us Balads, she answered herself trying to
believe it. Now she lets herself ignore Aera's despondency and asks her, ``Did you cook? Did you bake? Did
you prepare food for the husband? For the father-to-be?'' She turns to Yamik, explaining, ``A
woman thinks only of herself and of what's inside her when it's her first time.''

``I didn't cook,'' says Aera. ``I didn't bake and I didn't prepare food for the
husband. For the father-to-be.'' She doesn't add 'Because he isn't together with me nor am I together with
him!'. She feels Mother's eyes penetrating her soul. Is she conveying her troubled feelings to Mother wordlessly? Her
heart goes out to her. ``Were you like me?'' she asks her, ``lazy and tired and heavy?''

``I don't remember,'' Mother evades the question and hastens indoors. How well she remembers!
When she was pregnant first time she became much more agile, much more alert, seven times more eager to take upon
herself any chore.~~~

Aera says, ``But I do.'' She's not sure Mother, by now indoors, has heard.

Clearly she has, for she floats a question from the kitchen alcove, ``My little wild one, how can~you
remember things that happened before you were born?''

``Because you told me about them yourself. I remember your stories. I live inside them. They've become mine
too.''

Mother appears outside again. ~Aera rushes to her crying, ``Mother dearest!'' She nestles in
her arms as if seeking shelter. Mother holds her close~turning her eyes to Yamik who looks perplexed. She tries to hint
that this is one of those moods a woman has when she's with child.

~

\chapter{}

The contractor unlocks the front door.

``Welcome,'' he says, as he shows Yamik in. ``There's nobody home.''


Not long after this nice Balad started working for him, he let slip something about his wife and children being
``open-minded.'' \ But until now,~whenever they stopped by his home, he'd gaze into space and mumble something about
having to drop in for a second and that he'd be right back. ~This is therefore~the first time that his Balad foreman
has entered his home.

``Make yourself comfortable,'' he indicates an upholstered chair in the hall.~

``With these?'' Yamik points to his dusty overalls.

The fact that this is also obvious to the contractor proves that despite his wish to behave matter-of-fact he certainly
doesn't feel comfortable. He goes to the kitchen and returns with a plastic chair, a glass of water and some cookies,
``Help yourself - '' he offers the refreshment to his foreman with a smile and then goes to
the adjoining room leaving the door ajar. He leafs through a stack of papers, opens and closes drawers.
``How are the cookies?'' he asks Yamik from where he is.

Yamik realizes the question is the contractor's way of keeping an eye on him as if by remote control. He responds,
``With all due respect, they're nowhere near ours.''

The contractor enjoys Yamik's spontaneity and responds in kind, ``You have the advantage over us. People
who wander adopt the best of wherever they happen to be and then improve on it.''

``The full half-cup.''

~An apt retort, the contractor thinks, and responds, ``A healthy attitude. And here at last are the
building plans.''

Upstairs a door opens and is followed by descending footsteps. Yamik springs to his feet. The contractor obviously hears
those steps too and hurries back to the hall, the roll of building plans under his arm.

``Ouli!'' \ The contractor stretches his free hand towards a young man in a khaki undershirt
{\ }and slacks,{ }a dog-tag hanging from his neck.

~``Dad!'' the young man smiles as they embrace.

``When did you get in?''

``Late morning -''

``Ma knows you've come?''

``She was just on her way out when I arrived.''

``How are you, son?''

``I was dead tired. Just took a nap.''

``Back to bed, then -''

``No way. I have to be at university.''

The contractor throws an arm across Ouli's shoulder and swivels him towards Yamik, ``This is Ouli, my
eldest son. Meet Yamik, my right-hand man.'' Ouli and Yamik exchange a nod. The contractor's eyes move
back and forth between them. They get the hint and their hands meet for a polite~handshake.

``Let~me raid the fridge,'' Ouli says, disappearing into the kitchen.

This is what normal family life looks like, thinks Yamik. Father, mother, children, proper home, furniture, appliances.
Day following day rationally, assuredly.

``We'll soon be off,'' the contractor tells Yamik, unrolling one of the plans and scrutinizing
it. He doesn't invite him to sit down again but he would dearly like Ouli to cut short his visit to the fridge and join
them in the hall.

Ouli reappears as if sensing his father's wish, in one hand a cold drink, in the other a bowl of nuts. He smiles at
Yamik, ``Does my father exploit his workers? You can tell me - I know the kind of man he is
-''

Yamik smiles back, ``He exploits his workers same as he exploits himself.''

``No discrimination, right?'' Ouli is full of good cheer.

``None whatsoever,'' Yamik agrees.

The three of them share a laugh.

The contractor turns to his son, ``Will you come home after the lecture?''

``Unfortunately no,'' Ouli answers. ``I have to rush back.''

``What's so urgent?''

Ouli gulps down his drink, surprised that his father would ask such a question in front of a Balad. Where's the man
living? Doesn't he know how things are? \

{}``Would you mind putting this in the truck?'' the contractor turns to Yamik, handing him the roll of building plans
``I'll be along in a minute.''

Yamik and Ouli wave each other goodbye. Yamik departs.

``When's your next leave?'' the contractor asks his son.

``Depends on the situation,'' Ouli answers, having decided not to comment on his father's
indiscretion.

``Time to go,'' the contractor says, and embracing his son adds, ``And don't
forget to be careful.''

``\textit{You} be careful -'' Ouli echoes.

``You mean the Balad guy?'' The contractor asks dismissively. ``He's
fine.''

``So be \textit{extra} careful -''

The contractor waves away his son's apprehension.~ ``Not to worry. He won't get involved in any nonsense.
He's too level-headed. And he's married - with a baby on the way. It's only thanks to him that I could take on the
renovation in the TV building. On top of the Eastern Hills Project it's quite a load. But I took it because it'll give
me a foot in the door for other projects. I'm also thinking of your future.''

``I appreciate it, dad,'' Ouli smiles. ``There was a big write-up in the papers
about this renovation.''

``I saw it and told Yamik about it,'' the contractor says, ``and he said my name
should appear in the credits on the program as the renovation's contractor.''

``Absolutely. And his?'' Ouli realizes that his father is looking for an answer and adds,
``A bad joke, I admit. And how are things coming along in that Eastern Hills project?''

``Nicely. But they're threatening to land us a 'Work Cessation Order'. Some kind of~legal point~has cropped
up. This is the bad news. The good news is that I'm meeting the schedule I'm committed to with the renovation job. \ I
have Yamik to thank for that. For the past few weeks we switched. He's been in charge of the auditorium while I was on
the Eastern Hills site. But there's a bit of bad news. Yamik has found another job closer to his family, which means
that unfortunately, very very unfortunately, he's leaving me. But not before we finish the auditorium job. He promised
me and you can count on him.''

``His command of our language is pretty impressive,'' Ouli says. ``It's only his
accent that gives his Balad origin away.''

``That's how it is with those we work with -''

``One reason to be extra careful -''

``I am,'' the contractor smiles, ``don't you worry -''

~

\chapter{}

Every movement she makes is so precise, so perfect -- thinks Dela - feeling Rimat's hands gently lowering her head onto
the pillow and tucking the blanket around her. Rimat then pulls up a chair next to the bed, sits down and holds her
\ hand while she says, ``Goodnight Moiku, ~goodnight Blaya, goodnight Tiqvi, sweet dreams.''

Dela is sure Rimat can visualize the scene she \ has described to her: their old home, the lamp on Zakod's desk in his
study throwing a beam of soft light \ into the girls' room, their three beds, three desks, \ exercise books, reading
books, pencils, crayons strewn on each, a bookcase~full of board games, dolls, albums, books, knick-knacks. Eeach girl
breathes in her own rhythm, the three rhythms weave into one melodious hum that fills her heart.

Every night Rimat tiptoes out of Dela's bedroom \ engulfed with this scene. She goes to the kitchen which opens onto the
living room. \ Zakod is dozing off in his heavy armchair in front of mute images~gliding across the TV screen. She
picks up her embroidery from the bottom shelf of the kitchen cupboard. Tonight, hopefully, she'll add another flower or
two to the unfinished garland.

``Rimat?'' Zakod is awake.

``Yes?''

``Is Dela asleep?''

``She is.''

``Is the garbage ready for me to take out?''

``I've already emptied it.''

``Have we not agreed that \textit{that} shall be my task? Why did you not wake me up?''

``I like to get my hands into garbage.''

``As a punishment, come over here and take a look.''

A short time after she had arrived he told her that he had stopped taking any interest in what was happening in the
world. Still, every once in a while he draws her attention to items in the newspaper, on the radio, on television. Is
it because he wants to draw her back to life? And thus - vicariously - draw \textit{himself} back into life? She puts
down her embroidery and comes over to him.

``Look -'' Zakod points at the screen.

``What's this?''

``A troupe of clowns,'' he says, turning up the volume. An oleaginous voice is heard,
``And therefore, ladies and gentlemen, my fellow citizens, in these demanding times it behooves
us{\dots}'' Zakod turns the sound off.

``A comedy act?'' Rimat asks and in a moment gets the point.~The two of them share~a laugh.
``Who are they really?'' Rimat asks.

``Your Prime Minister, your Defense Minister, your Minister of the Interior, your Minister of Foreign
Affairs.--``

``They're yours, too,'' Rimat interrupts

``After the news there's an hour of sing-along.''

``I don't sing anymore.''

``The two of you were~always singing,'' Zakod reminisces, ``in unison, in two
parts. I once joined in and you clapped your hands and said, 'You're so off key, my standing ovation!'''

``How Saffia berated me afterwards! \ {}'How can you~insult someone like that!' Anyway{\dots} I can't sing
anymore.''

``Cannot?''

``Ever since{\dots} ~no voice ~comes out. When Glin was small I couldn't even sing him a
lullaby.''

~Zakod is thinking about things abruptly~excised from his life. Surgery without anesthetics. ``But~Glin
Saffia is so very much nto music,'' he says softly

``He was like that already as a baby. And it became more and more so as he got older and wanted to listen
to music on the radio or the gramophone. I told him then that music gives me a headache. The inventions of Walkman and
earphones solved the problem. But it was only when he showed no interest in playing Saffia's violin that my heart could
rest easy.''

``Why does he think that Saffia died of heart failure?'' ~

``It's not\textit{ untrue}, is it? I didn't want him to know what his father had done. Since the beginning,
right from infancy in fact, I tried to teach him that bombs and killing and wars and armies are terrible.~ To abhor all
of them. I wanted to protect and guard~what~Saffia left me for safe-keeping. And I lied to you: Glin doesn't have any
disability. He's completely healthy. ~It was because he listened to me that he became a conscientious objector and
could thus avoid conspriction.``{ }

Zakod is impressed, ``I am proud of him!''

``Really?''

``Really.``~

``I didn't know you were that way inclined.''

``Have you forgotten all the rows that Saffia and I had?''

``How could anyone? \ But I thought that by now{\dots} you'd have changed your mind -''

``Because of our tragedy?'' Zakod takes her hand and she cannot stop him. ``The
tragedy did not change my mind, it only strengthened it.'' He turns off the television altogether. The
screen becomes~a flat metallic grey.

Except for a hazy beam of light coming
from the kitchen Rimat and
Zakod are in semi-darkness.

``Since we're into telling all kinds of truths,'' Rimat says, ``why didn't you
tell Dela that we know each other from way back?''

``To save her more suffering. Had I told her, her pain about the void in our marriage \ \ would have only
got worse.''

Rimat gets up to leave.

Zakod follows her. ``It is the same with me as ever,'' he says.

``No, no -'' Rimat makes for the basement door.

Zakod bars her way. ``For a while I had an arrangement with a small hotel in town \ on the weekly shopping
day. The condition was that she should not be young nor pretty and absolutely quiet. Preferably dumb. ~Once there was a
mishap - the same one came again. I had not foreseen this predicament. It was then that I stopped. Who else in the
world can I tell this to?''

~

\chapter{}

Glin leaves the Center his heart pounding \ {}- \ a letter from Aera in his hand! He wouldn't of course open it in the
presence of the kind man who runs the Center. He'll cross the street. There's some shade there next to the brick wall.
He'll lean against it and open the letter without being disturbed by pedestrians or passing cars. He's already there
when a handsome well-built young man approaches him,~

``Glin?'' Yamik asks~extending his hand.

``Indeed,'' Glin answers and feels the stranger's eyes fixed on the envelope he's about to
open.

``I'm Yamik. Aera's cousin.''

``Oh,'' Glin shakes Yamik's hand. For one moment he feels he should return the letter to him.
~

``Aera would like you to know that she's well,'' Yamik says. ``Very
well.''

``You're a courier for words written and oral?''

``That is so.''

``Then please tell her that I too am well. Very well.~ I've just left a letter for her at the
Center.''

``I'll be there in a minute,'' Yamik looks Glin over. Aera had described him accurately.
Skinny with a Walkman and earphones hanging from his neck. There's a certain naivete about him. He's obviously crazy
about Aera. Could he really prove to be of some help like she says? ``Nice to meet you. Maybe it wouldn't
be a bad idea for us to exchange a few words.''

``About what?'' Glin can't suppress his curiosity.

``About what's on Aera's mind,'' Yamik answers

``You mean the Pale Blue Valley?'' Glin asks.

``The very same -'' answers Yamik and looks around. ``But this calls for
somewhere less busy, less noisy, less open.''

Glin is excited. \ Aera's courier wants to have a talk with him! That will surely \ strengthen the bond between them!
``I live not far from here at the western edge of town,'' he says, ``on a farm
belonging to a man called Zakod.''

Yamik smiles and asks ``Should I knock at the door and introduce myself, 'I'm Yamik. Yes. You guessed
right, I'm a Balad. Glin and I are~friends. Can I speak with him?''

\ ``Every day around noon I move irrigation pipes from here to there at the far end of the farm,'' explains
Glin. ``There's a row of cypresses there. ~In the middle there's an old fig tree. It's now in leaf.''

``With plenty of shade under it,'' Yamik continues~amiably.

``I'll be there \ to-morrow around noon -''

``We'll wait there for each other,'' smiles Yamik, ``depending on who gets there
first.'' ~

Once again he extends a hand towards Glin for a handshake. Glin responds. Yamik places his left hand on their two
clasped hands. Glin covers it with his. The letter held between his fingers hovers like a white dove over their clasped
hands.


\bigskip

\chapter{}

Shisha puts her hands over her ears to shut out \ the terrible racket all around, careful not to smudge her fresh red
nail polish or - God forbid - get some of it on her hairdo.

``How much longer do we have to suffer this?'' she complains to the elderly security guard.

He was pensioned off a year ago but the powers that be rehired him on special terms since he's so experienced and
trustworthy. As far as Shisha is concerned his intuition was right. Her dumb-headedness, provocative dress and heavy
makeup are misleading. Inside she's a good girl. He can tell. He can't stay away from her information desk.

Shisha doesn't feel that he's flirting with her - \ he's old enough to be her father! But his hanging around her all the
time does get on her nerves. Though it has its advantages. She's new to{ }\ the job and he fills her in
about the different people in the building and their various functions. He certainly helps her get her bearings.

``Nothing we can do about it,'' the grey-haired guard shakes his head, ``just a
bit more patience and it'll all be over.'' ~

``At first I was very excited about the renovation,'' Shisha admits. ``It was
like those pictures of 'before' and 'after' over there -``~she indicates with her chin the exhibit at the
entrance to the lobby set up a few days ago. The elderly guard had explained it all to her -- the way the TV building
looked in the past and how it had changed over the years. It was also interesting to see pictures of people who work in
the building -- the way they look now and how they looked when they were young, even as chubby babies.
~``But I never thought it would mean this kind of hell. The dust, the dirt, the mess,'' she
says, blowing on her sparkling nail polish. ``Do you think they'll finish in time?''

``Of course they will,'' the elderly guard assures her. ``They've hired a
top-notch contractor.''

``Shhh{\dots}'' Shisha manages to caution him in time and he hurries back to his post. She
brings out a small mirror from under her desk and checks the contours of her lipstick, smartens {\ }her
posture by{ }throwing back her shoulders, finally thrusts her fingers into her curls~to puff them out.
Oh! The nail polish! She'd forgotten all about it! Hopefully it's already dry. The main thing is that she's ready for
the guide now emerging from the adjacent corridor,~followed by his tour group. He~crosses the lobby and stops at the
exhibit, ready to address the group. When they're settled around him he begins, ``Ladies and gentlemen,
having visited the main areas of the TV building and learning about its activities, it's now worth viewing this exhibit
- BEFORE AND AFTER.''

Shisha has been waiting impatiently for the return of the tour group since it disappeared into the building's bowels an
hour ago. Having not long since broken up with her boyfriend because of a nasty tiff, she fantasizes about a new
romance with this handsome guide. She must find a way to stop him at the end of the tour.

Meanwhile the descending elevator stops at the lobby and people step out. The last to leave are the CEO and his aide. In
a display of shiny red fingernails Shisha waves to the guide to get his attention. ``Mr. ~Bonimi Saffia -
the CEO,'' she mouths. He gets the message.

Rimat intercepts this inaudible message. Could that~really~be Boni's back? CEO of the Public Television Network? She
tries to remain hidden while not letting him out of sight. He's deep in conversation with the young man by his side.
Suddenly he turns round as if searching for somebody. Herself? Still hidden she manages to get a glimpse of his face.
It's him! He's filled out and grown taller. But where did he get that scar on his cheek? What kind of injury could that
finicky youngster have suffered? Her anger has not waned. She hadn't expected to bump into him when she planned joining
this tour{.} She's here because of Alir. \ Since reading that dreadful
poem she's wanted to settle accounts with him. Just as it's thanks to Gidal that she knows that Boni has been appointed
CEO of Public Television, it's thanks to Zakod that she knows that Alir has been given a room in the TV building as
emcee of the{
}coming{ }inauguration ceremony for the renovated auditorium.

After reading out to her that item in the newspaper, Zakod had asked her what she made of it. She'd just shrugged her
shoulders. After his and Dela's horrific tragedy she didn't want to touch on Alir's horrible poem. Zakod may even be
unaware of that poem's existence. ``I'm a complete ignoramus when it comes to poetry,'' Zakod
had added then, ``but ~I do remember there being something disarmingly candid in his poems``.


And then a day or two later Zakod shared with her another item in the papers - that ~Public Television had begun free
guided tours of its premises for the general public. So she'd made arrangements to join this one on the day she does
the weekly shopping in town. Reaching the fourth floor she'd found what she was looking for - Alir's name on one of the
doors.

The~tour guide thanks Shisha with a meaningful wink, and \ then gestures to his group to step back and clear the way for
the CEO and his aide. He goes straight up to Bonimi, introduces himself and then adds, '' I'm the guide of
this group today, sir. Would it be possible for us to have a look at the~auditorium? Even though it's still under
renovation? Just a quick look?''

``By all means,'' the CEO is happy to oblige. ``Just follow me.''

The young man is delighted. This feat of his will be noted in his CV in the Public Television Network's PR Department.
~He blows Shisha a kiss across the lobby. She'll hang onto it.

The elderly guard has also noticed that ephemeral kiss. \MakeUppercase{A}t the first opportunity he'll warn Shisha
against flirtatious young men. That too is~part of his reponsibility.

Bonimi continues on to the auditorium. Before entering he turns his head for another glance at the tour group. No. He
was mistaken. No face there resembles Rimat's.

Ezlip separates the two plastic~sheets suspended across the entrance to the auditorium for the duration of the
renovation. Bonimi goes through, followed by Ezlip. The tour guide and his group follow. Rimat stays hidden at the end
of the group. When Security is deep in conversation with the pretty receptionist, she slips back into the corridor
unnoticed.

Shisha remarks to the elderly guard, ``I bet that guide served in one of those elite combat
units.'' \

He vehemently disagrees. ``After demobilization, those guys find better things to do than shepherd guided
tours in a public building.''

``You can never know,'' Shisha stands her ground. ``Maybe it's just for the time
being.''

Having no convincing argument to the contrary he merely remarks ``I have a picture of myself when I was his
age -'' \

Shisha asks, ``Did you also look like a movie star?''

He blushes, the creases in his face turning dark purple. ``So they said.``~

``I'm sure you were,'' Shisha gushes over and then asks, ``Tell me, why do they
make such a fuss about security around this entrance?''

The guard is more than happy to provide an exhaustive answer, ``Public Television Network is an arm of
the~government. I hate to think what could happen if somebody~unauthorized~took it over.''

``What could happen?'' she asks.

``Anything,'' the elderly guard opens his eyes wide. ``Everything.''

~

\chapter{}

~Rimat quickly retraces her steps to the elevator bay, pushes the `UP' button, enters the first available elevator and
presses 4. The second the elevator stops and the doors slide open she steps out, trusting her memory to find the right
room. She knocks on the door. No answer. She carefully opens the door just a crack. Nobody's inside. She closes the
door behind her and takes a look around. The desk is covered with books and brochures. A wornout coat hangs from a peg
on the wall. She guesses, rightly, that its bulging pockets are stuffed with notes and{ }newspapers
clippings.

Alir enters the room with a handful of papers. ``Sorry,'' he steps back diffidently on finding
a strange woman inside, deducing that he's in the wrong room.

``Alir, you didn't make a mistake,'' Rimat says. ``It's your room.''


``Rimat!'' His papers drop out of his hands onto the floor.~ ``It's so many years
that I've had this daydream - I open a door and \textit{there you are}!''

``Come in,'' Rimat takes hold of his hand to~pull him inside and with her free hand closes the
door behind him.

~Alir throws his arms around her, ``Why is it that I can do now what I didn't dare do then?''

``Because all the cells in our bodies have changed hundreds of times over since then, if not
thousands.'' Rimat collects the scattered papers and hands them to Alir. ``But \textit{you}
still have papers all over the place as always. Nothing changed there.'' She pats his hair.
``There's a new color here. Gray, to be exact.''

``But you haven't changed at all!'' Alir exclaims.

``So much so that you didn't recognize me. Poetic license.''

``I can't believe it. I can't.''

``What's there to believe or not to believe?''

``How did you get here? And where from? Why did we disappear from each other all these years? And the baby?
Do you remember how I would grab the cradle with him inside -- grab it by the new handles your friend Gidal
fixed{ }on{ }{}- and
spin it round and round? And you'd almost faint from fear? Who does he look like? You? Saffia? Does he play the violin?
How did you find me?''

``My sixth sense,'' Rimat replies.

``Do you know what I'm doing here?''

``I heard about it -''

``I was in a bad way. Lost the job I had at the school - don't ask me why. I was drowning in the deepest of
depressions. I would go to Shouba's... Remember him?''

Rimat freezes.

Not sensing it Alir continues, ``I'd go to his place and do nothing but cry. Like a baby. And then a phone
call - out of the bluest of blue skies. And it's our old Boni who was appointed CEO of the Public Television Network -
whatever they call it - asking if I'd be ready to plan the ceremony for the inauguration of the renovated auditorium
and present it.''

Rimat steps back to put some distance between them. ``How could you write that horrid line?''
Her hand~shakes as she takes from her bag the folded leaflet and hands it to him. He ignores it, and the leaflet
flutters down and lands on the carpet between them. Rimat quotes from memory: ``'Next to~a throwing hand,
I'm the grenade --'{\dots} How could you?''

Alir stutters, ``I{\dots} I{\dots}''

Rimat locks his right hand in hers, ``How could this hand write such words?''

``Let me tell you how it happened{\dots} listen -- ''

``What's there to listen to?'' she fiercely cuts in.

``I wrote this while I was watching a TV documentary about a group of evacuated Balads. A little Balad girl
was crying about her doll that she dropped. And she couldn't pick it up because her mother was carrying her in her
arms, running{\dots}''

Rimat stops him short, ``A little girl's tears about her lost doll inspired you to write such a poem?~ And
then somebody reads your poem and it incites him to throw a grenade! To kill!''

Alir is beside himself, ``That wasn't the idea of that third last line. I'll explain it to you
-''

``There's nothing to explain!''

``But you must listen, you must! The first two lines describe my identification with any human predicament.
And the third line -- the last \ {}-- expresses the idea that as a result, I, like it or not, become part of any evil
done. It's a shocking line. Yes. It's hell. As much as I'm part of anything that's good, I'm also part of anything
that's evil. Now do you understand? It's predestined. It's a curse.''

``You need to be a philosopher to understand all this. Whoever reads this poem is sure that you're inciting
people to throw~grenades! To kill!''

Alir's voice is stifled, ``No{\dots} No{\dots}''

``You only think about yourself. Only about what words mean to you. Have you forgotten what happened
because of words?''

A shiver runs through Alir's body. He knows what she's talking about. ``Rimat, you must believe me. I had
nothing to do with~what happened to Saffia... nothing - ''

``Yes you did! You didn't do anything to stop~Saffia from being carried away by such rhetorics. The words
that led him to that deadly course of action. You didn't write words against those words. And now you wrote a poem
ending with such a line. You're self-centered. So egotistic.{ }You
don't understand what other people might and can read into your words!''

Turning away from Rimat Alir covers his face with his hands. In his mind he sees that little Balad girl crying about her
doll. He feels as if he's crying with her. ~He whispers, ``I'll change that line. At the inauguration
ceremony I'll say that I'm replacing these words with others.~ I'll say this in front of the whole
world.''

Rimat quietly lets herself out of the room.

Unaware of this Alir continues in a whisper, ``I promise. \ I'll be true to my promise.``~
Hearing no response he uncovers his face and looks round. The room is empty. Was Rimat here? He goes out to the
corridor. No one there. He returns to his room. What's that on the floor? That leaflet with the poem! Yes{\dots} Rimat
was here. Here in his room!


\bigskip

\chapter{}

Stretched out on the bed Aera lays her hands on her belly. ``This baby doesn't let me be!''
she moans.

``A healthy sign,'' Mother leaves what she's busy with at the stove, comes and bends over her,
beaming.

``I'm scared,'' Aera clings to Mother{.}

Mother says, ``I was also afraid the{ }first time.''

``I don't believe you,'' Aera presses her lips{ }to Mother's~dry skin, breathes
in its musky aroma. ``You only want to make me feel strong. You don't know what fear is.''
She leans her head against her mother's breast. ``I want to be back inside you - to be and not to
be.''

``My silly little goat,'' Mother plants a kiss on her daughter's forehead.
``There's already someone who needs you. And I'll be with you all the time. I won't leave you for a single
moment.''

``Mother dear, how is it going to happen?''

``Oh my precious, you're so forgetful. How many times have I already told you? When it's time. Things take
care of themselves.''

``But the pains,~dear Mother -- ''

``Those~pains~bear fruit and then they're gone.''

``But, Mother, my one-and-only, what kind of a life will my baby have?''

``A life of milk and honey.'' Sweet words flow from Mother's lips but her heart is heavy. What
\ has happened to her Aera? She was always so sure of herself, always facing each new day chin up. Is it Yamik she's
worried about - the father to be? She goes over to the cradle in the corner, picks it up and sets it down next to
Aera's bed. ``How clever of you to get this at the flea market,'' she says.

``You can find there almost anything want. Except for a priceless Mother like you.''

``Once the baby's in the cradle all your worries and fears will be over. And now I must prepare some food
for you and your baby, and for his grandma who's already crazy about him.''

``Mother, my lovely Mother --'' Aera sobs. ~

Mother feels she ought to stay with her daughter a little longer. ``I look at this cradle and say to myself
that it's not new but is{ }still in very good condition. You can tell that it belonged to fine people
who loved and cared for their baby who lay in it. It's a cradle that brings good fortune. And look at the
handles!'' Mother tugs at the handles. ``\textit{Are they sturdy}! You can see they're newer
than the cradle itself and were fixed in place by someone with good hands. Let me feel the mattress, see if it's
well-stuffed -''

``The main thing is the idea - not the small details,'' Aera responds using Glin's words.

Mother lifts the mattress. A slip of paper detaches itself from underneath and{ }flutters down to the
floor.

``What's that?'' Aera is alarmed.

``A bit of nothing-'' Mother picks up from the floor~an old yellowed photo of a man who looks
dead. She conceals{ }it between the palms of her hands.

Aera sits up, ``\textit{What is it}?``~

``Nothing, just an old snapshot.''

``But if somebody's looking for it?''

``Who?''

``Whoever brought the cradle to the flea market. Give it to me -''

``It's not good for you to look at it -''

``Mother, \textit{let me have it!}'' Aera insists. Mother is forced to give in. Aera flattens
the photo and scrutinizes the man's face. ``This man looks as if he's just woken up. When he's washed his
face and combed his hair he'll probably look handsome. I'll go to the flea market and return this photo to that
stall.'' She puts the photo under her pillow. Later on she'll find somewhere to hide it. Doesn't the man
there look a bit like Glin? One day she'll ask Glin about this photo. Let that day come soon!

``Don't put it there,'' Mother begs her. ``It could bring bad
luck.''

``On the contrary,'' Aera retorts, ``the cradle obviously belonged to fine people
as you just said. It'll bring me their good luck.'' She turns to the wall and asks, ``Mother
darling, tell me again, why did our people leave Pale Blue Valley?''

``I've told you so many times and I'll tell you again. We didn't choose to leave, we were forced to. They
were after us, chasing us away.''

``Why?''

``They wanted our Pale Blue Valley for themselves. ~And they were strong and we were weak. We thought that
after a while things would change and we'd go back. But it didn't happen. Now Yamik and the men will make it possible
for us to go back. Pale Blue Valley is so beautiful. It is there that once a year{\dots} ''

\ ``Pale blue flowers bloom very early in the morning for one hour only,'' Aera joins in, ``for one week
only, only once a year {\dots}'' She takes a deep breath then says, ``You know what, Mother
darling? Let's make a flower bed in front of this place{\dots} Let's find some pale blue flowers to grow there
-''

``A wonderful idea,'' Mother says mother,'' but it won't be instead of our real
Pale Blue Valley.''

``Of course not,'' Aera agrees

\ Aera seems to be feeling better now, more relaxed. Hopefully she's even fallen into a wholesome sleep. Mother goes
back to her chores at the kitchen alcove. Suddenly Aera asks, ``Mother dearest, will you ever stop wearing
black?''

``It's my{ }skin,'' Mother says without a moment's reflection.

~

\chapter{}

``Your embroidery's coming along nicely,'' Dela nods towards Rimat's busy hands.

Rimat sighs, ``I've yet to see the end of it.''

``What made you pick~such a large a piece of cloth? \ Big enough for what, exactly?''

``It picked \textit{me} up,'' Rimat retorts. ``There it was collecting dust on a
shelf in the general store in that village. Nobody wanted it, so I took it. With no special aim in my empty
head,'' She giggles.

Dela joins her. They giggle together like two young girls. A thought goes through Dela's mind - \ Rimat and Glin have
brought the girls' laughter back to me. Especially Glin.

``Did you ever go in for embroidery?'' Rimat asks. It hadn't taken her long to realize that
Dela was not devastated by questions about the past. On the contrary - they seem to energize her.

``No, I didn't. I was never any good at handwork. Nor headwork, for that matter. I was
useless.''

Rimat knows~there's no point arguing with Dela's self-dismissal. The fact is that had Zakod not finally~decided that the
time had come for him to marry and start a family, and had Dela not been the only~available woman. \ she would have
remained single and continued working as technical secretary in the university's History department until retirement.

~``Our girls were gifted,'' Dela adds. ``Moiku loved to embroider. Blaya loved
doing ceramics. Tiqvi would run to the neighbors to play their piano. Zakod had already put down a deposit for a piano
for her.'' \ Her emaciated arms, lifeless wooden sticks, rub against the wheelchair's arm-rests.

Rimat stands up and goes over to the cooker to stir the porridge. She tastes it, adds a pinch of salt and some sugar,
gives it another stir and tastes it again. She turns off the flame, fills a bowl with the porridge, puts it in front of
Dela and carefully fits a tablespoon into her hand at the right angle. ``Bon app\'etit s-''

Dela thanks her and tastes the porridge with the tip of her tongue. ``The very same taste that I used to
make for the girls,'' she smiles. After eating a few spoonful's she turns to Rimat, ``Zakod
explained to me why those Balads did it,'' she says, ``They had a reason and a purpose. He said this is how it has
always been in history.''

``That senseless stupid history that keeps repeating itself,'' Rimat mutters.

~Dela swallows a few more spoonsfuls of porridge then says thoughtfully, ``I saw the girls fall. And then
there were flames all around. I was choking. I passed out.'' She finishes her porridge.

Rimat asks, ``Would you like some more?''

``That's~how I used to ask the girls. 'Would you like some more?' Thank you, Rimat, but I've had my
fill.''

Rimat sprays the window with cleaning fluid. Wiping the windowpane she sees something{ }in the distance.
She blurs the glass with another spray of the fluid.

``You're working too hard,'' Dela remarks. ``That window was already
sparkling.''

``Not quite,'' Rimat says. She rubs the window clear again and looks out.

``Enough, enough,'' says Dela. ``You haven't even had your coffee yet.
And{ }I want to drink my coffee with you.''

``Soon -'' Rimat lowers the venetian blind leaving the slats horizontal and then goes to heat
the coffee.

``How is it outside?'' Dela asks.

``Looks like spring,'' she says airily, gripping the cups, saucers and teaspoons tightly to
allay a wave of unease.

Dela asks, ``Do you remember what I wanted us to do come spring?''

``Of course I do,'' Rimat responds.

``I know you've already done it,'' Dela says, ``But please~make a note again of the ages and the colors. ~I
want to say them again.'' Rimat picks up her pad and pencil and waits. Dela dictates, ``Moiku
would be eighteen next birthday and the color she loved most was blue. Blaya would be sixteen next birthday and the
color she loved was red, and Tiqvi would be fourteen next birthday and what she loved most were polka
dots.'' Her voice thickens, struggling with the tears. ``The color didn't matter. Why is it
that of all things, it's my memory that survived?'' Rimat dries her tears for her. ``Zakod
doesn't let me say things like that. And I -- if I don't say them -- how shall I remember, how shall I believe that I
was once a healthy woman? That I was once a mother. That I had three daughters. That I cooked and cleaned and washed
clothes. And ironed them. And put them away.''

Rimat encourages her, ``Go on talking, speaking, telling.'' She wipes away the crumbs around
Dela's mouth, collects the dishes and washes them in the sink.

Dela says, ``I can already imagine you going into town and buying the material - blue, red and any color at
all but with polka dots. And your finding a seamstress, asking her to make the dresses, specifying the ages. And when
they're ready, bringing them home to show me. Afterwards we'll think about where to donate them. We'll do this every
year, right?''

``Every year, every year,'' Rimat responds enthusiastically. She knows there's no point
asking~Dela whether Zakod has hired a new laborer. She's come to understand that from the very start of their life
together Zakod has never shared with his wife his interests, activities and responsibilities{. }She
certainly won't {\ }ask Zakod heself. Maybe it was a passer-by who chanced upon Glin in the distant
field and stopped there to exchange a word or two, and they decided to sit under the{ }leafy old fig
tree. ~She returns to the window and widens the gap between two slats of the venetian blind. ~The scene that worried
her has disappeared.

``I too enjoy looking at the row of cypresses back there,'' says Dela,
``especially early in the morning when it's still misty. I make myself imagine the girls there. How they
would have loved to play hide and seek between the trees there, maybe climb that old fig tree in the
middle.''

Rimat comes over to Dela from behind and places her hands on her shoulders.

``Was Glin's father sick for a long time before he died?'' Dela asks.

``He was never sick. Heart failure is sudden.''

``How old was Glin when it happened?''

``Just a month old.''

``What was his name?''

``Saffia,'' the name lingers on Rimat's lips.

``I know that that's your family name,'' Dela says ever so softly. ``What was his
first name?''

``It was his first name too,'' Rimat hears herself say with some kind of relief she cannot
account for. ``His first and last names were the same. His friends would sometimes call him 'Double
Saffia' for fun.'' \ Had she not feared drowning Dela in her own pain Rimat would have clasped \ Dela's
shriveled hands.

Dela asks, ``Do you have a picture of him?''

``I had an album-full,'' Rimat says. ``Saffia's younger brother~was very good
with his box~camera. The number of pictures he took of us! But the album got lost when I left our place and went to
\ that village in the north. Like many other things. I had only one snapshot left, also taken by his brother, a picture
that I purposely left out of the album. It didn't do his handsome face justice -- it was taken just as he woke up from
a nap. I didn't want Glin to see it, didn't want him to think his father looked like that -- face puffy, hair
disheveled, eyes unfocussed. So I hid it under the mattress of Glin's old cradle which I turned into a sewing basket. I
kept it there and looked at it only when Glin wasn't there. And what happened? When we were getting ready to come here
and were already overloaded with luggage plus the violin and all kinds of stuff, and there was no way we could take the
cradle, Glin put it outside with the things we'd decided to discard. Then I suddenly remembered the photo! I waited for
an opportunity to go and rescue it when he wasn't around, but when I eventually did, the cradle was gone. Someone
must've~fancied it.''

``Oh my!'' cries Dela in true sympathy.

~``Who knows whether whoever took this old cradle-cum-sewing basket was curious enough to look under its
mattress.''

~Dela waits a moment before asking, ``Were you and Saffia also lovers...?''

``Yes{\dots}we were.''

``I sometimes watch movies on TV. Men and women embracing, kissing. All those movies that Zakod l thinks
worthless.''

Rimat returns to her chair hoping to add a flower or two to the garland.

``What I really and truly had were the babies, the girls, the mature women they were going to become. I so
long for the body of a baby, a small child to hold close, very close to my own body. And then watch it grow, come into
its own, have its own likes and dislikes. Moiku was once very cross with me and she said,' Moumli' --that's what they
used to call me -- ' Moumli, don't call us all together 'the girls'. Each one of us is different.' \ \ And Blaya once
said, \ \ {}`Look, Moumli, there are so many whites in the room - the walls, the moon in that painting over there, the
writing paper in the notebook - and each white is different.' And Tiqvi once said, `Moumli, in music we can say how
much we love and how happy or sad we are, a lot better than in words.' They opened things for me. They made me
understand things, made me see, hear. As if it was \textit{they} \textit{who gave birth to} \textit{me}.\textit{ They
who were bringing me up.}''

Rimat is almost moved \ to tears by Dela's gift with simple words.

Dela continues, ``I so enjoy watching Glin. Looking out of my bedroom window I sometimes see him standing
on the patio outside his room downstairs. His eyes are shut and he waves his hands~as if he's conducting an
orchestra.''

``He knowsby heart the pieces of music that he loves. Every note. Every bar.''

``He told me that he inherited his love for music from his father.''

``His father taught himself to play the violin. Without a teacher. Just by ear. He always sang as he worked
-- he worked as a painter and decorator, you know.''

``A few days ago I saw Glin talking to his Walkman -'' Dela remembers.

``It was a dream.'' Rimat is happy for~Dela at having had such a pleasant dream.

``No, I really did,'' Dela insists. ``It wasn't a dream.''

Rimat ~is intrigued, ``How did he talk to the Walkman?''

``Like this --'' Dela lowers her mouth above her skeletal hands.

Rimat is frightened by Dela's graphic demonstration. It immediately revives what she'd observed a few moments ago in the
distant field-- Glin and an anonymous broad-shouldered man sitting together in the shade of the old fig tree having a
friendly conversation.

``Rimat,'' Dela's emaciated arms again rub against the wheelchair's armrests,
``wouldn't you like to have another baby? You're still young. How I wish there was a baby in this
house!''


\bigskip

\chapter{}

``Shouba! Shouba!''

Is Alir in trouble again?~ Shouba rushes from the kitchen sink to the front door wiping his hands on his apron. He opens
the door and Alir practically throws himself inside. Fingers still damp, Shouba picks up his friend's beret that fell
off in transit.

``I've just filled twenty bottles with cleansing fluid for felt. You spray a little on, wait fifteen
minutes, remove the foam and - abracadabra! If the spot doesn't disappear, you do it again{\dots} you can repeat it as
many times as necessary{\dots}'' His jovial patter fizzles out as he absorbs Alir's frozen state of body
and mind. ``What happened?'' He

removes Alir's coat for him and drops it on a rickety chair. \ ``Alir, tell me what
happened?''

``Were you there? After the tragedy?'' Alir covers his face with his hands.
``When Rimat screamed 'murderers'!... her braids \ loose{\dots} dress hardly buttoned... screaming{\dots}
nonstop -''

``Yes,'' Shouba whispers. What could have resurrected that horrific scene in his friend's
mind? \ {}``Yes, I was there -''

Alir~shakes his head, ``I don't remember seeing you there.{ }I was so horrified by Rimat's
screams that \ I... I simply{\dots}ran away. \ Only after Rimat left with the baby to go to that village up north did I
return to pick up my notebook and a few odds and ends I'd left behind.''

Shouba waits for Alir to carry on, but he doesn't. So he gently asks, ``But what happened
today?''

``She suddenly appeared{\dots} an hour ago -''

``Who?''

``Rimat -''

``Rimat? She came to your place?''

``No. To my room in the TV building. When she'd gone I thought that it must have been a ghost. Because she
often appears in my dreams at night. In daydreams too. But it wasn't a dream - it was real. She left
this.'' He checks his pockets till he finds the crumpled flyer.

Shouba takes it from him; hundreds like it are stored in his cabinet in the corridor ready for the next protest rally.
``She's angry about the last line,'' Alir goes on to explains. ``She accuses me
of inciting~bloodshed. I told her what inspired this poem{\dots} what was on my mind at that time -''

``I remember,'' says Shouba pointing at the TV set. ``We watched that documentary
together. You wrote the poem as we watched it.''

\ They both well remember watching that scene together on this old TV set: a group of Balad villagers being chased out
of their homes. The camera zoomed in on a mother with a little girl in her arms; the child held a doll and suddenly the
doll had slipped out of her hands. The little girl burst into tears. The mother, in a panic to get to the lorry waiting
on the roadside to take them away, was unaware of the reason her little girl was crying - she was facing forward~while
the little girl was looking back~at the doll now being~trampled~by the Balads' feet behind them.

~``She said~it was these kind of words that led to what happened{\dots}'' Alir mutters.

``To Saffia?''

Alir nods. ``I wanted to tell her that the Balads turned my poem into a political pamphlet without asking
me - but I didn't get that far.''

``Why?''

``She disappeared -''

``What do you mean 'disappeared'?''

``She was suddenly gone{\dots} \ Vanished into thin air -''

Shouba feels for his sensitive friend. Poor Alir! If only he could protect him from Rimat's fury! ``So many
years have passed between~Saffia's tragedy and this poem -'' \ he says shaking his head.

``The time lapse between the two events has nothing to do with it. It's the essence that counts. It's how
we feel about things. Rimat is right. People could take the last line as a kind of incitement.''

Shouba doesn't feel qualified to enter into a discussion about poetry and its{ }ramifications.
\ {}``Look, Alir,'' he tries to explain, ``Rimat blames each and every one of us for what
happened to Saffia. She screamed at me then, 'What kind of rotten bomb-timers did you concoct?' And I know -- I'm sure!
- that they were perfect. Saffia and I tested them countless times, went through the whole process again and again.
~She also screamed at Michlor --''

``Who's he?''

``Don't you remember Michlor Nissen? Who \ went to medical school?''

``I'd forgotten him -''

``She screamed at him~for studying for his anatomy exam instead of rushing to give Saffia first aid. What
she screamed at Boni I don't remember. But scream she did{\dots} I'll make some tea,'' he says without
budging, Rimat's screams still resounding in his ears.

Alir paces up and down between the battered pots and pans and chipped china. ``When I lived in the Saffias'
glassed-in veranda with Boni, I couldn't help hearing bits of what you were all talking~about.~ But it was so remote
from me. I only wanted to love Rimat - the woman that \ she was. And to love Saffia - the kind, humane person he was.
And bask in the knowledge that she was nice to me and that he treated me like a real friend. And to write
poems.''

Absentmindedly he picks up from the floor a battered chess box and places it between the legs of an upturned chair.
``I remember an argument~one evening between Saffia and Zakod in the other room. I was sitting on my bed
on that glassed-in veranda. A poem started to take shape in me about the bed opposite mine.~ Boni's perfectly made bed,
so indescribably tidy. About a bed \ never been slept in, a life never been lived. But my pencil merely doodled on the
page because of the raging row on the other side of the wall. Saffia was boiling - \ not like him at all. \ He let fly
at Zakod, 'Go teach your pure uncontaminated history! In your ivory tower! While we get~into the mud and blood of real,
actual, history!' and Zakod hurled back at him, 'You're too drunk with it! You're not rational anymore!''
And he walked out in a fury slamming the door after him.

``That was the last time Zakod was there,'' Shouba nods his head.

``I didn't notice.''

``Saffia told me.''

Alir sinks onto a heap of old~clothes and continues, ``After Zakod left, Rimat came out to the veranda.
Without knocking. I didn't want her to see my scribbles, they embarrassed me. So I pulled out that page,crushed \ it
and threw it sideways on the floor. But she said 'No --', and she bent down, and picked \ it up, and flattened it out,
\ and handed it back to me.``~ He skips over the next memory: Rimat's body so close to his own. Her big
belly with a baby inside rubbing against him. She lays her hands tenderly on his head as though comforting a small
child. \ This is a moment all his own - a moment almost to die for{\dots} He says, ``She pleaded with me
in a whisper, 'Go speak to Saffia, tell him to stop.' I said, ' Me, of all people!?' And she said, 'Yes. Yes. You of
all people. Because of all the people around him it's only you he'll listen to. Because you're not \textit{in it}.' I
didn't say a word to Saffia, of course. Maybe that's what she holds against me.''

``Don't blame yourself. There's no reason to, none whatsoever.~ How does she look?''

``As beautiful as ever. Even more, because of some changes.''

``What kind?''

Alir closes his eyes. ``A touch of grey at the temples. A couple of creases between the
eyes.''

``They were always there -``~

``I don't remember -''

``They appeared when she got angry. And that was not infrequent.''

``Her special kind of beauty{\dots} the depth of her special kind of beauty -''

They're silent.

After a while Shouba asks, ``And what about herself? What did she say about herself?''

``I told you. She just disappeared.''

``You only spoke about the poem?''

``I don't remember what about anything else. I wasn't myself. I was overcome... overwhelmed
-''

~``After the tragedy she moved with the baby to that village in the north,'' Shouba
recollects. ``She broke all past ties. Did you ask her about the baby? I mean the boy? He must be grown up
now - ''

``We only spoke about the poem. Like I told you.''

``He must be around seventeen now{\dots} eighteen?{\dots}nineteen?{\dots} How many years is it? I've lost
count{\dots} ~Who knows what he inherited from Rimat... What from Saffia... If anything -''

``Saffia Safffia will forever remain young,'' Alir muses. ``We'll get older and
older in the natural course of time. You'll invent another paste, another cleanser. I'll write another poem, rhymed,
free verse, metered, unmetered. And then{\dots} Curtain. End.''

Shouba goes to the kitchen to put the kettle on. On returning to the front room he fingers through the contents of an
ancient cardboard box. ``Just look at what I unearthed!'' he calls out.

Alir immediately recognizes those thin paper sleeves of records. He's as moved at the sight of them as Shouba.
``How Saffia saved up for those records! The almost holy reverence he felt for each one of
them!''

``The upstairs neighbor got in touch with me after Rimat had gone and told me she'd left a few things
behind - in case I was interested. So I went over there. There was nothing really of value, but I just couldn't leave
Saffia's records there. I took them home, put them away~and forgot about them. Only recently did I happen to come
across them when I was looking for something else.''

~Shouba and Alir carefully lift out the records, read their covers one by one and replace them in the cardboard box.

Shouba says, ``I wonder if she also blames Zakod -''

``His tragedy!'' Alir gasps. ``D'you know anything at all about
him?''

``Only that he resigned from the university. Seems he didn't want anything more to do with Academia with a
capital A. He bought a farm somewhere west of town where he works and takes care of his wife who got pstyislly
paralyzed as a result of the explosion. Did you follow the case in court?''

``What case?''

``Against the Balads they arrested?''

``I couldn't take it. It would've finished me.''

``Zakod hired the best lawyer for their defense. He financed it.''

``Unbelievable --'' Alir says and a chain of words forms within him about a disaster that
befalls a close friend, a catastrophe no one can avert.

``In a Letter to the Editor in the newspaper he explained the reason: so they`d be given their day in
court, a fair trial and not a public lynch. I understood him. Still, it was weird. Rather going to the other extreme,
don't you think?{ \ }In the end they were acquitted because of `a shadow of
doubt'. Everybody knew they'd done it.''

There's a knock at the door. Shouba opens it. The skinny young Balad standing in the doorway is familiar to him from
previous visits. ``Hello there, young man, come in,'' He greets the lad with a warm smile.
``Good to see you -''

``I came to leave a letter,'' Glin says, ``and perhaps there's one waiting for
me?''

Shouba opens a drawer in his cabinet. ``You're already familiar with the ins and outs of this post
office,'' he gestures amiably to~the~young man to come over.

``I appreciate it,'' Glin thanks him. Yes indeed, there's a letter from Aera. He takes it and
in exchange leaves his letter for her. On his way out he notics the box of records on the table. ``May I
take a look?''

``By all means,'' Shouba is delighted. ``It's just that they're covered with dust
-''

``They're old,'' Glin says as he carefully picks them up one by one and reads the labels.
``Rare performances. A treasure. Thank you. Good bye.'' He leaves.

``I never ask their names,'' Shouba tells Alir, ``or where they live or what they do. Poking into their
private lives seems patronizing to me. I only want to help them. ~And now it occurs to me that I should have told that
Balad he could borrow any records he's interested in and return them later. You could see he's a
music-lover.''

``No way,'' Alir objects. ``Those records are too precious.''

``You're right,'' Shouba agrees with him and carries on. ``Whenever one of them
shows up \ I think, what would Saffia have advised them to do? Keep on demanding their Pale Blue Valley back? Accept
their eviction from there? By the way, did you notice that young Balad's speech? No trace of an accent! He could easily
pass as one of us! Oh dear! What's that smell? Oh!'' He rushes to the kitchen. ``The kettle's
almost burnt out!'' he cries from there. The kettle sizzles as he refills it. ``This time
I'll stand here till it boils.''

Alir joins him in the kitchen and asks, ``And where's Golden-Hands-Gidal these days?''

``Probably in that village in the north.'' By the time the water has boiled Shouba has washed
the dishes. ``He'd distanced himself from Saffia even before Zakod did,'' he recollects.
``He was so shaken by Saffia's planning that bombing and just fled north. Thanks to that, Rimat had whom
to turn~to with the baby.'' He makes room on the crowded slab for two cups. From the larder he produces
some aged biscuits. ``The bunch that we were -'' He checks the teapot inside: yesterday's
tea-leaves still look quite good. ~He pours on them the scalding water.

Alir, watching him, says, ``I'll read this poem at the ceremony.''~

``What for?'' Shouba frowns. \

``So that I can change the last third line. Rimat is right. That last line could indeed be interpreted~as
incitement to bloodshed. I promised her I'd correct it.''

``You could send her the corrected line to where she's living now -''

``I promised her I'd correct it publicly during the ceremony. In front of everyone.``~

``You're asking for trouble, you know,'' Shouba is serious.

The tea has brewed, he pours out two cups.

``I want to be good to my promise,'' Alir declares.

Shouba realizes this is not the moment to make Alir change his mind. He will try and prevail on him some other day.
There's still time.

``I'll try and~get you a complimentary ticket to the inauguration,'' Alir says.
``The tea is excellent!''

~

\chapter{}

``Come in, Alir, come in -- '' Bonimi recognizes the timid knock at the~door. He stands up to
welcome the overgrown schoolboy who enters his room - hesitantlyas usual. Despite their shared living quarters way back
in the past - feet-to-feet in the glassed-in veranda~at Saffia and Rimat's{ }{}- he never really got to
know him. He was only aware of Rimat being nice to him, but mainly of his older brother's deep affection for him.
Lately he has come to feel his own growing warmth towards this disarming poet.

``What's the good word?'' He lays a hand on Alir's shoulder, directs him to the settee and
sits down next to him. ``Any \ problem? There's always one problem or another. Par for the
course.''

Alir takes a deep breath. ``At one point during the ceremony I too would like to read something
aloud.''

``By all means,'' Bonimi responds, adding jovially, ``why discriminate against
the MC?''

``I mean \textit{that} poem -''

Bonimi, immediately catching on, cautiously asks, ``Do you think it's appropriate?''

``I'd like to change the third line.''

``Sorry, but I don't exactly -- ``

Alir declaims: ``'Next to a human being, I'm a human being / Next to a Balad, I'm a Balad / \textit{Next to
a throwing hand, I'm the grenade}{}'.'' After another deep breath he goes on to explain:
``That third line is misleading. What I meant to say was:~ if I identify myself with any `human condition'
- as I describe in the first and second lines -~ the conclusion is that when somebody throws a~ bomb,~ like it or not -
I share in this low, despicable, act. ''

``That's what I understood myself,'' Bonimi responds spontaneously. ``When the
Balads appropriated your poem, I felt that they'd misunderstood it. I thought that since they're so focused on their
own agenda, they see everything through that prism alone to the exclusion of anything else.''

``The problem is that they're not the only ones in the wrong. Some of ours are also in the wrong. Some of
us also think that I preach, that I incite, that I encourage {\dots} killing -''

``Narrow-minded people were, are, and always will be among us - '' \ says Bonimi and \ though
realizing he's failing to prevail upon Alir \ to change his mind, asks, ``Why bring this up again? It's
all in the past. You can rely on people's short memory.''

``Not all of them. Some people have a~long memory. Certain rumblings have reached me. This is why I want to
find a spot during the ceremony to explain myself, to say I'm erasing~this line and replacing it with
another.''

Belaboring the ins and outs of the unfortunate poem, andt this at the actual \ ceremony seems to Bonimi an untenable if
not altogether an off the mark idea. That's on the one hand. On the other, \ he wouldn't want Alir to feel that he's
making light of his distress.

``There's another reason not to make this correction publicly,'' he says. ``People will conclude that you
got orders from above. That will put things in the wrong light. And even if Public Television Network issues an
official statement clarifying its position on the matter, \ it won't help. On the contrary it'll only strengthen that
opinion.''

``But I must{\dots}'' Alir stammers.

Bonimi tries another tack, ``Think of yourself - if I may be so bold - what kind of impression does a poet
who corrects himself make? What about the moment of inspiration? Can a later stroke of inspiration replace the
original?''

This argument makes sense to Alir. He lets it ride for a moment but then confesses, ``You see, I promised
someone I'd make this correction, this change -''

``Somebody in the political scene? In the media?''

``Rimat -''

``Which Rimat?''

Alir lowers his eyes. ``Ours -''

Bonimi is thunderstruck. ``When? Where?''

``She came by. To my room here. A few days ago. ~I came in and there she was.''

~Bonimi gasps. So it was indeed~Rimat in the lobby with that guided tour! He wasn't seeing a vision! He almost whispers,
``Just like that?''

``She came to speak to me about the third line. She said these kinds of words were the cause of Saffia's
tragedy. That he was carried away by this~kind of{ }rhetoric. She was wild{\dots} and I promised her I'd
change it. In public. It haunts me.''

``What else did she say?''

``I don't remember. I was{\dots} I wasn't myself. I was too overwhelmed. I only remember what she said
about the poem{\dots} then she just{\dots} disappeared -''

``Disappeared?''

``Just as she'd appeared. Suddenly.'' Alir's gaze is lost in space. ``The
nicknames you and Saffia called each other - 'Big-Saff' and 'L'il-Saff'! How alike the two of you were. And how
different!''

``Saffia had his own original mind, his own original soul,'' Bonimi fears he may be losing his
grip. The scar begins to itch. ``I'm only a technician. I only file things away in their
compartments.''

``We called you Boni for short,'' Alir says. ``You didn't like it. You wanted to
be called by your full name - Bonimi.''

``You're absolved for old time's sake,'' Bonimi smiles warmly.

Suddenly they find themselves together in the past.

``I once had a poem in mind and wanted to write it down,'' Alir recalls, ``and
there was no room on the table for the piece of paper I held in my hand. ~I looked for something stiff to put it on and
picked up a novel that Rimat was reading at the time. And she got really mad, 'Don't take away my love story!'
\textit{But you are my love story, Rimat, }I wanted to say, \textit{You are the one who's taking away my love
story{\dots}} I don't remember if you were there -''

``Indeed I was,'' Bonimi says. ``I remember that. Yes. You were all in love with
her.''

``You too?'' Alir asks.

``Young as I was -'' Boni nods.

Together they fall into in silence.

Alir is the first to break it: ``And the poem? What shall we do about it?''

``Whatever you decide.''


\bigskip

\chapter{}

Through the closed door of the upstairs landing Glin hears his mother speaking with Dela in the kitchen. \ She's
probably folding the laundry and Dela's explaining where things go. That should take a while. As for Zakod -- since
he's just shut himself in the store-room, he can be relied upon \ not \ to emerge outside for at least another half
hour. In other words, now is the time to practice the technique of communicating with Yamik. He puts on the earphones
and speaks into the mike plugged into his Walkman, ``Yamik, can you hear me?'' He lengthens
the antenna. ``Yamik, can you hear me \textit{now}?'' He changes the angle of the antenna.

The door above opens and Mom is coming down the stairs. He disconnects the mike and collapses{ }the
antenna. Eyes closed, Walkman on his chest, he stretches out on his bed.

Rimat crosses her room and \ stands at his open door. As always she'd moved by the sublime expression on her son's face
when he's listening to music -- something between concentration and elation. She turns back to her room.

Glin mumbles, ``What-where-why-when?''

``Nothing to do with what-where-why-when,'' Rimat \ almost in a whisper.

``But{\dots}?'' he raises his voice, demanding.

``I don't want to disturb you.''

``You already have. You've already spoiled everything, ruined everything.'' He waits a second
for her reaction - to no avail. ``Please mom, I beg of you, what's \ \ the problem? \ And~don't beat about
the bush!''

``Okay, straight to the point,'' Rimat returns to his room, a dress of hers draped over her
arm. ``A few days ago I thought I saw you with someone under that old fig tree far away in the field
-''

``This ' I thought' of yours is what's called beating about the bush. Since when has your vision been
impaired? You saw~right. But why did you wait until now?''

``I don't know -''

``Your admittance of ignorance is indeed praiseworthy. It was a guy who lost his way.''

``Where did he want to go?''

``To the lake. I told him they're spraying~pesticides there so he'd better make a detour.''

``It took you long enough to explain this -''

``He got interested in my Walkman. We started talking - about music, recordings, standards, and all
that{\dots} Come. Sit.'' Rimat complies and sits on the edge of his bed.~ ``What's eating you
up?''

``The two of you spoke so earnestly -''

``I was trying~out being earnest{\dots} a new experience for me -''

``Serious conversations scare me.''

``Why?''

``Because they do.''

``It was only a show. Don't you worry your pretty little head -''

``Seems I'll always worry - pretty little head or no.''

``A waste of emotional resources.'' Glin remains apprehensive - hasn't he seen with his own
eyes~the pile of clean laundry upstairs waiting to be folded and sorted out? So why did Mom rush down now instead of
taking care of it? To spy on him? ``What's with Dela?``~

``She's busy with some patterns for those dresses she wants to ordr for her three poor
girls.''

``Shrouds - '' Glin shudders.

``She'll donate them to the needy afterwards. Doing this is a kind of{\dots} I don't know what to call it.
Anyhow, I told her we could copy a pattern of one of my dresses and she wanted to have a look at it. That's why I came
down now - to get the dress. ~And -- lucky me - get on your nerves while at it.''

``Think nothing of it, Mom -''

Rimat wonders~whether she should bring up the real reason for coming to his room - ~ Dela's description of his speaking
into the Walkman. She decides against it. ``What are you listening to?''

``A trio by an anonymous composer.'' How fortunate, he thinks, that Mom is not into music and
won't ask to hear whatever he's listening to.

``Happy listening,'' Rimat waves her old dress, turns round, crosses her room and ascends the
steps to the main floor.

Glin waits until the door above closes then~makes another attempt to get through to Yamik. ``Yamik, do you
hear me?''

Yamik's desperate voice comes through the earphone, ``Glin, I'm not~getting the hang of it{\dots} Glin, I'm
not {\dots}''

The connection is cut off. He and Yamik must meet again. Yamik needs another practice session.

~

\chapter{}

``Home, sir?'' asks the~chauffeur.

``Home indeed,'' the CEO answers. The darkness inside{ }the limo's back seat,
the routine question and answer between him and the chauffeur, the even hum of the motor -- all these give Bonimi a
sense of peace he yearns for during the hectic hours in the TV building and the{ }bleak solitude of his
home. He knows that once the limo merges with the traffic~the chauffeur will comment on something in the news, the
results of the last poll, whatever. \ The words will filter into his ears as if through a thick blanket. He pats his
scar lightly.

``Today's the last time, sir, that we're taking the old route home,'' the chauffeur says as he
navigates the roundabout.

``Why's that?''

``On the dot of midnight our good old Old Road becomes pedestrian.''

``I all but forgot about that,'' says Bonimi who reacts impassively to external changes.
``So this is our swan-ride. No way to stop the wheels of progress, is there?''

``This morning they cordoned off the place where you, sir - when you had some spare time - would get out of
the car and take a short walk. They're going to put up a municipal information center there. 'MIC' for
short.''

``Well, it's about time -'' Bonimi agrees. {\ }

``Especially its being so close to the Citadel. It's always so crowded with sightseers and tourists round
there. They've put on~an exhibit there of old weaponry. I went to see it with some relatives here on a visi. They were
most impressed. And so was I.''

The chauffeur's words drone on in Bonimi's ears till they turn into a mishmash of disconnected syllables. Rimat's sudden
shriek cuts through: \ {}'Boni! Saffia's gone out! Did you give him the pump? \textit{Did you}?' He's torn from deep
sleep and immediately knows where Saffia was heading. Hadn't he heard snatches of what went on between him and his
buddies? He grabs the pump from under the bed and rushes out. It's still dark. One street before the Citadel he
literally bumps into~Saffia - \ the bike's rear light is of course off. 'What are you doing here, L'il-Saff?' his
brother whispers. He whispers back, 'I brought you the pump, Big-Saff.' And Saffia whispers, 'Thanks a lot, L'il-Saff.
~And now back to beddy-bye!'' He realizes that Saffia is pushing the bike rather than riding it because
it's weighted down. 'I'm staying here with you, 'Big-Saff,' he says, and Saffia says, 'No way. L'il-Saff~still needs
his sleep, he's a growing boy.' \ {}'No. Big-Saff needs L'il-Saff's help,' he says. He grabs~the~bike's handlebars.
Saffia pushes him aside. He tries again but Saffia whams him out of the way. Saffia has never hit him like that. He
falls face down on something sharp. His left cheek{ \ }is bleeding. An excruciating pain paralyzes
\ him. 'So very very sorry, L'il-Saff,' Saffia apologizes as he sprints off~with the loaded bike towards the Citadel.

He tells himself it's going to be all right. Weren't those bomb-timers made by \ his brother's close friend Shouba? But
in a few a minutes there's an explosion. And then another one and another. Why so soon? A huge flame shoots into the
air, smoke all around. He must do something about the pain in his left cheek. He gets up and runs, runs, runs. Suddenly
his foot hits the bike thrown down on the road. He runs some more. And there lies Saffia. He bends over him, whispers
to him but Saffia doesn't respond. Somewhere inside he knows that Saffia is dead. With a huge effort he manages to roll
the body into a ditch. His blood and his brother's mingle{. }He covers the dead body with stones and
rubble. Why? He covers with \ earth. Why? Because he's afraid. ~Afraid of what will be said about Saffia, about what he
did. Afraid of what'll happen to \textit{him} because of it. He runs back home. Some of Saffia's buddies are already
there. Rimat's standing in the middle of the room half-dressed, hair flying, the baby in her arms, hysterical.
'Murderers!' she's screaming, `Every one of~you!' \ Seeing him she yells, ``You're another one! You're no
longer Glin's uncle!'

The chauffeur decelerates, ``Here, sir --'' he points.

``Wh...what{\dots}? What's here?'' he asks.

``This is where they're putting up The Municipal Information Center. MIC for short.''

``I see.'' He looks out. The new structure is covered with scaffolding and nettings.
It{ }clearly promises to be an impressive building, both official and friendly, accessible to one and
all.

They drive on in silence until the chauffeur exclaims, ``What a long line in front of the
Citadel!'' Bonimi turns to look. ``What a crowd!'' the chauffeur is excited.
``Even little kids in prams and strollers. Wow, and how many tourists! ''

``Yes indeed,'' Bonimi agrees.

The chauffeur slows down because of \ a traffic jam. ``Lately, you know, the Citadel has been rated a top
tourists attraction,'' he says.

Boni surveys the line of people waiting at the gate, winding round the building. Once they're in, he knows, they'll be
guided, shown, enlightened. Probably told about the series of explosions towards the end of the so-called Phony
Occupation that wrecked an entire wing of the Citadel. About the mass grave of unidentified dead. \ Parts of their
bodies were found scattered all around. In the open, under rubble, some hit directly. some killed by the blast. Among
them -- Saffia.

``By tomorrow this street will become pedestrian,'' the chauffeur \ announces almost formally.
``To-morrow we inaugurate the ring road.''


\bigskip

\chapter{}

``I think you've got the hang of it now,'' Glin says.

``Yeah, I think so,'' Yamik grins. He turns off his Walkman and puts it in his basket.

Glin leaves his on. ``All the same, how about another try-out,'' he taps his Walkman, ``just
to be on the safe side?''

``No need.'' Not that Yamik is super-confident, but he now feels quite capable of handling
this apparatus. He winds the earphones' lead into a neat spool and tucks it next to his Walkman. ``Every
hour, on the hour exactly, right?'' \

``Right,'' Glin says, ``but what if one or both of our watches go a bit slow or
fast?''

``We should sync them twice a day with the radio signal,'' Yamik suggests.

``Perfect,'' Glin agrees.

They stand up and brush the sand off their clothes.

``How d'you get to be so expert at this?'' \ Yamik is curious to know.

``A response to an emotional need, I guess,'' Glin answers. ``Because of being
crazy about music as long as I can remember. And since my mother's allergic to music, I had to find a substitute for
the radio and gramophone. Until some genius invented earphones! From there I went into sound engineering while I was
still at school and the rest - as they say - is history. Yamik,'' Glin is serious now, ``this
may be the last time we meet under this fig tree -''

``Very likely,'' Yamik agrees. Every time they've met here under the old fig tree he's
imagined himself soaring high up in the sky~and looking down on the leafy canopy concealing the two of them. He
reflects on~the stroke of fate that brought them together. While at it Aera looms in front of him radiant in~the full
bloom of her pregnancy. ``Have you prepared the ground?'' he turns to Glin who's busy with
his own earphones.

``I kind of mentioned something at the gas station - just by the way,'' Glin replies.
``Same with some of the neighbors.''

``And the old man, your boss?''

``Wow! What he had to say! Don't ask!'' Glin goes on to imitate Zakod's furious voice,
``Joining that shameful army! Becoming a soldier of evil!''

``Perfect,'' Yamik is satisfied.

Glin shakes his head, ``It's so hard to get into a heated argument \textit{artificially}.''

``There's no other way, buddy{\dots} So -'' Yamik is quiet for a second and then continues,
``barring anything unforeseen I'll get to your place when its time to do the swap.''

``Any guess when?'' Glin asks.

``I'll only know by the end of this week. Through my boss. He's a good guy, but what a blabber! Anyhow, his
son's a recruiting officer -- and that's how I'll find out when the next recruitment's due. And there's another piece
of luck: the Cessation of Work Order in the Eastern Hill Project has been~extended for at least two months, so you
won't be disturbed there.''

They're quiet for a moment.

``Things are falling into place so neatly,'' Glin says. ``like in a jigsaw
puzzle.''

Yamik nods, smiling. Maybe too neatly, he's thinking - but in the face of his companion's enthusiasm keeps the thought
to himself.

{}``This'll knock them flat, won't it? When we pull it off!'' Glin is ecstatic.

``That's the whole point,'' Yamik shelves his worries for the moment. ``To take
full advantage of the time and place!''

``I have every confidence in our success!'' Glin announces.

~Yamik \ nods in assent. ``By the way,'' he asks, ``Can you drive? Do you have a
driving license``?

``Twice 'yes','' Glin responds, ``why do you ask?''

``Who knows - but it could become handy. And while at it - do you have a letter for me? It'd spare you a
trip into town to the Center.''

Yamik's invasion into the privacy of his and Aera's correspondence has begun to irritate Glin not a little. But Yamik's
practicality must take precedence. \ He takes the envelope out of his pocket and hands it to Yamik ~

``Did you write her about the explosives?''

``But of course!``~

``I'm sorry, but that's not good.''

``Why?'' Glin feels reproached and disappointed. True, he did want to show off a bit. But the
main idea was to please Aera \ and prove his commitment to Pale Blue Valley.

``The idea that someone might actually get hurt really bothers her,'' Yamik explains.

``What's there to worry about? I described it all to her in my letter, that the way it's planned, they
\ \ wouldn't want to suffer any damage, injuries, \ casualties.'' As he's talking Glin wonders, am I no
longer \textit{one of them}?{ }He has never given~this a thought. Who are those
with whom he's '\textit{We}{}'? Who are those who are '\textit{They}{}'? Being bonded with Aera body and soul has
nothing to do with it. It's only since he got to know Yamik and what ensued from that acquaintanceship that these
questions \ sometimes trouble him. Sometimes he fears that if he isn't careful he could be sucked into an abyss. Then
the only thing he could to do would be to cling to the pit wall and tell himself to hang onto Aera, to music, to the
Balad ideal of Pale Blue Valley. That's all.

``Of course they wouldn't want any damage, any injuries or casualties,'' Yamik says.
``This is what the whole plan is based on. Absolutely. But Aera refuses to understand that if we want to
be taken seriously, that \textit{unless we actually, physically,} \textit{put} \textit{the explosives in} \textit{place
}it will leave our threat empty.''

``I explained all this to her in those very words,'' Glin pleads.

``She won't understand because she doesn't want to understand. She's stubborn,'' Yamik insists, ``I know
her. You may write her about anything you like except our plan and the explosives.''

Frustrated as he is by this veto Glin naturally accepts Yamik's authority. He tears up the letter and puts the scraps in
his pocket.

When all of this is behind us, Yamik reflects with~longing, I'll tell Aera how I got hold of the explosives, how I hid
them, how I knew where to place them around the TV building.

He brings a bag of cookies out of his basket and offers them to Glin. ``Help yourself, it's from Aera
-''

Glin takes one and remarks ``I didn't know she baked -''

``O-ho!'' Yamik exclaims. ``There's nothing my cousin can't do. And with flying
colors! So what shall I tell her?''

``My thanks, with a mouthful of her cookies,'' Glin bursts laughing and is tempted to ask
Yamik how is it that beautiful Area is still unattached - but immediately check himself.

~

\chapter{}

Yamik covers the last lap home at a sprint. He bounds into the shack.

``Aera! Auntie!'' He hugs his aunt, kisses Aera and peers into the cradle,
``Ah{\dots} still empty!''

``A~man who is about to become a father for the first time has no patience,'' Auntie humors
him.

``Gifts!'' announces Yamik. He hands Auntie the first parcel he takes out of his rucksack then
lays the second in the cradle. ``I already want to spoil our baby{\dots}'' he smiles indulgently. ``And
last but not least,'' he turns to Aera, ``for you -''

``It's so stuffy in here,``~Aera gets up before he manages to present the gift and goes
outside - her heart heavier than ever.

With the merest nod of her head Auntie conveys to Yamik that Aera's swings of mood can \ be put down to her condition.

Hanging onto his rucksack Yamik follows Aera outside. He strokes her shoulder - and feels it stiffen under his hand. As
they reach the tree he gently turns her round to face him: ``Let me look at you -''

But all she can manage is a curt rebuff: \ \ ``I don't want you giving me the letters in front of
Mother.''

``Has that ever happened?'' he asks, offended, and immediately apologizes,
``Sorry, Aera, but what I brought you is a real gift -''

``I'm sorry too,'' she says humbly. What right does she have to torment this pure-hearted
brave man whose life is in constant danger? ``I don't know what's happening to me -''

``Never mind, love,'' Yamik soothes her readily. ``And here it is!''
The parcel he takes from his rucksack is wrapped festively and tied with a colored ribbon. He says, ``I
don't have a letter this time. I didn't have a spare minute to call in at the Center.''

Aera's fingers tighten around the \ parcel he's given her. What do his words actually mean? Does he want to conceal from
her the fact that Glin hasn't written? And in that case - why? Or has her husband begun to sense there's something
happening between Glin and herself and that's why he didn't call in at the Center?

``Anyhow, that skin-and-bones who can't live an hour without his Walkman and earphones writes the same
thing in every letter, '' Yamik smiles at her. ``Still, the main thing is that he does write
-''

``Nothing we can do about it, is there?''~Aera says her heart at rest.

``But the way he talks sometimes{\dots}'' Yamik adds, ``like a standup comedian
-''

Aera feigns surprise, ``As I recall he's as dry as plywood.''

``I guess he's more relaxed with me -''

``Seems so.''

\ Whenever Yamik reports to her about having met Glin she must be on her guard not to let slip a tell-tale word. How
well she remembers what Yamik said after first meeting Glin on~the street opposite the Center, 'His eyes \ scanned me
from top to bottom as if searching for you, as if I'm hiding you somewhere -- ' She almost broke down then. Mustering
all her strength she barely managed to utter a dull~ 'Ah --'

Mother comes to the window to invite the young couple to the table. Looking out, she{ }notices the easy
flow of conversation between the two of them. Let them be. Her heart at ease she returns to the cooker.

``What's the good word?'' Yamik pats Aera's belly.

``Same,'' Aera smothers a desire to shake off the unwelcome hand.

``Kicking a lot?''

``It's such a lazy baby,'' Aera says and begs the baby to cooperate and stay still for the
duration of this moment.

``I love that baby with a kind of love I didn't know I had in me. He's our togetherness.''

Aera knows she ought to say something that would echo his feelings, but she can't. She touches the hand lying on her
belly intending to remove it but Yamik, interpreting this as a desire for closeness, presses his other hand on top of
hers. ~

Making a small movement to free her belly she says: \ ``A couple of the men came over the other day. They
complain that you're keeping yourself to yourself. Mother told them that it's not like you to do any such
thing''

Yamik turns his gaze to the shack, ``Let's move a bit further away.'' They go round to the far
side of the tree / they take a few steps further from the shack. ``I'm taking note of your
advice,'' he speaks slowly, measuring each word.

Aera asks herself what kind of advice she gave Yamik? Then she remembers. ``Does that mean they suspect
that you may be planning something on your own?'' she asks. Yamik nods. \ ``What is
it?'' she takes his hand.

``Something.'' Yamik makes do with a single word.

``Yamik, tell me - I can keep a secret. Is it to do with the TV building?''

Yamik nods.

``What?''

``I'll tell you another time.''

``Yamik, what?'' Aera insists.

``I'll tell you when it's final.''

Aera stretches up as high as she can and folds her arms round Yamik's robust neck, her face against his unshaven cheek,
her belly rubbing against his hard torso. ``What?''

``After the renovation there's going to be a big to do there. It'll be broadcast over all radio and
television channels. Like you once said - we need to tell them about Pale Blue Valley. They need to have things
explained to them, to as many people as possible. \ So I asked my boss the contractor if he and I could swop places.
That way, he'd be on the Eastern Hills Project site and I'd be in charge of the TV building auditorium renovations. And
- he agreed! \ Meaning that now I can begin to plan something.''

``Like what?'' Aera is intrigued.

``To take over the broadcast --''

``You'll need some help from \textit{their side }for that -''

``And \textit{there is}.''

Aera tries to conceal her trepidations, ``Ah, you mean {\dots} him!?''

Yamik wonders what's that strange twitch in her right eye? Does that, too, come from her prenatal moods? He says,
``The very same.''

Aera feels that the momentary flutter of her eyelid has passed, she's regained her composure. ``I don't
think he's capable of anything like that,'' she dismisses Yamik's choice of Glin out of hand. ~

``You're wrong,'' Yamik contradicts her. ``You don't really know him. You don't
know about all the things he's capable of. Not to mention his expertise in sound engineering, transmission, and all
that.''

``That expertise of his is one thing,'' Aera tries to sound objective, ``his not
being a serious person is another. \ He doesn't know how to take on~responsibility. All he cares about is his music.
And you yourself say he's a clown. If~something funny crossed his mind he wouldn't keep it to himself. He'd blather,
ruin everything.''

``No, no,'' Yamik is adamant. ``Underneath all that he's serious, he's
responsible. And he wants to help us. He's completely committed to our Pale Blue Valley. So it's very good and wise
that you've told him all about it.''

``He's like a little boy, impressionable, easily excited.'' Aera tries to sound flippant.
``He would be a hazard.''

``He's with us all the way,'' Yamik says, and hesitates a bit before going on,{
}``for you, he tells me,~the sky's the limit. Pale Blue Valley is the limit.''

They hold each other's gaze for a moment.

Aera doesn't believe that this is what Glin actually said but prefers not to question it. She should be careful not to
focus the conversation on the relationship~between Glin and herself. She rubs her eye to prevent a further flutter.

``I probably know him better than you by now,'' explains Yamik. ``We've become
friends.''

Aera forces herself to shrug as if to say `That's your affair', and asks~nonchalantly, ``Will it be
dangerous?''

``No,'' Yamik says, though he doesn't sound too convincing.

``And if they get wise{ }to it? Find out{\dots}?''

``They won't.'' \

``How can you be so sure?''

``Because I am.''

``You've suddenly started believing in the supernatural?'' Aera giggles. ``In
luck? In fate?''

``The plan will be perfect.''

``Like the perfect crime?'' she stares at Yamik who nods in affirmation, ``You must tell me!''


``I've told you,'' he says. ``When it's finalized.~ To the last dot.''

``You're afraid I'll tell mother? There wouldn't be a squeak out of me. Come on, tell me just a tiny bit
-'' she screws up her face like a peeved child.

``You'll know in time.''

``Promise?''

``I promise. And don't mention it when you write to him. He must think that it's a secret between him and
me only. I'll go in to say goodbye to Auntie. I must get a move on.''

Aera stops him, ``Do you remember what we said about{\dots} killing?''

``It won't happen.''

``You promised -''

``I did,'' Yamik reaffirms and then adds, ``You haven't opened your gift
--''

``I'm saving it --'' Aera hopes these words sound convincing to Yamik. She doesn't accompany
him back inside.

``I have to rush, Auntie,'' Yamik tells his mother-in-law.

``Won't you have a bite first?'' she points to the table set for three.

``Sorry, but I must get back -''

Auntie hands him a \ a food package. ``Here, some of your favorite morsels,'' she says
lovingly. ``And I've told your friends who came by \ how well you know that acting alone is another word
for weakness and stupidity.''

``Loads of thanks, auntie.'' They exchange kisses.

He goes out{ }to Aera for their farewell. ``Don't forget to eat and drink for our
baby!'' he tells her, then his voice becomes serious, ``Aera --?''

``What is it?'' Aera is all ears.

``I don't know when I'll manage to get here again. You understand why. The men will be our go-between.
We'll communicate by a code of our own. That's how you'll let me know about the big event.''

Aera yields to his arms and lips. They end by shaking hands silently. ~He turns to leave. She doesn't take her eyes off
him. Every few vigorous steps he turns round and waves goodbye. Eventually his image blurs as he disappears into the
dark night.

``What did Yamik bring you?'' Mother asks her as she enters the shack.

Aera unties the colored ribbon and undoes the festive wrapping to reveal a silver-plated brush and comb set.

``Beautiful,'' mother says. ``Yamik is crazy about your hair. My gift is a packet
of knitting yarn - the right gift for a grandmother. And what did he bring our baby?'' She opens the
parcel lying in the cradle. ``Look!'' Mother exclaims at the sight of the folded fabric.
``He knew that I wanted to make a new cover for the cradle's mattress and he bought a piece of cloth
printed \ with a pale blue flowers!''

Aera stretches lazily on her bed and asks~absently, ``Is there enough material?''

Mother holds the length of material at the corners so that it unfolds to its full size. ``More than
enough,'' she's delighted. ``Enough for a mattress cover \textit{and} a sheet. For
both.'' Her laughter ripples through the house. ``Our baby will lie in a field of pale blue
flowers!''

Aera says, ``It's a long time since you burst out laughing like that -''

Mother returns the piece of cloth to the cradle and embraces Aera. ~``I'm drunk~with happiness because of
the baby! And you, my child, why have you stopped laughing?''

``I'm tired,'' Aera sighs, ``tired all the time -''

``You're eating for two, drinking for two, and also tired for two. Just rest yourself, my little one... my
treasure -``~

Mother gets up from Aera's bed and once again spreads out the cloth. ``The flowers are almost the same
color as those in our Pale Blue Valley. That once a year{\dots}''

``Pale blue flowers bloom {\dots}'' Aera joins in in a cadence of legendary litany,
``very early in the morning, for one hour only, for one week only, once a year -''

``You remember the old tale...'' Mother whispers. ``Just close your eyes, my
precious,~and go to sleep -''

Peace descends on the shack.

Suddenly it is shattered by Aera's cry, ``Mother!''

Mother is quick to rush to Aera, holds her trembling body. ``What is it, my little one?''

``Mother, Mother, Mother --'' Aera mumbles, ``I'm afraid, Mother, I'm so scared
--''

``Is Yamik planning something dangerous? All by himself?''

Aera huddles in her mother's arms, ``If the baby is fatherless -''

``Why such bleak thoughts? Our baby will grow up with a mother and a father the way it should
be!''


\bigskip

\chapter{}

Zakod, waiting in the store-room, almost pounces on Glin as he enters: ``What happened? What kept
you?''

``One of the sprinklers got jammed and I{\dots}'' Glin starts to explain.

``Well that is a relief. I worried for nothing. You look tired out.''

Glin pours himself a glass of lemonade from Zakod's thermos. He{ }slakes his
thirst in one gulp. Refreshed, he perches on the edge of the desk, feet dangling.

{}``If that is the case,'' Zakod continues, ``maybe there are other sprinklers that need replacing.''

``Stands to reason,'' Glin nods, thinking it might well be an idea to check all the sprinklers
before leaving. ``But what got you so worried?'' he asks, all innocence.

``That you may be going ahead with your plans.''

``Oh, that -- '' Glin laughs the idea off. ``Don't worry. Not yet, as you
see.''

``Anyway, it is kind of you to at least spare your mother knowing what is on your agenda. As for me, I have
not yet lost hope of convincing you to get back on the one and only right track.''

``Don't bother -'' Glin, pours himself more lemonade.

Zakod feels like pacing up and down the way~he used to in his lecturing days when he needed to focus, but he remains
seated. Still, he cannot throw off his old didactic authoritative tone of voice. ``The Ascendency Of Man
Over Beast,'' he declares, ``with \ capital letters. True we are ninety-nine percent made up of chemical
processes and impulses, but one percent is rational thinking. And that one percent is what enables us to make rational
choices.''

``Isn't that a chemical process too?'' Glin asks{
}earnestly.

``It is, but of a different nature.''

``We've often talked about it and thought it over. \ Zakod, you did your best.''

``No, I did \textit{not},'' Zakod insists. ``I still have to try and prevail upon
you to change your faulty mind-set. Okay. Let us not rehash the moral principle that it is your privilege to choose
\textit{not} to refrain from risking your life when others risk theirs. But do you know what the army is busy with
nowadays? Would \textit{you} be able to push Balads around? Drive them~from place to place? Evict them? Frighten them?
Beat them? Threaten them with live ammunition? And if something gets out of hand? And that live ammunition really comes
to life? And it hurts and~injures and kills?''

Glin cries out within, if you only knew the truth!? Out loud he says, ``I'm such a diehard nonentity, so
superficial! I love{\dots} music. But that too is superficial. Because I love it for pleasure, not for studying~it. Not
for adding to it by composing something new. I'm not a deep person, I'm tissue paper thin. And now I must go, I
promised mom to take in the laundry before it gets dark.''

As he's almost at the door Zakod speaks: ``I once had that kind of disagreement with a very dear friend.
The situation was different. The considerations were different. But the principle was the same. I carried on until I
stopped. I simply{ }grew tired.''

``And what happened then?'' Glin asks.

Heavy gloom descends on Zakod. \ Saffia's image looms large before his eyes. His voice grows hoarse. ``It
ended in tragedy.''

Glin scrutinises Zakod's face. ``Why should{ }I have to pay for a
precedent?'' he asks almost in a whisper.

~

\chapter{}

Aera opens the door and shades her eyes against the sun. Whose footsteps are these?

``Oh! If it's not our good friend Gidal!'' She welcomes him happily. ``What a
lovely surprise!'' She steps down and goes over to the
cradle~under{ }the tree.

``Good day,'' Gidal responds as he gets nearer, ``I haven't congratulated you yet
-''

``You haven't been up here yet,'' Aera retorts emphatically, her face radiant. She draws back
a corner of the mosquito net. ``And here is{\dots} nature's wonder!'' ~

Gidal bends over the cradle. A perfect small round head covered with shiny hair and adorned with a tiny rosebud ear
peeps out from a field of pale blue flowers. The thought about \ this baby's conception in the darkness of the \ makes
him uneasy. ~From the first time he spotted Aera and Glin~stealing inside he always stood by at a distance in the
shadows - like a guard protecting their trysts. ``A beautiful baby -''

``A little bitty chick,'' Aera releases the net.

With a gallant gesture Gidal presents her with the jar of honey he's brought along, ``To sweeten your good
fortune -''

``Thank you so much,'' Aera surveys the jar. ``It's from the village beehives,
right? Thanks from Mother too. She's gone shopping.'' Aware of the special relationship between her mother
and Gidal she adds, ``She'll soon be back.''

Having familiarized himself with the daily and weekly routine of the woman-in-black, as he thinks of her, Gidal, for
some unknown reason, timed his visit to-day to find Aera alone.

\ ``I don't~know much about babies,'' he says. ``I don't know what kind of gift
to buy for a baby.''

``You brought him the best gift of all. He shares with me whatever I eat and drink. You're so good. You
made it possible{ }for us to work in the village. You arranged for us~to live
in this~shack after renovating it yourself. And now you've brought us this sweet honey.''

``Your mother told me that you'd given{ }birth to a baby
boy.'' At the time he'd been struck by the woman's unbound joy at sharing the good news - as though she
was unaware of the infant's{ }problematic paternity. ``What's his name?''

``He doesn't have one yet. He's waiting for his father to come home so we can find a name for him together.
Meanwhile he's just plain Baby.''

Gidal is flabbergasted. Does she mean to spell out the identity of the father as Glin? To relieve himself from the
tension he says, ``Baby is smiling in his sleep -'' ~

``Baby's dreaming sweet dreams. Please take a seat,'' Aera points to the bench
outside{.} ``It's so comfortable - wasn't it made by your own
golden hands. I'll bring you with a cup of tea.'' She hurries inside.

Gidal calls after her, ``Please don't bother!''


\bigskip

``I'll join you,'' Aera calls back. ``I need to drink too - baby's orders. We'll
have a tea-party!'' She soon emerges carrying a tray with two cups of tea plus some cookies.
``We'll sweeten the tea with your honey,``{ }she
places the tray on a small stool. They enjoy the aromatic brew and the \ home-made cookies together.
``Isn't this great?'' Aera says brightly. Gidal nods in agreement. She can't help thinking -
this man was born good-hearted. Was he also born to be alone?

Baby whimpers. Aera takes him out of his cradle, seats herself on the doorstep, covers herself
{\ }discreetly{ }with a corner of the
sheet of pale blue flower pattern and~gives the infant her breast. Gidal, watching, is awash with
emotions{ }he's only once before experienced - when Rimat nursed baby Glin that evening when \ they had
first arrived. He's timed his visit well to be alone with mother and baby. He returns his empty cup to the tray and
goes over to the cradle. ``A nice old cradle,'' he
observes{.}

``I picked it up for next to nothing at the flea market,'' Aera tells
him{.}

How could that be? - Gidal wonders. The story of the cradle is indelibly inscribed in his memory. A short time before
Rimat was going to give birth, Saffia had come by it in lieu of payment for a plastering
job{ }he'd done. Rimat had cleaned it inside out~and fashioned a mattress for
it from an old pillow. Then Glin was born and it became his cradle. Later he'd replaced the cradle's frayed handles
with new ones. How well he remembers Rimat and Saffia's appreciation. No less unforgettable was how Alir - the Saffias'
homeless poet friend who shared the glassed-in porch with Saffia's young brother Boni - would swing the cradle back and
forth with Glin inside, sometimes even~swing it full circle - to Rimat's shrieks of horror. And then{\dots} a few days
after{\dots} Saffia's tragedy {\dots} Moshko the village guard~had come to tell him there was a woman with a baby at
the gate asking for him. The woman was Rimat~holding this cradle with Glin inside, her other hand gripping a suitcase
and two bags, with{ }Saffia's violin case slung on her shoulder. In time Glin
outgrew the cradle and Rimat turned it into a sewing basket which he saw whenever he came to visit. Not long ago Rimat
told him the cradle would be one of the things she and Glin would not be taking with them when they left the village to
move to Zakod's {{}--} having more than enough luggage without it. So who could
know how the cradle found its way to the flea market and how Aera came by it?

``What would you say if I got hold of four wheels on a metal frame and fixed the cradle on top? And it
would become also a baby carriage?'' \ \ he suggests.

``I'd thank you more than you could ever imagine -''

Baby is sated. Aera hoists him onto her shoulder and he responds with a loud burp. She breaks into a laugh that breaks
into a cry.

``Oh -'' Gidal whispers - what else could he say? ``I'll be off then
-''

Aera collects herself, dries her eyes. ``So nice of you to come,'' she smiles at him.
``And again many thanks for the honey. From Mother too.'' Aera watches him as he begins the
descent downhill. His back has started to bend. She calls after him, ``Come again, \ and often. Baby would
like to see more of you!''


\bigskip

\chapter{}

~``Next to a throwing hand - let my arm wither.'' Does it sound right? Alir tears up the~piece of paper
he'd jotted these words on{ }and tosses the scraps behind him. Another failed
attempt. \ A poem about his present predicament begins to congeal in his mind as the scraps of paper pile up like a
snowdrift. But he resists the temptation and once again confronts the challenge. He writes down, ``Next to
a throwing hand - who am I?'' This scrap immediately joins the heap. The same with ``Next to
a throwing hand - my hand isn't mine.'' He then simply{ }writes,
``Next to a throwing hand -'' He contemplates these words. Can that stay? Leave the dash?
\ Replace it with three dots? With a question mark? An exclamation mark? Add both ? No good.

He gets up and wades through the snowdrift. ``Ladies and gentlemen,'' he hears his own voice,
``dear friends, honored guests,~listeners and viewers, please bear with me. A while ago I wrote a poem
entitled ``Next to``. It had~ three lines, ``Next to ~a human being, I'm a human
being ~/ Next to a Balad, I'm a Balad ~/ Next to a throwing hand, I'm the grenade -'' A most
painful{ }condemnation was hurled at me for the third line. I was told it could be taken to mean that I
condone violence, that my intention is to incite it. No. A thousand times~NO. \ Nothing could be further~from the
truth. Nothing negates my fundamental, spiritual and ethical credo \textit{more}. I've tried to correct this line,
believe me. Again and again. And I have not found the right words. Please let me explain myself. ``Next to
a throwing hand, I'm the grenade --'' When a despicable, base, vile, criminal individual throws a grenade,
I too become base, despicable{ }and vile. I too become a criminal. Rimat, do
you hear me? Do you understand me?

~

\chapter{}

``Glin --?'' Rimat murmurs.

Even though he'd unlocked the front door soundlessly and crossed her room barefoot on tiptoe, she'd woken up.
``Shhhh{\dots}'' he whispers.

``What's this shhhh{\dots}.?'' she asks.

He answers, ``Not to wake you up -''

But Rimat is already sitting up in bed, ``Please, Glin, can we do without the gags? For once? I was worried
stiff about you -''

``Why?''

``Didn't you hear?''

``No.''

``A bus was ambushed. Two killed, fifteen injured - three seriously.'' Since he was a teenager
she's stuck to her resolution never{ }to ask about his
free{{}-}time diversions, but tonight she can't help it.
``\textit{Where were you}?''

``I took in a movie. A stupid{ }comedy but OK in
parts.''

``And only \textit{now} you got home?'' ~

``I met a girl there, very funny girl. Went to a pub.'' Glin compliments himself on the
nonsense nonchalantly rolling off his tongue. He keeps it up, ``And \ you don't hear the news there. The
important thing is that I came home safe and sound, right?''

``Right. On the Balad side three were killed and a few injured. They didn't say how many. Terrible. It's
all so terrible. When will it end?''

``Let's not dwell on it, mom. Weren't you supposed to sleep upstairs to-night? With Dela?''

``Zakod didn't go to that colloquium.''

``Why?''

``The ambush made Breaking News on TV at seven and people were told not to take to the roads until~the all
clear.''

``It's almost two, mom, go to sleep.''

``I brought down a tray for you -''

``Heartfelt thanks, mm -- or should I say 'bellyfelt' thanks?'' Glin responds, hopeful his
mother is satisfied. With a bit of luck she'll go back to sleep now.

But he's out of luck. ``Glin, where are your CDs?'' \ she ventures.

``Since when have you become so nosey?''

``I did some cleaning -''

``I discovered a music store in town where they exchange oldies for newies.''

``So where are the newies?'' Her agonizing worry again gets the upper hand.

``They're on order.''

``You never mentioned it.''

Glin is worried. Why has she suddenly become inquisitive? Suspicious? But he keeps up the patter, ``Music
didn't use to be one of your priorities. Now, I gather, it is -'' Glin yawns
loudly{,} ``Nighty-night, mom -''

``Ditto, son -''

Glin waits. Mom stretches on her bed, turns over once, twice. Finally she breathes the even breaths of~sleep. He tiptoes
to the door between their rooms to check. Yes, she's fast asleep. After shutting the door very~quietly he takes off his
thick overcoat and woolen sweater. Waiting for Yamik for so many hours outside in the cold he thought he'd freeze
\ despite that protection. He switches on the bedside lamp, tiptoes to the window and opens it and the shutter too as
quietly as possible. They've~opened without a sound. How fortunate that
h{e}{}'d taken{ }Yamik's wise advice to
oil them beforehand.

\ ``Yamik --'' he whispers.

Yamik materializes in the dark outside the window and asks, ``Where's your mother?''

``In the next room. Sound asleep.''

Muscular Yamik springs, cat-like, into the room.

``Welcome,'' Glin points to his bed.

``And you?'' asks Yamik.~

Glin lays a straw mattress on the floor and sits down on it. ``Am I not allowed to be the hospitable host
in my own pad?''

Yamik takes off his muddy boots and sits on the edge of the bed.

``Feel better now?'' Glin asks him. Yamik shakes his
head.{ }``I thought you got used to it -''

``There was time in the past when I got used to it. But now it's like starting all over
again.''

Glin considers this for a moment, ``I thought that once you get used to it, that's it,'' he
says.

``I also used to think that way,'' says Yamik. ``Well it doesn't - not with me anyway{\dots} Damn this
operation that went wrong,'' he hisses, then adds after a brief pause,'' I wasn't too near
the shooting, but when it suddenly{ }broke out I
just{ }wanted to get the hell out of there. Except that once I saw what was
happening to my friends I couldn't \textit{not} shoot.''

Glin shuts his eyes. Multicolored spheres collide with each other under his eyelids. ``I'd do the same for
you,'' he says,

``Do you understand what that means?'' Yamik asks. ``What killing
means?''

``I don't want to understand,'' Glin says. ``I want to believe that if it's
necessary, then \textit{it is}.''

~``You waited for me for hours, in the dark, in the cold,'' Yamik says. ``I'll
never forget that -''

``Had I been a sweeter person I'd have frozen into a fruit-ice.'' Glin can't help saying,
which tickles them both.

``So sorry about it,'' Yamik says earnestly. ``For the life of me I'd no idea
it'd take me so long to get to you. First the operation that went wrong, then the damned road blocks you had to
bypass.''

``It's okay. I survived. You survived. We both did.''

After a moment Yamik asks, ``Did you write those letters?''

``Yes.''

``Dating them the way I told you?''

``Yes.''

``The first dated to-morrow and the rest one week apart?''

``Right.''

``May I see them?''

Glin retrieves a packet of letters from its hiding place in the cupboard and hands it to Yamik.

Yamik counts them. ``There are only four here.''

``I didn't finish the fifth. The one that's going to be the first for her to read.''

``Where will you leave it?''

``On her bed.''

They fall silent again.

Yamik stretches out on Glin's bed. ``Let's get a bit of shut-eye now. We need~all our senses when we make
the swap.''

``How long can we sleep?'' Glin asks.

``A full hour. Do you have an alarm clock?''

``I'll set my mind to a piece of music~that lasts an hour.''

``You and your music -''

They share a chuckle.

After a \ minute's silence Yamik says, ``This attack on the bus came just at the wrong time. Now they'll
probably double security.''

``But we'll carry out our operation all the same, won't we?'' Glin tries to encourage Yamik to
encourage himself. Yamik keeps quiet and Glin asks again, ``Won't we, Yamik?'' Yamik's
silence continues. Glin doesn't break it.

Eventually Yamik says, ``I sometimes feel that I've already lived out my life.''

``You don't really mean that, do you -''

Yamik doesn't reply.

Glin begs him, ``Yamik, say something --''

``I must try and stop wanting \textit{to be}. Only then will I free myself of the fear of
death,'' he says.

What does he mean? \ Glin wonders. \ He doesn't ask him.


\bigskip

\chapter{}

It feels like there's somebody in the room. Rimat wants to open her eyes but her eyelids are lead heavy. For a brief
second she \ does succeed and realizes that it's Zakod.

``Rimat, wake up,'' she hears his voice, ``\textit{Rimat, you must wake up.}{}''\textit{ }His soft voice
doesn't conceal its harsh core.

{}``I can't{\dots}'' \ the two syllables mush together in her mouth and the words `\textit{Mom, I had to}.' throb in her
ears.

{}``\textit{You must force yourself to wake up!}{}'' \textit{\ }\ Zakod insists. He gently puts his arms around her
shoulders and pulls her like a rag doll into a sitting position. ``Here, I am putting a pillow between your back and
the wall. Lean on it. There you go.'' Once again she forces her eyes open. Her face suddenly feels cool. It's the damp
sponge in Zakod's hands.

{}``Try to keep your eyes open,'' \ Zakod urges, ``it seems that the sleeping pill I gave you was rather strong for you.
I brought you some lemonade. Try and drink it.''

{}``I can't,'' she mumbles. ``There's a horrible taste in my mouth -''

{}``It is the pill\textit{. Make yourself drink}.''

She takes a few sips. It's the lemonade she made for Dela.

{}``Here is a bit of hamburger I warmed up. Try and eat it.''

{}``No -'' and the words \ {}`\textit{Mom, I had to'} keep throbbing in her ears. Now she even hears them in Glin's
voice

{}``The more you eat and drink, the sooner the effect of the pill will wear off.''

Rimat gives in, munching and swallowing mechanically.

{}``How do you feel now?'' Zakod asks.

{}``OK{\dots}'' she mumbles.~ Now her eyelids lift a little. ``What's that light outside?''

{}``It is early evening,'' Zakod says.

{}``Such a strange light,'' she gazes at the window.

{}``Maybe because you are always upstairs at this time of~day.''

{}``What are you doing?''

{}``Checking your forehead.''

{}``Am I injured there?''

{}``Only a small cut. It was bleeding before. Now it has stopped.'' Once again he dabs her face with the wet sponge.

{}``That's nice,'' she says, ``I forgot that I fell -''

{}``Eat some more.''

{}``Don't force me -''

{}``Take one more sip at least - please.''

{}``OK -''

{}``And do not fall asleep.''

{}``I'll try -'' Things start to get clearer. Glin's terrible letter. The fall, the pain, the scream. The bleeding.
``Why did he decide \textit{now}?'' she asks, ``\textit{why suddenly now?}{}''

Zakod says, ``Why does the Vesuvius erupt at one specific moment in time and at no other?''

{}``You think this wasn't on/ impulse, but the idea of enlisting was already there?''

{}``The need of an individual to belong to a large social group is basic. It is one of the items in the kit called
Equipment For Survival. EFS for short.''

{}``Please, Zakod,'' Rimat~implores, ``please{\dots} no lecture now -'' She feels his hand enclosing hers. There has
never been such physical closeness{ }between them. Now it's vital to her so that she can ask, ``What's
going to happen to him now?''

Zakod is quick to respond: ``He has a very good chance of getting out of it unscathed even though he has
left civilian statistics and entered military statistics.''

``Where's his letter?''

~Zakod~ retrieves Glin's letter from the night table and hands it to Rimat. She can't get her eyes to focus on the
words.

\ {}``When will he get~some time off?'' \ Zakod ruminates \ and says, ``That will depend on several factors
which I am not familiar with.'' \

{}``What service will they put him in?''

{}``What he is most fit for - in my opinion - is Communications. Relatively speaking it is~less dangerous than other
services.''

{}``Communications?'' Now things begin to fall into place like pieces in a jigsaw puzzle. That~chance meeting with the
so-called stranger under the old fig tree. The conversations he had lusing his Walkman. The disappearance of the discs.
Of course! He must have been recruited some time back - but \ only now did it become official. ``May I close my eyes
for a bit?'' she asks.

{}``Only if you keep on talking.''

{}``How's Dela?''

{}``She is alright.''

{}``Did you give her her medicine?''

{}``Yes.''

{}``Did you give her{ }her lunch?''

{}``Yes.''

{}``And did \textit{you} eat?''

{}``I did. Do not worry. I told Dela what happened. She is also worried.~ She has become attached to both of you. She
loves both of you so very much. She loves to depend on you. Rimat''

And I love to take care of her, Rimat thinks to herself. But everything has now lost its meaning because of what Glin
did. He betrayed her. And he removed himself from her physical protection.

When{ }Zakod had rushed down at the sound of Rimat's cry and read the letter clasped in her hand he
immediately understood what had happened. There and then he had decided{ }not to share with Rimat what
he had known for quite some time - about Glin's conflict between two alternatives: ~remaining a~conscientious objector
or joining up. Had he done otherwise she would have blamed him once again. He didn't stop the father in the past -- and
he did not stop the son in the present.

~Zakod and I refer to Glin in the third person, Rimat thinks to herself. We don't mention his name, it's too
intimate,~too painful. \ \ ``Dear mom,'' he'd written, ``I joined up. I had to do it. Forgive me if you
can. I'll keep in touch. Your Glin.''

These words will keep throbbing in her ears forever. Until she's finished with this life.


\bigskip

\chapter{}

Coming home after seeing off the men who had called in Mother finds Aera still lying inert on the bed. She clears the
table. How ravenous they were! How thirsty! She was so happy to offer them food and drink. She must carry on with life,
go on working in the village for the sustenance of her daughter and Baby\textbf{ }and herself, and here at home she
must continue with housework.{ }But above all she'll continue to worry over Aera. Why has her daughter
become so distant? She never imagined this could happen. ~Their shared feelings and worries should have strengthened
the bond between them. What has come over her child? The~only one - out of her five - that survived? The perfect
daughter she's always been to her? What is it that's~gnawing at her heart? ~Aera shows no interest in the flower bed
she has planted outside the front door even though it was she herself who suggested it. ~She smiles only at Baby. It's
him alone she cares for, feels for, talks to, sings to.

Aera finally gets up and goes to have a look at Baby. He's sound asleep. She should help mother. She takes the broom and
sweeps out the dried mud left by the men's boots. She can't stop thinking of that old photo she'd found stuck
to{ }the underside of the cradle's mattress. She'd hidden it \ at the back of the drawer. She mustn't
forget to take it out and show it to Glin. Let that day come sooner than soon.

Mother is cheered at her daughter's sudden display of activity. ``I walked with the men all the way to the
fork in the road,'' she says chattily. ``Again I reminded them that Yamik took part in the
ambush of the bus together with them. \ Clear~proof that he isn't keeping imself apart.''

Aera is irked, ``Why do you meddle?''

``They feel that he has put distance between himself and them,'' Mother explains.
``That's the reason they came by.'' After a moment's silence she adds, ``As they
were leaving, the eldest of the bunch - who seems quite sharp - asked 'What happened to Aera? Why does she keeps so
much to herself? Why doesn't she say anything?' I told him that you're tired.''

``You said the right thing, mother darling,'' Aera wishes to reassure her. The sweeping
accomplished she goes back to bed.

Baby whimpers. Mother picks him up, ``He cries more when his pale blue flower cover is in the wash and he
has to make do with a substitute,'' she says.

``That's wishful thinking, mother love,'' Aera says, ``but be my guest
-''

~As Mother changes~Baby's diaper, she whispers to him, ``You beautiful Baby - ~you and I know the truth.
Your favorite sheet with the pale blue flowers will be washed and ironed for you by to-morrow.'' She
brings Baby to Aera. For a while the shack is filled with sound of hearty sucking. When Baby has had his fill, Mother
takes him from Aera, burps him to her satisfaction \ and then lays him~back in the carriage. Wheeling him back and
forth she hums the Pale Blue Valley song. Baby falls asleep. She studies the infant's relaxed features,
``I see your face in his'' she says, ``Yamik's too. Separate and mixed together -''

Aera says, ``I see many faces in his face -``~

Mother thinks, it's been so long since Aera expressed such{ }pleasant words.

``That good man from the village fixed the cradle onto{ }an excellent metal frame with
wheels. And now Baby has a real baby carriage. 'Golden-Hands-Gidal' they call him there,'' she tells her, ``and rightly
so.''

She puts on the kettle to make some tea for Aera. With her back to her she says, ``I didn't tell their
leader, the clever one, that you keep quiet altogether.'' She brings Aera a cup of tea with milk and a
piece of cake.

``Thank you, mother dear,'' Aera says, ``but I'm not thirsty, I'm not
hungry.''

``It's not for you,'' Mother says, ``but for my grandson. Eat up and drink
up.''

Aera heeds Mother's bidding. Afterwards she again stretches herself on the bed and turns to the wall,
``Mother darling, tell them at work that you need to have a day off next week because you have to baby-sit
for your grandson. I want to meet Yamik in town. I'll prepare~a bottleful of milk for one feeding, and if I'm late for
the next one there's ~milk powder in the fridge.''

Mother is more than happy to comply. ``Good,'' she says, ``if the weather's nice, I'll stroll with him
outside in the fresh air. Now that you made this clever pouch to attach to the carriage, I can put his bottle inside
and he can have a picnic under a tree .''

``Make a sandwich for yourself too,'' says Aera cheerfully, ``so that you can
join him in the picnic.''

``It's so daughterly of you to think of me,'' Mother retorts in kind and adds,
``I'll send some goodies with you for Yamik.''

``That will be great,'' says Aera.

``Won't you tell me \textit{anything}?'' Mother implores her.

``Don't ask me, Mother darling, and I won't have to make up stories.'' Aera turns to the wall.

Mother says, ``But now \textit{I }will tell \textit{you} something that I never told anyone. Never talked
about it. It happened years ago when I was young, not yet married. We lived in a small village far away. It was after
they had expelled us from Pale Blue Valley. One day we saw a man on~horseback galloping towards us. ~When he came close
we realized he was an officer in \textit{their} army. He said there was an order that we'd soon have to move~elsewhere.
And then he galloped off. A huge commotion broke out{\dots} what to take with us? {\dots} what to leave behind{\dots}?
I hid my father's axe under~my coat, got on a horse, rode after that officer and called after him. He stopped. I got
close to him as though I wanted to offer him some food and drink. He leaned towards me and then - with all my strength
- \ I struck him with the axe and he fell off his horse. Dead. On the way back I threw the axe into a deep gully. I
said not a word about it -- not to anyone. That night they pounced on us. They asked who killed the officer. Everybody
said it was \textit{not} one of us but \textit{they} said it was. And they went through the village killing and raping,
and drove us out. And we wandered and wandered until we found a deserted place to settle in. I've never told anyone
what I did.''

After a long silence Aera's voice is heard, ``Would you do the same to-day?''

``No, of course not,'' says Mother. ``Today I would act more wisely. Today I
wouldn't act alone but join with others.''

``You knew very well what I planned to do to that guard in the village. Why didn't you tell me what you
yourself had done in the past?''

``It was my secret.''

``It may have encouraged me.''

``I didn't think you needed encouragement.''

``Not everybody is as brave and strong-minded as you.''

``Aera,'' Mother sits down next to her on her bed and pats her shoulder, ``Yamik
\textit{does} distance himself from the men. I feel it. What is he planning to do on his own?''

``An exchange of secrets{\dots}?'' Aera hides her face in Mother's bosom and bursts out
crying. ``Mother darling, shhh --''


\bigskip

\chapter{}

Janha,~pen in hand, sits facing the CEO ready to take notes.

``I'd like to extend \ the meeting with the Selected Cadets by an extra half hour,'' he says,
``a half-hour meeting is pointless.``{ }

Ezlip, sitting next to the CEO, objects, ``You mean to waste another half hour on that?''

``Half an hour here or there won't make much difference in terms of waste,'' the CEO smiles at
his aide and returns to Janha, ``I know I can trust you to juggle the timetable. ~And can we introduce an
innovation - that from now on each soldier{ }carries a name tag.''

``Very good, sir,'' says Janha jotting down the CEO's two requests, ``and may I add --''

``Don't bother with the 'may I','' the CEO smiles, ``I take it{
}as an order -''

``I hear that this tour provided for Selected Cadets is going down very well with them,'' the
secretary continues, ``I also hear there'll be a column about it in the press, accrediting you with the
idea.''

Bonimi accepts this compliment with his usual grace. He doesn't relax his hold on the list of Selected
Cadets{ }scheduled for the next tour when Janha stretches out her hand to take it. ``I'd
like to go over this again,'' he says. ``There're quite a few interesting details in some of
the boys' CV's.'' \ {}'Saffia, Glin' among them shrieks at him. He initiated these tours for Selected
Cadets hoping the that this would happen. And{ }it did!~~

'' I'll have to notify security about lengthening the meeting,'' Ezlip heaves a theatrical
sigh,

``I hope that won't be too much~trouble,'' says Bonimi, fully aware that his aide's
reservation about the extra half hour stems from the fact that the idea of that guided tour and the{
}ensuing meeting with the CEO did not come from \textit{him}. ``And as for security,'' he
goes on, ``according to the last briefing, I understand that the security level has been somewhat relaxed.
They don't see a threat lurking round each and every corner.'' He puts on an impromptu mime of
``searching'' -- which even elicits a short laugh from Ezlilp - and then continues, ``Who knows whether it
isn't meant to boost a sense of confidence.''

``In whom?'' Ezlip asks.

``On our side, on theirs,'' the CEO shrugs as he waves Janha farewell.

This trait of his boss's, Ezlip believes,{ }the CEO's friendly and tactful
behavior with everyone who crosses his path, should also be highlighted when he finally agrees to stand for~public
office. ``If I may be so bold, sir,'' he says, ``I'd like to repeat my suggestion that you go into
politics.''

``I admire your perseverance,'' the CEO says.

``First of all, you've become a public figure,'' Ezlip is all too happy to harp once more on
this his favorite theme. ``And secondly, you have natural gifts and leanings.''

``Like what, for example?''

``The ability to think and~decide.''

``I'm beginning to impress myself,'' Bonimi says, ``starting to
blush.''

``And, third,'' Ezlip continues, ``there's a heroic chapter in your biography.''

``What are you talking about?'' Bonimi asks. Here comes the welcome itch!~

``Your brother died {\dots} as a hero in the so-called Phony Occupation.''

``It's my brother. Not me.''

``But you can be identified with him.''

``No way!'' The scar's beginning to throb. How Bonimi loves it. He's tempted to touch it but
naturally holds back.

What a shame the boss didn't take his advice and have his scar removed~by that plastic surgeon - the thought goes
through Ezlip's head. Still, even with that scar on the left cheek, his picture in the media alongside mention of his
heroic brother will win him quite a few points.

~

\chapter{}

``Rimat, there is a big surprise for us in the paper today!''

These remarks made by Zakod sitting in his armchair in the living room to her in the kitchen always irritate her. They
break her stream of inner thoughts. {\ }How many times does she have to
explain{ }that the fact that she's busy with housework doesn't mean her mind is
free for all? But in vain.

``There is a write-up here about that big inauguration ceremony for the renovated auditorium in the TV
building,'' he goes on, ``Now I know why our poet Alir was appointed to emcee it. Do you know who is Chief Executive
Officer of our Public Television Network? Bonimi Saffia! Our old Boni! Your brother-in-law! Never too late to be
updated.''

``Really,'' Rimat emits softly and continues scraping the carrots. She~doesn't tell him that
Gidal had long ago shared with her that piece of information. Nor has she told him about her guided tour of the TV
building and its two outcomes - having it out with Alir about that terrible poem he wrote and incidentally catching
sight of Boni. She preferred to keep all this to herself. ~

``I am so sorry,'' Zakod apologizes. ``I forgot. I have invaded the sovereignty
of your inner thoughts. Still, I allow myself to continue. I guess I can put two and two together. Bonimi appointed
Alir to emcee the inauguration ceremony for old time sake - when the two of them shared the glassed-in veranda at your
place. Am I right in feeling that there is no contact between you and Boni?''

``Enough said.''

Zakod goes back to his newspaper.

``I'm going to go to that event,'' she declares: she'll \ know then if Alir keeps \ his
promise to her.{ }

Zakod is taken aback. ``I would never guess that you would go in for such nonsense,'' he says.

``Wonders never cease, do they?'' \ she retorts.

``I shall order a taxi to take you there and back. And do not start protesting. You should not be alone on
the roads at night. ~Especially so since this grand splash might end in ungodly hours - if at all.``~

Zakod comes into the kitchen newspaper in hand, and sits down at the table. He continues to talk to Rimat's back as she
peels vegetables for the soup. ``Saffia and Boni were so close. How they loved each other. How devoted
they were to each other{.} \ The nicknames they used to call each other -- 'Big-Saff' and 'Li'l-Saff'.
They were not at all alike - neither physically, nor mentally. But one trait they shared -
~{ }their basic human feelings for their fellow men. Yes. I am quite sure that
it was for old-times-sake that Boni asked Alir to emcee~this hullabaloo. It is also a brilliant idea. As far as I
remember, Alir cannot but be authentic, disarming, spontaneous. He will charm and impress them all. Did you happen to
come across the poem he wrote about the Balads?''

Her first impulse is to deny it. But then a second impulse takes over. Zakod judges every issue according to its merit.
She would very much \ want to hear what he thinks of the poem. ``Yes. I did. Did you?''

``I certainly did.''

``And what do you think of it?''

``A very powerful poem. Compressed in about twenty words, cadenced in three lines, Alir succeeded in
conveying a whole powerful idea.``~

``This is what you say about a poem that encourages aggression? ~Incites killing?''

``If this is your question it means that you too, my dearest, have failed to understand the last line like
too many others. Including those unfortunate Balads.''

Rimat quotes from memory, 'Next to a throwing hand, I'm the grenade' -- how else can that be understood?''

``The poem progresses from the general to the particular,'' Zakod explains in his own pedantic
way. ``The first line -- 'Next to a human being, I'm a human being' --~is about the overall universal -
\ the kind of identification that any individual can feel towards his fellow man on the simplest, most basic level. The
second line -- 'Next to a Balad, I'm a Balad' -- here we are on the level of a specific human predicament that is so
very well known to us. But it has to be understood that the 'Balad' in the context of the poem does not only represent
a Balad who lives among us, but also the Balad Experience, capitals B and E.``{ }

{}``What does that mean?''

``It is about being a displaced, homeless individual, being at the mercy of others. And the third, final
line, 'Next to a throwing hand, I'm the~ grenade'--~ is the basic ethical issue. That when something base, despicable,
horrible, happens in the world, like it or not, we all have a share in
it.``{ }

Could that have been Alir's line of thinking? \ Rimat wonders. \ A line of thinking that I was blind and deaf to? And
when I came to have it out with him he couldn't explain it to me, because he felt like a frightened schoolboy in an
oral exam? She turns to{ }Zakod, ``But most people understand it
quite differently.''

``What can one do when most people's minds are~limited.''

``Writers and poets should write so that people can easily understand them -''

``Rimat, what are you saying? You are totally wrong! Writers and poets should gear~their creativity to most
people's deficient level of comprehension?''

Rimat sits down at the table facing him. His explanation is beginning to grow on her. When the event is over she must go
up to Alir and apologize for having demanded that he~correct{ }that last line.

``It was nice of Glin to ask about Dela and{ }myself \ and send us his~
regards,'' Zakod says. ``Please return ours to him.''

``I certainly will,'' Rimat responds. Since it's Zakod who brings in the mail from the mailbox
out there on the main road, he'll surely have noticed the \ two previous letters - the first and second - that Glin's
had sent her. She so appreciated the fact that Zakod didn't ask her about their contents. They were too intimate for
her to share. ``So good to know'', she says, ``that Glin seems to take basic training in his
stride.''

``So good indeed,'' Zakod agrees.

``And I hope that eventually -- as you can imagine -- they'll put him in Communications.''

``Safe and sound there,'' Zakod is happy to chime in. ``And far-far from those
unsavory frictions with the Balads.'' The atmosphere between the two of them now is open, pleasant. It's a
propitious moment to ask, ``Do you remember the morning of the explosions?''

Rimat forces herself to be calm. She says, ``I mainly remember my shrieking~that morning -''

``Do you remember that I had come over to your place a few hours~earlier?''

``Yes{\dots} No... It's all so hazy -''

``It was still dark when I knocked on the front door. By the way you looked when you opened it I realized
that I had woken you up. I apologized and said that I came to speak with Saffia. You said that he had already left. I
asked, 'For work?' You said so very curtly, 'Yes'.''

``The only thing I wanted{ }at that moment was to get rid of
you,'' Rimat recollects. ``I was bursting with milk. Pure agony. ~After you left I picked
Glin up, nursed{ }him and then laid him next to me in bed. We dozed off together. And then it hit me:
why had Saffia left so early without telling me? When it was still dark? Then it dawned{ }on me. I went
crazy, ran out to the veranda yelling - 'Boni! Saffia's gone out! Did you give him the pump? Did you?' And Boni woke
up, pulled the precious pump from under his bed and rushed out with it. I went back to Glin, covered both of us with a
blanket and mumbled, 'No-no-no'. I thought that if I kept on mumbling 'no', it wouldn't happen. But then there was a
huge explosion. And then another and another. And everything started to shake. \ Sirens wailed from all directions. And
suddenly Shouba appeared. And I let out a scream that has never stopped --''

Zakod waits a minute and then says quietly, ``That night, when I came over to your place,{
}it was for the first time after a very long absence. I had not been back{ }to
your place since that last heated clash between Saffia and myself that ended in~a bitter falling-out. I walked out
banging the door behind me never to return. ~Some time passed. And \textit{that night} I was suddenly hit by the idea
that Saffia might indeed carry out his plan \textit{imminently}. And I wanted to try once more to talk him out of it.
Using all my powers of persuasion make another attempt. I so loved Saffia!. I never loved a man~the way I loved him. I
never shall. I went to your place and knocked at the door. You opened it and set my heart at rest when you told me that
Saffia had left for work. So I went back home with a sense of relief. \ It was on the way that I heard the explosions.
I realized then that I should have trusted my premonitions. So I ran to the citadel. By the time I got there they had
cordoned off the area. I tied a white handkerchief around my arm to look like First Aid. The whole place was in flames.
Smoke. Walls collapsing, roofs caving in. Where was Saffia? I ran about like a madman. It was a scene from hell. A
blood bath - just as I had predicted. In the funeral six children and thirteen adults were buried as if individually -
to spare the families deeper grief. Saffia's friends stood far back during the ceremony. It was whispered that he was
among those buried in the mass grave of the unidentified. \textit{But that was ~not true}. I know what happened. That
night -- heading with his loaded bicycle to the citadel-- that night at long last he understood that no aspiration, no
idea or ideal, can justify killing, blindly murdering. And he found himself in the clutches of excruciating inner
strife. On the one hand, his desire to execute his plan was gone, while on the other he did not have the strength of
mind to admit to his friends that his plan of action was basically, tragically, flawed. So he fled. Disappeared. That
was his solution. Later, when I heard that you had gone away with Glin, I concluded that you made arrangements to
reunite in some distant place. But as time elapsed and I heard about you being alone with Glin in that village in the
north, I reached another conclusion. That night Saffia ran away from\textit{ you too}. He could not face you. It was
then that I began to write to you.''

``You're out of your mind,'' Rimat says under her breath, wondering why she's still sitting
opposite Zakod at~the same table. ``How do you explain the explosions in the citadel?''

``The citadel was an arsenal of ammunition. There was a mishap. A frick accident. Whenever such an accident
occurs, they always blame underground cells first. History is full of such examples.''

``You and your examples in history-''

``In history there are also examples of Saffia's insoluble predicament,'' Zakod stands his
ground. ``Not many, but there \textit{are} some.''

Rimat freezes. She's overwhelmed. Maybe there's a grain of truth in what Zakod says? \

``He{\dots} he left me{\dots}?'' she manages to mutter. ``We had our disagreements. Even quarrels. He was
an angel. I wasn't. To think he \ could be living a different life? Another woman? Other children?'' She
pulls herself together. ``No! No-no-no! Saffia and I were one. Body and soul.~ You can't understand this.
You have a logical way of thinking that can't be argued with. But you have no proof. And I'll tell you why you think
this way: it's a dodge. Because you too are guilty of what happened to Saffia. You stopped coming over because you
didn't want to have any more of these fights with Saffia. The luxury of a professor in Academia. You \textit{should've}
continued to come over, \ keep on with those bitter disagreements. \textit{To save Saffia from himself}. You are
guilty. And Alir is guilty because he ran away to write his poems. And Shouba is guilty because the bomb-timers he made
were defective. And Boni is guilty because he always hid his precious bicycle pump under his bed. And Michlor is guilty
because he crammed for his exams instead of rushing to give Saffia first aid. And Gidal is guilty because he was a
yellow-belly from the start. You're all guilty. Every one of you. And I too am guilty. That I didn't throw myself down
on the front steps, block his way. I was exhausted. Tired to death. Nursing the baby, the endless diapers, his crying
at night. The~only thing I wanted was to sleep. Just sleep and sleep.''

``Are you still waiting for him?'' Zakod ventures. ``Only in books and films or
the theatre does a husband come back after eighteen years.''

``I'm not waiting for him,'' Rimat nods her head. ``I mourn him. For me, for Glin
who could have had a father like Saffia. He didn't leave us. He died. I don't know where. But he died. I can't prove
it, I just know he died.'' Rimat goes to the kitchen door leading to the staircase and opens it to go
downstairs.

``Rimat,'' Zakod says to her softly, ``we two are lonely people -''

``I'm alone. But I'm not lonely.''


\bigskip

\chapter{}

Finally she appears at the end of the path. At long last! Glin's body has grown stiff from standing
tense{ }all this time looking out of the window. What's she holding in her
hands - a basket? What's stopping him from running up to her? \ It's his tears.

Aera walks in, looking for him, ``Glin?''

~He steals up from behind, puts his arms around her. They kiss.

~``Aera~ --''

``Glin -''

``Aera~ --''

``Glin``-

``I \ went out of my mind whenYamik told me you would be coming -''

``And I cry all the time. I was never like that, but now I can't help it -''

Glin licks her tears then dabs his own{ }with her hair, ``Your
shining hair smells warm and sweet, like milk -''

``I cut it because of you. And now I'm growing it again for you.''

``Yamik's on his way, he'll soon be here,'' Glin says and then \ steps back, throws
out{ }his arms, flapping the baggy~sleeves and pinching the loose material on
both sides, ``How do I look in Yamik's shirt? Like a scarecrow? ''

Aera laughs, ``Come on, put on a bit of fat.'' She looks around. This room - \ like the rest
of the \ building and in fact the whole neighborhood \ {}- is half{{}-}built.
Resting on a pair of trestles there's a rough wooden plank with two sets of Walkmans and earphones on it.
``Just a sec,'' Glin gently pushes them aside and she produces from the basket a \ red and
white chequered tablecloth and spreads it over the plank. ``How are you?'' She ask.

``Hungry. And how are \textit{you}?''

``Starving.'' She deftly empties the basket of crockery, cutlery, food, drink and puts them on
the table.

``That basket of yours is a horn of plenty!'' Glin exclaims at the set table. He brings over
three empty wooden crates instead of chairs.

``Bon app\'etit!''

``Bon app\'etit!''

``Salute!''

``To you!''

They clink glasses and launch into the food. To disguise the milk stains on her blouse Aera -- acting like an unruly
child - purposely lets the food spill down her front.

Unobserved Yamik, approaching noiselessly in his commando boots, stops~at the doorway only to see his wife dining
intimately with his close friend at the makeshift table.

``Hello, good friends!'' he greets them as he enters the room. ~

They look up. ``Hello to you,{ }good{ }friend!'' They returns
his greeting.

He comes up to them, removes{ }his military cap, takes off \ the rucksack and
unbuttons his shirt. He then lays a friendly hand on Glin's shoulder - careful not to touch Aera.

``Go ahead and show Aera,'' Glin prods Yamik.

The short moment \ Yamik leans over towards Aera is sufficient for her to read the name engraved on the
dog{{}-}tag he's wearing, 'Saffia, Glin', beneath it some seven-eight digits.

Glin points to the rucksack, ``Looks heavy -``~

``Military~surplus,'' Yamik winks.

Aera asks, ``What's it for?''

``Contributions for the men,'' Yamik is quick to answer.

A furrow deepens on Aera's forehead, ``I thought you'd distanced yourself from them -''

``It's always worth while keeping up some contact.''

``What for?''

``Aera,'' Yamik tries to curb his impatience, ``trust me.''

But Aera must verify, ``It's not about the TV building{\dots} \ or is it?''

``Why on earth?'' Yamik hastens to put her mind at ease. ~

``Where do you keep it?''

``Hidden iIn the ground under those bushes,'' Yamik points outside.

``Look what Mother sent,'' she invites him to the makeshift table.

``Auntie's home cooking!'' he beams as \ seats himself on the third crate.

``Are there any changes in the plan?``~Aera asks,

``Why aren't you eating?'' Yamik responds with a question.

``I've eaten,'' she says. ~

``Come on, Aera, eat some more!'' \

Doing her best to please him, Aera forces a few more forkfuls of food into her mouth. ``I asked if there're
any changes -?''

``None whatsoever!'' Yamik is emphatic, and once again pleads with her, ``Eat up, drink up
-{ }the more the better!''

Again she yields to his plea. Most of what she drinks spills down the front of her blouse till it's now one big wet
stain. She clears a space on the plank in front of her and on it lays a sheet of paper she takes out from her basket.
``This is the declaration I've drafted.''

Yamik bends down to scan it. ``Excellent,'' he says, turning his eyes to Glin, who responds
with a questioning look.

Intercepting this mute back-and-forth between the two, Aera realzlizes that Glin knows nothing of Yamik's short
nocturnal visit to the shack after she'd sent him word of Baby's birth. Yamik rushed up to the shack \ drunk with joy.
He wanted to eat Baby \ up. And he{ }couldn't get enough of kissing her. ~Mother -- in her wisdom -- had
left the new parents to themselves for the short duration of that visit. Among the few words the couple exchanged,
Yamik had confided in her the details of the TV operation. And then made her a request. Since her knowledge of the
official language was far better than his. and since she naturally \ far more familiar with the ideal of Pale Blue
Valley than Glin, he'd asked her to draft the declaration to be declaimed once they take over the inauguration of the
the new TV auditorium. He told her he would send her a message via the men about when and where to deliver it to him
and Glin. She was delighted to be thus involved. As soon as Yamik left the shack she set to work on her
assignment{. }

``Where will he sit?'' she asks Yamik as though Glin isn't present.

Yamik produces a ticket from his shirt pocket, ``Here's his ticket.''

Aera takes the ticket and reads out aloud, ``Public Television Network New Auditorium Inauguration
Ceremony. Row 7. Seat 15.''

``It's next to the aisle,'' Yamik explains.

Aera hands the ticket to Glin without looking at him and asks Yamik, ``And \ what happens after he finishes
reading the declaration?''

``What do you mean?''

Aera stifles her anger at this pretense of naivet\'e, taking care not to reveal her worry, ``How's he
supposed to get out of there?''

``What's there to worry about?'' Glin cuts in{.}

Aera looks right through him, ``Won't you be apprehended?''

``I'll threaten them,'' Glin says. ``I'll warn them that they'd better release me
and leave me alone, not followed me! I'll definitely threaten them!''

``With what?'' Aera mimics wide-eyed childlike curiosity.

``With{\dots}'' Yamik interrupts with a slight stammer which he immediately overcomes,
``that's to say that if they do anything to Glin the whole world is going know about it! The whole media
is listening, watching!''

``Just as well that you're telling me this,'' Aera says, ``because I need to add
it at the end of the declaration.'' She takes out a pen and writes down the necessary addition at the end
of the declaration she's drafted.

Yamik and Glin make eye contact.

Aera puts her pen back in her handbag, turns to Glin and asks perfunctorily, ``Where will you go after
you{\dots} get out?''

``Yamik will leave a car for me in the next street,'' Glin says. ``I'll drive out
of town, then stop and get out of it at some previously agreed spot between me and Yamik, get into a car which Yamik'd
have left for me there, change clothes and identity card, drive on some distance, then slow down almost to a stop and
jump out. The car will roll along until it falls over the cliff edge. Total loss.''

Yamik takes over, ``Glin'll disappear there into a \ stocked with all the necessary provisions for a few
days...''

``Among them a pile of my discs,'' Glin interjects with a chuckle

Aera says, ``Vital for you.''

``And I'll join him a few hours later,'' Yamik again takes over. ``We stay
put~there for a few days. And then we'll get to another hiding place and then another. That's what we'll do for a
while. Nothing to worry about. How's my aunt?''

``She's fine,'' says Aera. ``She didn't go to work today.~ One of the nieces
asked her to baby-sit for her baby.''

``How is that baby doing?'' Yamik asks.

``Fine,'' Aera says. ''He's coming along very very nicely.''

``Does he have a name yet?'' Yamik asks.

``Not yet,'' Aera says. ``They're waiting for his father to come home and help
choose a name for him together. Meanwhile{ }he's called Baby. ''

``Let's hope Baby's father comes home soon,'' Yamik says. ``I have to leave here
in quarter of an hour,'' he goes on, ``I'm a candidate for a 'Selected Cadet' citation and I mustn't endanger it by
getting back late to the base. I can~walk you to the bus station later.'' He turns to Glin and points at
the rucksack, ``Could you{\dots}''

Glin heaves the rucksack onto~his shoulder, takes hold of a shovel{ }propped up
in the corner and goes out.

Yamik and Aera are now alone together. He buttons his uniform. She collects the~dishes, folds the tablecloth, puts
everything back in the basket. Only the sheet of paper remains on the rough wooden plank.

Outside{ }earth{ }is hacked and shoveled.

``He'll do anything for you,'' Yamik says. ``He's not a standup comedian
with{ }only music in his head.''

Aera says, ``But what's going to happen to him afterwards? On \textit{their} side he'll obviously be burnt
out.''

``I'm thinking about it,'' says Yamik. He wishes
Aera~would{ }stop needling him on the subject.

Aera, afraid she may be too open about this worry of hers, tries to sound matter-of-fact, ``I'm sure you'll
find a way.'' She scrutinizes Yamik's face trying to guess what's going on inside his head. \ Plucking up
courage she asks bluntly, ``Were you \ with the men in the bus operation?''

Yamik gulped. '' I{\dots}well, I was around there -''

``So you did make contact with the men -'' she wants to look
him{ }straight in the eyes but he averts them. ``Why do I keep
asking you questions when I don't know whether you yourself believe in your answers.''

``Oh, Aera, it's too complicated,'' Yamik sighs. ``If \ I detached myself
altogether, they'd be forever snooping on{ }me. In that case I certainly wouldn't be able to act
independently. I'll send you a message the first possible moment.'' Resting his elbows on the now bare
plank of wood he holds his head in his hands. ``I'm tired, my love. I want to be home. With you. I want
us{ }to choose a name for our baby.''

Aera knows that this is the moment she should grant Yamik the warm hug he deserves, but she doesn't make a move. And now
that Glin has returned with the shovel and the empty rucksack she's missed the moment she could have \textit{not}
missed, that she \textit{should} \textit{not} have missed. She hopes Yamik didn't sense that when he asks her - \ as if
continuing their chat, ``Did you have any problems getting here?''

``The bus was stopped twice,'' Aera answers. ``Once at some place along the
highway and once entering town. They searched the basket but didn't pay any attention to the declaration. They took it
for a bit of scrap paper.'' After a brief pause she asks, ``What exactly will the two of you
be doing at the ceremony?''

``What needs to be done,'' Yamik is quick to answer.

``Show me -''

``Glin and I~ have been through it more than enough times already.''

``But I won't be there,'' Aera protests pouting, ``I drafted the declaration.
Don't I have the right to see how it's going to be read out declaimed?''

``Okay,'' Yamik yields.

To clear some space the two men push \ the three crates and \ the wooden plank and the trestles
away{.} Yamik points with his hand, ``Let's say that the stage is
over there. The dais is in the middle. The CEO and the MC and some VIP's sit behind it. The stand with the mike -- this
spade - ~is over here. The audience is over there. Glin sits in row 7 next to the aisle. The ceremony
begins.''

``Wait a sec - '' Aera interrupts him. ``What happens if something goes wrong the
last minute and you're not where you're supposed to be?''

``There's ~a safety valve,'' Glin takes over. ``First thing I do when I reach the
TV building I pass by Yamik to check that he's there. Only then do I continue into the building.''

``Throughout all this Glin and I will naturally be in contact -- he inside and I outside -- \ by means of
our communications system,'' Yamik adds.

Aera asks him, ``Where will you actually be standing?''

``Behind the building,'' Yamik says. ``I'll be~the only one
posted{\textbackslash} positioned there.{ \ }May we please go
on?{\textbar}''

Aera \ nods graciously.

``So this is the setup,'' Yamik resumes his description. ``Glin is in his seat
with the earphones on, removing them from time to time.''

Glin, sitting on one of the crates, his Walkman in place, demonstrates Yamik's description of him. He says,
``This is so the{ }people around me get used to this weird
character who can't do without his Walkman even during the ceremony. I'll also whisper something~to that effect to
whoever is lucky enough to sit next to me.''

Yamik takes hold of~the other Walkman and puts on his earphones, ``I'm now walking to the spot behind the
building where I'm stationed.'' He walks away towards the end of the room and almost disappears in the
dark there. His voice carries from there, ``Glin, can you hear me?''

Glin says, ``Yes.''

``Give me the sign -''

Glin clears his throat three times.

``Good,'' Yamik confirms and then continues, ``let's say that about half an hour
has elapsed since the beginning of the ceremony. I now count ten, nine, eight, seven, six, five, four, three, two, one,
zero{\dots}''

Glin dashes to the imagined stage and grabs the{ }spade-cum-microphone. He takes a sheet of paper out of
his pocket, flattens it and starts to read aloud, ``Ladies and gentlemen, please permit me to present
the{\dots} ''

``Hold on!'' Aera interrupts him, ``What are you reading?''

``The declaration,'' says Glin.

``But that's not the declaration I drafted -'' Aera spins round to look at Yamik but he's
checking his watch. ``I didn't know you also had drafted a declaration,'' Aera addresses Glin
who is watching Yamik perplexed.

``He did it for our practice,'' Yamik explains.

``So let's compare them-'' Aera suggests in a spirit of camaraderie.

``Oh, no --'' Glin folds his piece of paper and pops it in his pocket.

``Oh, yes --'' she presses.

``Time's running out!'' Yamik announces~impatiently from the far the end of the room.
``It's late!''

Aera ignores Yamik and begs Glin, ``I'd really like to -''

``But I'd rather not,'' he says.

``Why?''

``I'm embarrassed -''

``About what?``~

``My spelling mistakes.''

Aera laughs, ``In your own mother tongue?''

``Yes. I was a slow learner, and still am -''

``So let's compare our mistakes then!''

{}``There you are,'' Yamik intrudes, emerging \ from the dark, ``now you have a picture of
what's going to happen. Let's go. I'll jeopardize my citation of Selected Cadet \ if I'm late.'' He picks
up Aera's basket, hands it to her and gives Glin his last instructions, ``Take a clean shave. Comb your
hair. Put on the jacket I brought you. And don't forget the tie.''

``I'll be the picture of elegance,'' Glin smiles. They shake hands. This is the last time they
see each other before the operation. On impulse they hug. Glin says, ``I never had a friend like you
-''

``Neither did I,'' Yamik responds and hurries to add, ``I almost forgot
-'' he takes a picture postcard from his shirt pocket and~gives it Aera, hanging onto her hand for another
precious second. ``Guess what it is --''

Aera takes a swift look at{ }the card, ``I've no idea.''

``It's a photo of Pale Blue Valley.''

``How did you come by it?''

``They gave us a tour of their National Library. In the photocopy department they said we could each ask
for a copy of the photo of our choice. So I leafed through some of their history albums - and found
this.''

``The colors are too bright, too glossy.''

``No matter,'' Yamik is peeved at Area's reservation. ``Please give it to my
auntie.''

``No way. She'd be very vexed that it's not like the real thing. You know how she is.''

``Auntie will understand and be happy with it,'' Yamik insists. Having pushed the Walkman and
the earphones into his empty rucksack he now slings the bag on his shoulder, puts on his army cap with a flourish,
stands to attention and gives Glin a comical salute. He checks his watch, ``I still have two minutes to
look around. I've done my share of toil and sweat here, right?'' Turning to Aera he tells her ``I'll wait
for you at the roundabout.''

Aera gets it: he's \ leaving her and Glin to say their goodbyes without his presence; she appreciates this gesture.
During \ this little space of time she'll do her best to encourage Glin in \ his critical task in the TV operation. It
also gives her a few more minutes alone with Glin who already has his arms around her. They kiss.

``Just imagine that we're after all that --'' Area says and suddenly realizes she's
\ forgotten to bring along that faded photo for Glin to identify that man - face puffy, hair disheveled, eyes
unfocussed. Hopefully she won't forget to bring it next time they meet. May that be sooner than soon.

~

\chapter{}

``Here's your ticket,'' Alir hands the ticket to Shouba. ``It's called 'a
complimentary ticket'. So feel properly honored.''

``I'll try and rise to the occasion. I planned to listen to it over the radio,'' Shouba says
as he scans the ticket. ``Maybe even watch it on TV if I get around to fixing that old piece of
~junk.''

``I'll be offended if you don't come,'' Alir says.

``I shall certainly come,'' Shouba promises. ``Shall we run through it
again?''

``No,'' Alir nods his head from side to side. ``I've reached a point where any
more rehearsing will only destroy the bit of self-confidence that you managed to instill in me.''

``Let's drink to your success!'' Shouba goes to the kitchen to make tea and calls from there,
``What you decided to say about the third line is perfect. It's like a further experiment that reconfirms
previous results.''

Alir says, ``If it gets to Rimat, it might convince her.''

``It most certainly will,'' Shouba peers into the room with the tea kettle in hand.

``When I wrote the poem -'' Alir reflects, ``it simply burst out of me. Like a geyser.''

``I remember,'' Shouba says. ``You wrote it~here as we watched~the evacuation of
the Balads on TV. That little girl crying over her doll~that slipped from her hands and fell to the ground
--''

``Now,'' Alir explains, ``when I thought about the third line quietly, from a
distance, and planned to correct it, change it, and at the end decided to let it be, I feel that I grew up. I matured.
At long last~I've reached my age. Now I feel to be deserving of Saffia's friendship.''

Shouba nods in agreement while placing the kettle on a pile of frayed blankets. The kettle sinks dangerously into the
depression it makes. He hastens to grab it before it topples over. He looks for a more secure place for it.

``Listen, I want to tell you what occurred to me about what happened to Saffia,'' he says when
he finally entrusts the kettle to the window sill. ``Do you remember what Zakod used to tell him? That
he'd be blamed for a blood bath?''

``I don't -''

``This is what he used to tell him. All the time. A blood bath. A slaughter house.'' Shouba
stops. When he speaks again he's uttering each word separately, ``So maybe at the end, when Saffia~was on
the point of setting the bomb-timer{\dots} perhaps{\dots} at that last moment{\dots}~\textit{he in fact set off the
bombs}{\dots}''

``What do you mean...?'' Alir asks \ aghast.

``That he activated the bombs and \textit{remained there}{\dots}''

Alir and Shouba look at each other. They bow their heads eyes closed.


\bigskip

\chapter{}

``As for hiring new employees in this establishment{\dots}'' -- explains the CEO as he \ winds up the
meeting with the selected cadets - \ {}``we advertise in the newspapers when the need arises. But it does happen that
someone comes along on his own initiative, asks for an interview, describes his experience, his interests, presents his
CV, his portfolio, and sometimes he'll be hired - to the satisfaction of all concerned.''

``For example,'' interjects Ezlip sitting next to the CEO, ``Some years ago a
young man came by, said that he dabbled in photography, brought along his portfolio of photographs - and the rest is
history.''

One of the soldiers raises his hand, ``Was it you?'' \

Ezlip says, ``Close, but not quite --''

The \ soldiers smile, some even laugh{.}

~Bonimi is embarrassed. ``From this you may assume that I was a failure as a photographer,''
he admits.

The soldiers applaud.~

The phone rings. Ezlip picks up the receiver, speaks softly for a minute, covers the mouthpiece and whispers to the CEO,
``It's Janha. The chairman of chamber of commerce is waiting outside.''

Bonimi whispers back, ``Just another few minutes.''

Ezlip relays the message to Janha in a lowered voice and replaces the receiver.

The CEO goes on, ``It seems that unfortunately I have to bring this pleasurable meeting to an end - they're
breathing down my neck. I hope you've been told that there are some complimentary tickets for you for
tomorrow's{ }ceremony inaugurating our renovated auditorium.''

Ezlip fills in, ``You can get them at the information desk in the lobby. The receptionist will be glad to
oblige.''

Bonimi takes over, ``Let me wish you an uneventful term of duty.''

The soldiers smile and get up. One by one they file by the CEO and his aide to~shake hands on the way out.

Bonimi takes pains{ }to read aloud{
}from their tags each soldier's name as he shakes hands with him - at the same time exchanging a word or two. Having
been{ }shown the list of these selected cadets in advance, at his request, he
knows that one of them should be his nephew. His heart races when the moment arrives. ``Selected Cadet Saffia, Glin -''
he reads out as he slowly, deliberately, shakes Yamik's large strong{ }hand,
``Any plans for after your service?''

Yamik answers, ``I'd like to go into audio-engineering, sir.''

``You might even join us here,'' Bonimi smiles.

``Thank you, sir,'' says Yamik and continues on his way to the door.

Bonimi{ }stops{ }him with a gentle tap on the arm, ``We share a surname
--''

Yamik looks~the CEO straight in the~eye. ``A common surname harbors both threat and hope,'' he
responds lightly.

Bonimi studies the young soldier's face intently, searching for some resemblance to Saffia, to Rimat, to branches of
their families. He forces himself to find some. He says, ``I'm sure your family{ }are proud
of you --'' hoping that Glin will come out with some details~about Rimat - maybe she remarried, has more
children. He's intrigued by the fact that Glin isn't curious about their common surname.

``They are,'' Yamik says. ``Thank you, sir. Our tour of the TV building was very
interesting. And we all really appreciate the time you took to meet with us. Good afternoon.''

``All the best to you,'' Bonimi salutes him. The tinge of a Balad accent in the young man's
speech sets him wondering. He won't allude to it of course.

Yamik returns his salute, turns and leaves, closing the door behind him.

The silence of the room clashes head-on with~Bonimi's inner tempest.

``You outdid yourself,'' Ezlip compliments the CEO. ``I don't remember ever
having such a successful meeting with any group of selected cadets. You were so personable, so open, so
spontaneous.'' The CEO spreads his arms dismissively. Ezlip slips several documents into his
briefcase.\textbf{ }

The CEO dares to guess, ``Source material for a book about our chapter in history?''

``On the button,'' Ezlip replies, shutting his briefcase with a click.

``How about calling it The Last Renovation?'' suggests the CEO.

``Apt title,'' says Ezlip, adding, ``I can barely wait for
tomorrow!''

``At long last -'' Bonimi sighs with relief. ``I'll be happy when it's over and
we go back to our usual routine, bless it.''


\bigskip

\chapter{}

The ~elderly guard points to the~small TV screen on Shisha's information desk, ``The auditorium is starting
to fill up{\dots} Oh, for God's sake!'' He's spotted a young woman wheeling a~baby's pram through the
entrance door into~the lobby. He rushes{ }over to her, touches her arm.
``Sorry, ma'am, but babies aren't allowed in there,'' he says politely.

Aera is flushed and sweating. ``You must help me,'' she almost orders the old guard.
``You must \textit{urgently} find the baby's father who's inside the auditorium and tell him to come out
\textit{immediately}!''

The old guard's heart softens instantly, ``Have you any idea where he's sitting?''

``Row 7 near the aisle. I don't know which side. He's very thin and goes around with a Walkman and
earphones.''

``Have a seat meantime, ma'am,'' \ the old man points to a bench near the bulletin board and runs to fetch
the distraught young woman a cup of water. In their short exchange he couldn't help noticing her slight Balad accent.
Brushing suspicion aside he hastens to the auditorium to look for the baby's father.

~Shisha's eyes dart for a second towards the woman with{ }the pram and then
return to the TV screen.

Hardly a minute passes before Glin rushes out of the hall, visibly frantic. What happened? Who could be asking for him?
He swiftly takes in the lobby. Aera! He rushes over to her. ``Aera!''

``You \textit{have} to get out!'' Aera orders him in a whisper.
``Run!''

Glin whispers back, alarmed{, }{}``What are you doing here?!''

``They're going to kill you!''

Meanwhile the old guard has returned to Shisha's desk. ``Shhh{\dots}'' she motions to Glin and
Aera. She and the guard exchange {uneasy looks }at the~family~tiff \ they are
privy to at the far end of the lobby.\textbf{ }

``You\textit{ }must\textit{ not} go to the cave!'' Aera whispers to Glin.
``They're setting an ambush for you there! They're going to kill you!''

Glin is confused. ``Who?'' \ ~

``Our men!''

Glin doesn't, can't, believe her. \ ``What does Yamik say?''

``It's out of his hands now! Get out! Run! ''

Glin recoils, ``No way!'' How can he not carry out~all Yamik's precise instructions?

Aera points at the baby in the carriage, ``This is our baby! Run! For our son's sake!``~

``\textit{Our }baby?'' Glin is stunned.

``I{\dots}I didn't tell you. I couldn't - ``she stutters.

He now recognizes his old cradle in this renovated baby carriage. He peers into it,{ }sees a small head
with baby hair, a little{ }body swathed in a pale blue coverlet. By sheer force
he detaches himself. ``Aera,'' he says, a tremor in his voice, ``I'm part of
Yamik's plan. I cannot leave.''

``I'm going straight over to Yamik right now. He'll order you to get out!''

Again Shisha at the information desk tries to get the couple to keep their voices down.

``You're imagining things,'' Glin loses patience with Aera. ``You are in the
way!''

Outside, running for all her worth, Aera suddenly remembers that faded photo of the strange man she had slipped into the
pouch on the carriage to show to Glin later. In her panic she had forgotten to show it to him. Next time then. Let it
be sooner than soon.

Glin hurries back into the auditorium.

``Look, look!'' Shisha tugs at the guard's sleeve,'' that woman ran outside and
left her baby behind!''

The old guard is about to go after Aera, but stops short when he sees the MC of the ceremony emerging from the
auditorium followed by the CEO.

``I can't, Boni, I can't -'' Alir shakes his head, ``Everything's gone blank. I just can't
-``~

``But it's all going so well,'' Bonimi encourages this {\ }overgrown boy as he
takes hold of his hand to pull him back to the auditorium\textbf{. }

``No, no,'' Alir moans,~not budging.

Shisha fills a glass of water from the decanter on her desk and hands it to the CEO. The CEO makes a mental note to
insert a few words about her solicitude in her employee file in the manpower department.

``It's over-excitement,'' he explains to her, handing the glass to Alir who absently downs the
water in one gulp. It dribbles from his mouth onto his elegant jacket hired for the occasion. With his ever-ready neat
handkerchief \ Bonimi pats him dry, and with the merest nod indicates to Shisha to lower the volume of the TV.

Oblivious of all Alir~shouts, ``Rimat is in the auditorium --''

``Rimat!?'' Bonimi is shocked.

``She's waiting for me to correct the third line,'' Alir explains as if it's self-evident,
``and I'm completely mixed up... I forgot what the correction was -''

``Look!'' Shisha points to the screen.

Bonimi turns his eyes to the screen and asks Alir, ``Who's that young man?''

Alir glances mechanically at the screen, ``No idea -''

The~old guard brings Alir a chair to sit on.

``What a performer!'' Bonimi beams with delight. ``The way he so naturally leapt
from his seat in the auditorium and ran to the podium to grab the mike!'' He puts his arm round Alir's
shoulder, ``Let's just wait until this part's over, then we'll go back inside.''

Alir vigorously shakes his head in protest.

Bonimi's eyes are glued to the screen, ``He's reading something out --'' he turns to Shisha,
``could you turn up the volume ?''

On the screen Glin is seen reading aloud from a sheet of paper:

``{\dots}And this is why we want to explain our demand to everyone, to the audience~in this auditorium,
to~TV viewers, to radio listeners. We demand to be able to return to the place where our fathers and forefathers lived
for generations - to our Pale Blue Valley! We will rebuild our homes there and no longer be{\dots}''

The CEO pales, he orders Shisha almost shouting, ``The intercom to Chief of Transmission!''
Once the phone is in his hand he steadies his voice:

``This is Bonimi Saffia, the CEO of Public Television Network speaking,'' he identifies
himself then raps out his orders: ``Stop external transmission immediately. Announce a technical hitch.
Show clips from last year's song festival. Continue transmission only in closed circuit, to me down here at the
information desk and to Security. These are Balads. They've hijacked~the ceremony.``~~~~~~

The screen that went dark for a minute now lights up. Glin is seen again, reading from the sheet of paper,
``I repeat the warning again: the TV building is booby-trapped with multiple explosives. If we do not
receive, \textit{within half-an-hour,} a clearly stated~assurance from an authoritative source that our demand will be
met, \textit{these explosives will be detonated}. If anyone tries to lay a hand on me, detain me, harm me in any way,
\textit{the explosives will be detonated}. Once we get this assurance, I shall leave the auditorium. Unless I am given
complete free exit, unaccompanied and without{ }being followed, \textit{the
explosives will be detonated}. The half- hour countdown begins{\dots} Now!''

Shisha, the security guard and the CEO stare at each other{ }blankly. In a trice
their shock changes into a realization of the situation. The horrific experience of others, what they've read about in
the papers and heard on the radio and watched on TV is happening here and now. Shisha bites her lower lip so hard that
it starts to bleed, spots of blood appear{ }on her blouse. The old guard can
barely breathe. Bonimi clenches his fists. Alir alone is locked in his own world, oblivious of all around.

The auditorium door is flung open, Ezlip rushes over to the CEO.

Glin's voice is heard on TV, ``The explosives will be set off if any other person leaves the auditorium. I
demand complete silence in the auditorium.''

The CEO updates Ezlip about~the instructions he gave to chief of transmission and adds, ``I'm sure Security
is already in control.'' He turns to Shisha, ``The mike to the auditorium, please
-''

Shisha's hands tremble as she does his bidding.

The CEO speaks into the mike/{The CEO's voice is heard over the system},
``This is Bonimi Saffia, Public Television Network CEO. All of you here, everyone, please stay calm. The
situation is under control. Official negotiations are about to{ }begin. I'd
like to request that the half hour{ }countdown{ }begins now. Is that
confirmed?''

Silence.

The CEO repeats his request, ``I would like~confirmation of my request that the half hour starts
\textit{now}.''

After another moment of silence the response comes over loud and clear. ``Affirmative.'' The
voice now is Yamik's.

The CEO says ``Thank you.'' He shuts off the mike. ``This is a new voice,'' he
tells Ezlip, ``This guy is{ }obviously the one in charge. Security must
be{ }looking for him.'' Bonimi speaks again into the{ }mike.
``I request that elderly people, women and children be allowed to leave the auditorium -''

Yamik's response is immediate, ``Request denied.''

The CEO again muffles the mike and draws Ezlip's attention to the young Balad standing on the podium. ``Looks like he's
also waiting for instructions.'' He again speaks over the audio system, ``I request that
children and elderly people be allowed to leave the auditorium -''

A second passes before Yamik's voice is heard, ``Request denied.''

The CEO says, ``I request that children be allowed to leave the auditorium -''

Two seconds pass before Yamik's voice is heard again, ``Request denied.''

The CEO says, ``I request that the half hour count begins now -''

Yamik's voice is heard, ``Request accepted.''

The CEO shuts off the system, nods his head in the direction of the TV screen then turns
to{ }Ezlip, ``For the life of me I'd never guess that that young
fellow there on the stage is a Balad. He looks like one of us. Speaks like one of us too.''

At the same time he's wondering why that new voice coming over,{ }with its
mixture of accents, sounds so familiar. Whose{ }is it?

The phone on the information desk rings. \ Ezlip picks up the receiver and hands it to the CEO, ``The
Chief-of-Staff --''


\bigskip

\chapter{}

Hidden in the shadows behind the television building Yamik suddenly sees his wife racing towards him.

{}``Aera? What in God's name are you doing here?! You must get out!'' he confronts her.

``The men{\dots} are going{\dots} to kill Glin!'' she pants.

{{\dots}}``What?{ }Are you
crazy..?'' -- is all he can say. \ How did she know?

Then, in a flash, the pieces fit together: the men had visited the shack and talked. She'd learned they were party to
Yamik's planned operation in the TV building and knew of Glin's involvement.

``Yamik, who will get to that cave first?''

``Glin.''

``So Glin will be at their mercy!'' Aera drops all pretense.

Yamik hears her desperate words but they don't register. All he wants is that this hour, this day should be over, and
that he and she and the baby will be together at last. But Glin's leanness, vulnerability and trust, pain him to the
core{.}

\ ``No,'' he says{ }confidently,
``no,{ }it won't happen, but{\dots}''

``But what?'' Aera hisses.

``It's too complicated,'' he says, ``no time now to explain. Just get away!
Now!''

``Did you give them the green light?'' Aera insists. ``Yes or no?''

Yamik repeats once again, ``It's too complicated.'' He
cannot{ }share with Aera what happened between himself and the men, can't tell
her they had decided to liquidate Glin after the operation. That was how he'd come up with the idea of the cave and
convinced them of its practicality. At the time he'd reckoned on reaching the cave before they did and one way or
another protect Glin. But how? For how long? He'd had to~put~these questions aside for the moment.

``You promised me,'' Aera fixes her eyes on his and they're both aware that she's alluding to
the promise he gave her after she'd had a hand in the death of the lame gatekeeper at the village .

``You must get away from here!'' Yamik pleads, ``At
once!``~~~~~~~~~~~

``Not before you promise me that you won't let them touch Glin!''

``Alright, alright! Just get away, Aera{. T}he building is
booby-trapped -''

``So you \textit{did}{ }plant explosives?''

``{I}t was{ }necessary. You
mustn't be in the way. Just go!''

~``Baby's in there!''

``\textit{Inside the~ building!}?''

``Yes!''

``You brought him? Are you out of your mind?!''

``I couldn't leave him on his own {{}-} mother was at
work.''

``Get him out of there! Now! Go!''

``I'm staying with you. And when I see Glin leave the building, I'll go with him to the
cave.''

~

\chapter{}

Ezlip hands the receiver to the CEO ``The Prime Minister's security aide.''

``Bonimi Saffia speaking,'' he identifies himself, listens then responds, ``Very
well, sir -'' he returns the receiver to Ezlip and updates him, ``Special Squad is moving in.
They've intercepted radio contact between that Balad in the auditorium and somebody outside.''

Yamik's voice is heard, ``There's a baby in a carriage in the lobby of the building. It should \ be wheeled
out.''

The CEO and Ezlip turn to look at the elderly security{ }guard who's pointing at
the baby carriage. The CEO's voice comes over: ``What do we get in return?''

``First, the carriage must go out.''

Ezlip gestures to the CEO to respond in the negative. Bonimi ignores him and asks over the system, ``Does
this time span extend beyond the limit?''

``It does,'' answers Yamik{.}

The CEO covers the mouthpiece and asks, ``Who's baby is it?''

The security guard is all too ready to respond, ``A young{ }woman
wheeled it in and insisted I get its father from the auditorium right away -''

Yamik's voice is heard, ``The baby should be taken as far as the new supermarket at the corner of this
street.''

The CEO gestures to Shisha~to wheel the baby carriage out. She rubs her fingers nervously as she leaves the information
desk. First time she's ever touched a~baby carriage. Security opens the door for her.

The CEO turns to Ezlip, ``No point haggling over terms of exchange, Special Squad has it in
hand.'' Over the speakers he says, ``We expect the countdown to be adjusted accordingly
-''

``It will be,'' Yamik's voice is heard.'' It will begin the moment the baby
carriage reaches{ }the new supermarket.''

Bonimi again asks himself why the voice coming over - witi its mixture of accents - sounds
so{ }familiar. All eyes are fixed on the
screen{/ }Bonimi, Ezlip and the security guard watch the skinny Balad standing
motionless on the podium, his fingers nervously tapping his microphone.

Then Yamik's voice is heard, ``The half-hour countdown
begins{\dots}{ }NOW!''

The phone buzzes. Ezlip picks up the receiver, listens briefly and puts it down again. ``The Chief Of Staff
wants you to know that he's much impressed with your presence of mind,'' he tells the CEO.

``A vital piece of information,'' Bonimi responds wryly.

``Look!'' The security guard cries out. Alir stirs from his torpor, gets up from his chair and
joins him and the CEO and Ezlip who are staring intently at the TV screen.

``Who's that woman walking over to the stage?'' Ezlip asks.

``That's Rimat!'' Alir gasps.

Bonimi stares open-mouthed at the woman making her way towards the stage, ``Our
Rimat{\dots}?''

~Glin's voice is heard over the system, ``Mom, what the hell are you doing here?'' Then
he{{}'}s seen stepping down from the podium and tripping over. Strange crackles
are emitted from the TV set.

Ezlip understands. ``The bloody fool got entangled in wires! It's going to {\dots}''

``Glin!'' Yamik's voice booms{ }out,
``Glin!!!''

Yamik's cry is lost in the resounding blast that rocks the building.


\bigskip

\chapter{}

Mother is weeding her garden. Footsteps are heard coming up the hill, getting closer. She hopes against hope that it's
Gidal, and{ }no one else. It is! She straightens up and waves.

He waves back. ``It's him!'' he calls.

Mother's hands tremble. She waits until Gidal~closes the distance~between them and only then
asks,{ }``Are you sure?''

``Yes!'' Gidal almost dances for joy, ``Absolutely sure!''

Mother doesn't trust this man who is guided by the goodness of his heart, ``How can you be so sure?''

``His pram stood outside and I recognized it immediately -'' Gidal can hardly contain himself.
``Because of that pouch your daughter fixed to the cradle and the frame with wheels I mounted it
on.''

~Needless to say he doesn't breathe a word about having recognized the cradle the moment he saw it in the shack. Babies'
cots do look alike, but there was no doubt about this one - it was baby Glin's old cradle. Hadn't he at the time
replaced the worn handles with those new ones? How happy Rimat and Saffia had been with his handiwork -- they couldn't
thank him enough!{ }And it was in that very cradle that Rimat had
brought{ }Glin to the village that night when she came looking for him. And
later on, when Glin outgrew it, she'd turned it into her sewing basket. Hadn't he seen it whenever he stopped
by{ }at{ }her place? Aera said she'd picked it up at the flea market. But he has another
idea about how it found its way into her hands. On that too, of course,~he keeps silent. ~~

``And{\dots} \textit{the little one}?'' Mother tries to~steady her voice.'' Did
you see him?''

``Only from a distance.''

``How does he look?''

``He's filled out and he's smiling.''

``\textit{Filled out and smiling},'' Mother echoes as she vacillates between belief and
disbelief. ``How did you get in?''

``The way you told me,'' Gidal says, and carries on, ``I said that I want to
speak with the matron of the nursery. They told me to wait in the corridor outside her office. So I waited. Then they
opened the door and invited me in. So I went in. There was a woman there and I figured she must be the matron. I said
what you told me to say: that sometime soon my daughter is due to give birth as a single mother, and that she would
like to give up her baby for adoption, which is why we're looking for a~nursery that will take good care of the little
one and find him a good home. I told her what you told me to say. So~the matron asked a nurse to show me around.~ And
then, in another corridor, I saw the~pram standing near an open door. I asked if I may go into that room and the nurse
said yes. There were four babies there and a couple of nurses - one of the nurses was holding\textit{ our }baby in her
arms. I was sure it was him because I recognized the coverlet he was wrapped in - it had a pale blue floral pattern. I
looked at him and could easily see that he'd filled out and was smiling. ''

Mother, still hovering between belief and doubt, again{ }echoes ``\textit{Filled
out and smiling}.''

``Yes,'' Gidal beams, delighted to be the bearer of such good tidings.

``Anyone followed you?''

``No one -''

Suspicious{ }Mother doesn't trust him. How can he be so sure? Maybe people on
their side associate Baby with what happened. Maybe they're waiting for someone to ask about him, show an interest. She
stops right there \ {}- she mustn't share her anxieties with this kind-hearted man.

``Did you ask the matron how long it usually takes them to find good adoptive homes for these
babies?'' she asks.

``I asked her. She said that it usually takes about a year.''

``A year!'' Mother cries in despair.

``But in some cases it happens earlier,'' Gidal is happy to encourage her.

``There's hope.''

``There is.''

``And~you'll keep your eyes and ears open. All the time.''

``I will.''

``And the young woman{\dots} the one who saved him?'' Mother wants to know, ``How
is she doing?''

``She's still in hospital.''

Mother is fearful, ``Does she claim{ }that the baby is hers?''

``No. In the papers they said that she's single, she never married, never had children.''

Plucking up courage Mother asks, dreading the answer, ``Has she got her memory back?''

``Not yet.''

``And Baby doesn't yet have a memory.'' Suddenly a new fear pierces Mother's heart,
``Has anybody come to claim~Baby?''

``No.''

``How do you know?''

``How do I know{\dots}?'' Gidal stammers, ``I assume that it would be known
--''

``By what means?'' asks Mother.

``They would talk about it on the radio, on TV, write about it in the papers -'' Gidal is
sure.

``You're right,'' Mother agrees. ``I can't begin to tell you how grateful I am
for what you're doing. For keeping quiet about it. You do keep quiet, don't you?''

``I keep quiet.''

Mother fills the watering can. ~Gidal takes it from her and waters the flowers. Since the disaster he's been coming here
almost daily. By now he knows how much water each plant~needs.

But Mother can't altogether rely on him. ``Add some more here,'' she helps him along.
``That's quite~enough over there.''

``The flowers are doing nicely,'' Gidal says cheerily.

``Aera~ always wanted a flowerbed here. You told me they buried her in the mass grave~for the unidentified.
But this is her real grave.'' She~doesn't tell Gidal that Aera always wanted to have pale blue flowers
grown here but so far hadn't found the right seedlings - it's too intimate. ``Back in a
minute,'' she excuses herself and disappears indoors.

Done with the watering Gidal washes his hands at the outside tap, seats himself on the bench and leans back
comfortably~against the wall. Mother reappears carrying a tray of food which she places on a low table in front of him.
She sits down next to him. They're used to having an evening meal together.

``Everything tastes so good,'' Gidal says. ``Delicious as always. Thank
you.''

Mother asks, ``Do they ask about Aera in the village?''

``No,'' says Gidal. ``They believe she's here with you and Baby -''

``Life is weighing me down{ }like a millstone,''
Mother sighs. ``A few of our men came by yesterday asking about her. I said she'd gone to the market. They
told me they had come to see her the day before the TV building explosion. I've been thinking: Maybe she went to town
that day because of something they had told her? ~One thing I'm sure of - she could have asked me to stay with Baby,
but she didn't because she didn't want me to know she was planning to go into town. She had her reasons and I'll never
know them.'' After a moment's silence she says, ``The boss at the packing house told me that the woman who
lived in the village with her only son -- those two who left -- that they~too were among those killed in the explosion
-- the mother inside the building and the son outside.''

Gidal is silent. What could he tell her about his own grief? ~

Mother continues, ``The boss told me that you loved that woman. For years and years. Forgive me for
speaking like this. I just want you to know that I know.''

Gidal holds his peace. He has a good feeling about Mother knowing of his love for Rimat.

``I remember her son,'' Mother continues. ``He~wasn't like the other youngsters
in your village. He was skinny, always with his{ music}. He would come to the
packing shed and make Aera laugh. I was cross with Aera about this. Now I'm happy about it. At least she laughed a
little before she... ''

Gidal yields to temptation, ``I once saw them walking together -'' he says quietly.

``You never did!'' Mother calls him to order.

Gidal continues, ``- outside the fence -''

Mother hisses emphatically through clenched teeth, ``\textit{No, you did not!}''

``But I did,'' Gidal insists obtusely.

Mother stops him adamantly, ``\textit{You never saw them. Never.}''

Gidal finally catches on. ``No,'' he shakes his head, ``No, I never saw them. Never.''

``Aera was a strong young woman. Aera had a good head on her shoulders. Aera knew what's right and what's
wrong.'' Mother, sensing she may have touched a tender spot, softens her voice, ``Have some
more food -''

``Thank you, but I had enough.''

``I don't prepare all that food for myself only. It's for you too. For the two of us.''

Gidal gives in and helps himself to another wedge of cheese. Suddenly he can no longer contain himself.
\ ``Where is Baby's father?'' he blurts out.

``He had to go away some place for some time,'' answers Mother. ``This happens to
our men not infrequently. Every day that he doesn't return spares him the sorrow about Aera. Every day he doesn't
return deprives him of the happiness of Baby's survival. ''

Meantime the sun has sunk below the horizon.

Gidal feels good about the darkness enveloping the two of them. ``I had one night of happiness in my
life,'' he says. ``That woman that I loved once spent a night in my room and she nursed her
baby. \ And the{ }one \textit{day }of happiness in my life was the day I came
here and Aera did the same, with her baby.''

Mother says, ``I had many days of happiness and many nights of happiness but they're all gone for me. Like
they're gone for you. The main part of our lives is gone forever. Now we're just living out our lives. My debt to you
knows no bounds. You found Baby for me.''

``It's you who found him,'' Gidal dares to correct her. ``You told me to read all
the newspapers carefully, watch television day in day out. Listen to the radio{ }day in day
out.''

~

\chapter{}

The hairdresser fans Shisha's hair dry then puts the finishing touches to her hairdo. ``You like it?''

``Thank you so much,'' Shisha responds mechanically, her eyes as dull as ever~when she looks
at herself in the mirror.

The hairdresser's heart goes out to her. She prays for her recovery. Looking at the mass of vases overflowing with
flowers that fill the room she says, ``You're the loveliest flower here. And look at all these
gifts!'' \ She points to the pile of packages in the corner. Shisha looks away, unable to bring herself to
undo the satin ribbons and open the glossy wrappings. She knows she should do it, find out who sent them, respond
with~appropriate thank you notes. The same goes for the bouquets of flowers. But she's not up to it. Not up to
anything.

The hairdresser lets a lock of hair fall delicately on Shisha's forehead. The young woman is impassive. The hairdresser
moves the wave a bit to the side to conceal the temple. Shisha does not react.{
}\

``In the papers it says you should be awarded the medal of courage -'' the hairdresser
chatters on.

``I wish they'd just let me be,'' says Shisha{.}

The hairdo is duly sprayed. Shisha doesn't protect{ }her eyes
\ {{}- }\ the hairdresser does that for her then removes the smock from
Shisha's shoulders and touches up the fresh hairdo with her fingertips. ``You are a beauty,''
she surveys her fondly. Shisha is silent. ``Everyone wants to know,'' she continues, ``when you'll~go to
see the baby.''

Shisha gets up from the chair and hobbles back to bed.

``Let me help you,'' the hairdresser offers.

``Thank you,'' Shisha answers. ``I've learned to get into bed by
myself.'' She sinks back into the pillows and covers herself with a light blanket. ``Is he
still in the nursery?'' \

``He still is. Do you think you'll go and visit him?''

``If the doctor agrees. I'll get him the biggest teddy bear ever.'' The hairdresser now has an
answer for the press photographer who lives on the floor above her. Since learning of her weekly visits to the hospital
to give a hairdo plus~manicure~to one of the only two survivors of the explosion that wrecked the TV building killing
over a hundred civilians and soldiers, he keeps badgering her with his request to discover if and when she goes to
visit the baby~she saved -- the other survivor. Always on the lookout for a scoop he wants to be the first to
immortalize that dramatic reunion. In return he has promised her to photograph her daughter's wedding gratis.~

``What about the~manicure?'' the hairdresser asks Shisha. ``Or have you had
enough for one day?''

``How would I~know?'' Shisha says. ``I only know that I simply forgot all about
the manicure.''

``I'll get organized and meanwhile you rest.''

Shisha shuts her eyes. What was it that made her grab the handle of that baby carriage? Why did she run with it to the
supermarket? These two questions give her no peace.

Meantime the hairdresser has packed the hairdressing kit~in one bag and taken out the manicure tools from another. She
lays them out on a small folding table provided by the hospital whenever she comes to Shisha, and places two
straight-back chairs on either side. She surveys Shisha lying there in bed, eyes closed but not asleep.
``Are you up to it now?'' she asks softly.

``It's not a question of being or not being up to it,'' Shisha says, painfully{
}moving herself off the bed to sit opposite the hairdresser. Robot-like she extends her hands. The hairdresser,
beginning~to file her nails, hopes that Shisha will initiate a conversation. This, as usual, doesn't happen.
``I read in the papers that the Ministry of Defense has done your place up,'' she eventually
remarks.

``I can't read,'' says Shisha. ``That too is gone, along with my
memory.''

``It'll come back,'' the hairdresser assures her. ``Who didn't come to visit you!
They had to put a traffic cop along the corridor. Toddlers who were at kindergarten with you, classmates from school,
whoever knew you from work or wherever. Not to speak of VIP's, big brass, celebs. What color nail-polish would you
like? ''

``I'd{ }prefer none,'' Shisha says, ``but
for the psychiatrist's sake you can use red. She keeps telling me that the best thing for me is to return to my normal
life as soon as possible... as if I know what that was -'' ~

The hairdresser notices that Shisha's right wrist is still swollen. Word is that -- luckily for her - the air blast from
the explosion threw her onto the baby's carriage. The metal rod jutting out of the new supermarket wall scraped her
right temple only superficially. Another millimeter and she'd have been killed. Off the record she's heard that the
doctors feel it's too early to tell the patient about this.

There's a light tap on the door. Doctor Nam enters, ``Good morning, ladies -'' he greets the
two with a pleasant smile.

The hairdresser and Shisha respond in unison, ``Good morning, Doctor Nam.''

``Am I intruding?''

``You never are, doctor,'' the hairdresser releases Shisha's hand.

``Please carry on,'' Doctor Nam tells her, ``I only dropped in for a moment.
Shisha, may I congratulate you on your lovely new~hairstyle.'' He lifts the tendril of hair covering her
right temple. ``Excellent. The scar is healing beautifully. ~In a few months there won't be a trace of
it.''

Shisha says, ``One thing that I remember is that my blouse was bloodstained \textit{before} I got that
bruise on my head and that horrible whack on my wrist.'' She knows she's said this a thousand times but
she still can't help repeating it. ``I just don't remember when or how.''

Doctor Nam too repeats what he's already said so often, ``All the doctors agree with you that there's no
connection between the bloodstains~on your blouse and your fall.'' Leaning on the bedrail at the foot of
Shisha's bed he reads the medical report clipped on to it. When he's finished he says, ``Well, today
you're leaving us, our brave girl -''

``I'm not brave.''

``We'll miss you -''

``What will I do at home?''

``You don't have to make any plans right now. Let time take care of things.''

The hairdresser-manicurist blows lightly over Shisha's nails to dry the polish. ``Finito,''
she says and packs her tools. ~``Bye-bye, dear, and be well -'' ~

``Thank you,'' Shisha says,~``you've been so kind. Please do me a favor and help
yourself to some of the flowers.''

``That'll be a real good deed,'' Doctor Nam adds. ``This room is like a flower
shop and so is the corridor outside. The whole hospital is enjoying the spillover.''

``Thank you,'' says the hairdresser. ``The bouquets are so nicely arranged here.
I'll take one from outside, in the corridor. Be well, my dear -''

~``Why not~keep making our Shisha more beautiful than she is already by nature, when she's at
home?'' the doctor suggests.

``Will that be possible?'' asks the hairdresser.

``I'll see to it,'' says Doctor Nam making a mental note to ask his secretary to call the
Ministry of Defense welfare department. They'll be more than happy to cover any expense for the young woman who put her
life on the line for an anonymous baby.

The hairdresser kisses Shisha affectionately on both cheeks before leaving.

~Shisha wants to return to her bed, but Doctor Nam lays his hands lightly on her shoulders. ``Why not stay
sitting on the chair a little longer, Shisha. Lying in bed all the time isn't good for you.'' \ Pushing
the small folding table aside he sits down on the hairdresser's chair. ``All your papers are ready. In an
hour{~}matron will come to accompany you outside. Somebody from the Ministry of Defense rehabilitation
department will drive you home. She'll go inside with you to check that all is in order there and make arrangements for
anything that you still feel necessary. You really do look very good indeed. And please come for a checkup next month.
But should any question arise earlier, any problem - even the smallest - don't hesitate to come or~phone. You now take
half a pill at night, right?'' He~takes another glance at Shisha's medical report. ``Let's
say you continue like that for another week and then cut down to a quarter{,}
for a week. I think that by that time you'll do fine without it. I think I've covered the ground, Shisha, but maybe you
have a question? A few questions?''

``No, Doctor Nam,'' Shisha says. ``I mean, yes.''

Doctor Nam smiles, ``What's on your mind?''

``Those nightmares. Are they going to stay? Will they ever go away?''

Doctor Nam says, ``The best thing -- as the psychiatrist suggested -- is to wake up completely. Get out of
bed. Take a sip of something,{ }whatever you like. Watch some television. In
time they too will go away.''

``Doctor Nam, I feel the handle of the baby carriage in my hands. But why on earth am I holding
it?''

Up to now the doctor would respond to this repeated question with the words, 'The day will come when you'll know.'
~Feeling that even though Shisha is of not above{{}-}average intelligence she's
still able to grasp complex phenomena if explained to her in plain clear language, to-day he says something else,
``Shisha dear, throughout your hospitalization the police and the army wanted you to be treated by
hypnosis so that you'd recall what happened. But the psychiatrist objected all along. And I've sided with her. We felt
that it was like an unnecessary operation. And since you were our patient, we had the say. But now, my dear, once
you're home, if they repeat this suggestion it will be up to you to decide.''

``So what should I tell them?''

``Both the psychiatrist and I still believe that the best thing is to let nature take its course. Not to
force memory. That it's better to let it renew itself in its own time, at its own pace. How are your nails doing? Has
the~varnish dried yet?''

``Almost, Doctor Nam.''

``Let me not spoil it.'' Cautiously he pats the back of Shisha's hand. ``You're a
brave young woman.''

``I wish they'd stop saying that. I don't remember what happened. But I know that I couldn't have been
brave. I've always been such a big coward.''

``So, see you in a month's time, my big coward young woman.''

On his way to the door Doctor Nam thinks about the young man who has phoned him several times to ask whether it would be
alright if he visited Shisha since he was somewhat acquainted with her -- he'd been a tour-guide in the TV building.
Though Doctor Nam was favorably impressed by the young man he didn't think Shisha was ready yet for this kind of visit.
When he calls again, Doctor Nam now decides, he'll tell him that Shisha is now out of hospital and back home. Yes, let
whatever happens between the two of them happen on Shisha's turf. Doctor Nam closes the door behind him with a good
feeling.


\bigskip

\chapter{}

Zakod pulls out from under the stairs the two suitcases that Rimat and Glin brought with them when they first arrived.
He takes Glin's to the boy's room and opens it on the bed. He collects Glin's clothes and things from around the room
and packs them in the suitcase taking care not to touch the dog-tag and the three~photos sent by the army. Dela
had~asked permission to keep them. Otherwise? He didn't know then what he'd do with them. Again he scrutinizes those
three photos. The same group of soldiers are photographed from three different angles - but none of their faces match
Glin's. Could the crew cut and the uniform have changed the boy so much? \ To his way of thinking, as he told Dela,
Glin wasn't on the photos because he was probably the photographer. She doesn't believe it. She's sure Glin is not on
the photos because it was not his style to be photographed in a group. One thing they do agree on, though, is that
Glin's musical gear was not returned because somebody at his base had taken a fancy to it.

\ Leaving Glin's room, he picks up Rimat's suitcase and places it on her bed in the adjoining one. Taking care of
Rimat's clothes and belongings is far more difficult. He braces himself, functions like a robot. Handling her
embroidery he looks for the needle and finds it pinned into an almost completed flower - it's like~touching Rimat's
hand. He's on the verge of tears{.} \ If Dela could embroider, he believes, she
might have wished to continue where Rimat left off. After a moment's contemplation he places the skein of embroidery
threads together with the thimble and scissors in the middle of the fabric, rolls it up and lays it in the suitcase. He
then takes the two packed suitcases and puts them back under the stairs. The lady who manages ``Hand to
Hand``~has promised him on the phone to send a volunteer to pick them up as soon as possible.

\ Returning to Rimat's room he takes a long look at Saffia's violin case on top of the book shelves. Not for
all{ }the world will he ever let it out of his hands. He reaches up and pats
the case. At his touch the cover sheds a shower of brown flakes - as though in live response.

He turns round, sits himself at the table, places a sheet of paper in front of him and writes:

~

Dear Michlor,

Thank you for your letter.

My wife and I are by no means well-to-do. Still, our economic situation enables us to raise the baby we have adopted.
This is the reason why we are returning to you the generous check you sent. Please accept it as our contribution to the
International Child Welfare Fund that you established.

As for your questions, I cannot begin to tell you about ourselves in detail. If you come for a visit we will have a lot
of ground to cover.

Let me explain how the adoption of the baby came about. After it was ascertained that according to all external signs
and marks he might be~of~Balad origin, no couple who qualified as potential adopting parents were willing to adopt him.
You might be able to surmise the reasons for it. This is why the authorities agreed to give him for adoption to us - a
couple evidently past the prime of life, the adopting mother also being an invalid.

This baby is balm to our bereaved life.

We received your letter of condolence after our tragedy. My wife and I deeply appreciated your words. My wife did not
respond to you because she was so badly injured. I did not respond to you because I could not muster the mental and
emotional efforts it required.

Your letter reminded me of our days of old which hold a deep meaning for me.

I wish you success in your life's work. You are very fortunate. You are one of the few in the world whose profession is
also their hobby.

Again, I wish to express to you the joint gratitude of my wife and myself.

Yours, in friendship,

~

Zakod goes over the letter, deleting a comma here, adding a full stop there, then signs his name, folds the letter,
inserts it in an envelope, affixes a stamp at the top right-hand corner, writes on it the necessary details and puts it
aside.

That done, he takes another sheet of paper and begins to write:

~

My very dear and old friend Gidal,

What you wrote is correct. Each of us mourns Rimat and Glin in our own way.

I will now address myself to your question. ~Since I did not watch that ceremony on television, I did not see the young
Balad who mounted the podium and grabbed the microphone. Therefore I cannot say anything about his looks. Be that as it
may, I can imagine to myself that the people in your village tend to jump to the conclusion~that it was none other than
Glin because -- according to your own words -- they ``so much hated and detested Glin for his refusal to
be conscripted.'' And this despite the fact that they only saw him very briefly, since the broadcast (as I
understood from the papers) was almost immediately blacked out after his appearance. Added to this is~the fact that
Glin's name appeared in the list of casualties among the soldiers. This too might play tricks with the ability to think
rationally.

From past experience I learned that wholesale animosity can easily serve as fertile ground for turning mere conjecture
into hard and fast facts; and that people tend to transpose~facts from one realm to another when in need of a basis
for~their animosity.

Your letter reminded me of our days of old which mean a lot to me.

~

In friendship,

~

Baby's babblings begin to be heard from upstairs.~~~~~~

Zakod leaves the letter. Later he~will go over it and prepare it for mailing. He climbs upstairs.

~

\chapter{}

The heater chimes and Zakod lifts out baby's feeding bottle.

Dela -- who'd rather~risk being accused of over-nagging than jeopardize Baby's welfare ~- gently advises,
``Baby should have the vitamins before his bottle -''

Where Baby's care is concerned Zakod accepts her opinion without question. He takes the small bottle of vitamin drops
from the refrigerator, places it in her hand then bends over the carriage. ``Pardon me, young man, for
interrupting your eloquent speech,'' he picks up the infant, ``but here's where you need to
be.'' He deposits him in Dela's arms. Baby opens his fledgling-like mouth and Dela pops in the drops. Baby
smacks his lips. Zakod \ takes the vitamin dispenser from Dela's hand and replaces it with the feeding bottle. Baby's
lips tighten around the nipple. He sucks on it gustily, and she imagines milk streaming from her breast. If only she
could share this fantastic sensation with somebody - but with whom? Certainly not Zakod. Her inability to share this
kind of physical intimacy with him is an ache that never leaves her. But this is how things have always been between
the two of them. She doesn't like what Zakod found to cover Baby - an old T-shirt of his - to replace the lovely pale
blue floral coverlet he was wrapped in when brought from the nursery - \ the~same material as the mattress cover. But
these two items have to be laundered from time to time. It's three days they've been hanging outside on the washing
line. They surely must have dried long since. Yet she doesn't remind Zakod to take them down. As it is, he's
overburdened beyond measure. Taking care of Baby is in addition to all the other household and farm chores he
shoulders. To say that none of them is natural for him~is a colossal understatement. Hopefully the delicate blue
flowers won't fade in the~sun. ``Baby has grown,'' she smiles at Zakod, ``he's coming along very
nicely.''

``Very nicely,'' Zakod agrees, allowing \ himself to be carried away with sheer joy.

The almost palpable sensation of having a suckling at her breast now assures Dela that Zakod doesn't regret the~
adoption she so craved for -- this in spite of the extra burden that it has placed on him. ``Did you
manage to do the packing?'' she asks,

``I did.''

``When are they coming to pick up the stuff?''

``The lady at their office said the moment they have someone to come and take it. I decided not to give
them the violin. Let's keep it.''

``Let's keep it -'' Dela repeats his words, her heart now at ease. Although Zakod spoke in the
first person plural, he was certainly the one to make this decision. She will never forget that Rimat's true love,
Glin's father, played that fiddle.

Both of them now look at Baby, each aware of what the other~is thinking: maybe he'll take up the violin when he grows
up. Eyes closed, at peace with the world, the little one has almost drained the bottle.

``Not just yet,'' Dela says softly as Zakod gets up from his chair to take Baby. Sitting down
again he gazes at this divine picture~of a sleepy baby~nestling in its mother's bosom. He doesn't remember any of the
three girls in this position. At that point in time he was waiting for them to grow older so that he could discuss with
them ideas and theories and historical events. Returning abruptly to the here and now, he says, ``I wrote
a letter to the man who sent us that check and explained why we are returning it, telling him this is our contribution
to the International Child Welfare Fund that he set up - the fund I told you about. ~I also wrote a letter to the man
in that village-- the one Rimat and Glin came from.''

Dela is happy to hear this. ``I wish our lovely~Shisha would understand that Baby has enough toys
already,'' she says, ``and \ that she can come to visit without bringing him any more.'' Both of them
glance at the collection of toys around them: the gigantic teddy-bear expropriating a substantial part of the couch,
the mobile swan hanging in the middle of the kitchen, the three~balls on the carpet, the row of raggedy Annes and
raggedy Bobs filling up the bookcase in the living room. ``I don't know how to bring up the subject with
her -''

``Neither do I,'' Zakod agrees with her.

``Without a memory she's not a real person -''

``The very thought that her memory might never return -- '' Zakod says realizing as he says it
that he cannot remember ever having had such a conversation with Dela.

``I did try to encourage her,'' Dela says, ``tell her that one day she'll have a
baby of her own. But who knows...~ I feel that she can't even stand somebody mentioning that the scar on her face has
almost disappeared.'' \ Baby's lips slip off the bottle with a sneeze. He half-opens one eye and yawns.

``Like a full-fledged grown up,'' Zakod smiles.

``How I wish we had some photos of Rimat and Glin,'' Dela sighs.

``I will add that to the letter that I wrote to that man in the village. If he has any, maybe he will not
mind sending us a few.''

``I hope so -'' says Dela, her eyes welling up. ~``You may take Baby now
-''

Zakod first takes the bottle from her and stands it on the kitchen counter then picks up Baby. ``I will do
him the honor of changing the diaper,'' he offers chivalrously.~

``Good idea,'' says Dela. ``Even if he wakes up, he'll go back to sleep with a
bit of rocking.''

Zakod lays sleepy Baby on the chest of drawers between the kitchen and the living room - a piece of furniture purchased
in preparation~for the adoption.

Now that he's standing with his back to her Dela ventures to ask, ``Was Rimat{\dots}
expecting{\dots}?''

``No,'' replies Zakod as he fastens a fresh diaper around the chubby little body.~ He is
visibly moved by Dela's courage to ask this question.

There's a knock at the door.

{}``That must be Hand to Hand.'' Zakod says. He settles Baby in the carriage and opens the door. No, he had
guessed wrong. A middle-aged{ }woman clad in black stands outside. ``Good morning,
madam,'' he says, wondering whether she might be a Balad.

``Good morning, sir,'' says Mother.

Her accent leaves no room for doubt.

``Can I help you?'' he asks.

``I'm looking for housework,'' Mother says. ``Any housework -''

``Please come in,'' Zakod shows her into the house.

She enters and follows him to the kitchen.

``Good morning, madam,'' she addresses Dela.

``Good morning to you,'' Dela greets her.

``I'm looking for housework, any house work,'' says Mother. \ Baby starts to whimper. She goes
over to the~carriage and peers inside. Applying all the strength she can muster she smothers her emotions.
``He might be hungry,'' she says.

``He just had his bottle,'' says Zakod,

``Perhaps he would like some dessert,'' Mother suggests.

Dela and Zakod exchange a look of mutual consent.

Zakod hands the bottle to Mother. She picks up Baby, sits down with him in her lap and gives him his bottle. Baby sucks
a bit, then less and less, and then lets go altogether. He falls asleep. Mother says, ``He only needed a
little bit more to help him on his way to dreamland.''

Dela says, ``This is the baby who was found after the explosion in the television building. You may have
heard about it. No one came to claim him. And no one wanted to adopt him. They said he might be a Balad. That~doesn't
bother us. They agreed to let us adopt him even though I {\dots}'' She points to herself, with a slight
shrug.

``He's so very beautifully taken care of,'' says Mother. ``He looks healthy and
content.''

``We bought him a crib,'' says Zakod, ``but he prefers this carriage in which he
was found.''

Mother says, ``Babies love to be nestled closely. Like chicks in a nest.''

Baby starts to whimper. She picks him up and rests him on her shoulder. ~He burps in his sleep. The three of them
chuckle.

``Doctor's orders,'' Dela comments warmly. ``You know his first tooth came up
last week.``~

``And you didn't catch a wink all night long,'' Mother guesses and adds, ``Babies
like to tell the whole world about each new tooth.''

``We had three daughters -'' Dela says. ``Their lives met with
disaster.''

``I weep with you with all my heart,'' says Mother. Gidal had told her about that tragedy.

Dela asks, ``Do you have children?''

``I had five children. Their lives met with disaster. My tears have dried up. What's the baby's
name?''

``We haven't decided on a name yet,'' Dela says. ``We call him 'Baby' for the
time being. Would you like to stay with us and help~with whatever is needed?''

``I would,'' Mother says. ``I can do everything. And also take care of
Baby.''

Zakod says, ``We have spare rooms downstairs.''

``I'm a Balad,'' says Mother her gaze lingering on Aera's hand-made pouch fixed on the front
of the carriage.

``I shall try and get you a permit,'' says Zakod following the Balad woman's eyes. He had
found in that pouch a faded photo of a man - face puffy, hair uncombed, eyes unfocused. Every once in a while, when
alone, he looks at this photo. He so wants it to be his beloved friend Saffia's face that little by little he has
convinced himself that it is.

Mother lays Baby back in the carriage. Once his sleep deepens, she'll go outside to take down the dry washing from the
line. Earlier, as she'd approached the house, she saw pale blue flowers fluttering there in the breeze. Welcoming her.


\end{document}
